\newcommand{\FigDIOBackground}{
\begin{figure}[tbp]
\centering
%\fbox{
\includegraphics[width=1.0\textwidth,trim=0 0 1cm 0.93cm,clip]{figs/backgrounds/Dio_BackgroundRateVsRuntime.pdf}
%}
\caption{\figlabel{bg:dio:rates}
The DIO background rate as a function of momentum threshold for different total running times.
Given a fixed the running time, the total number of stopped muons is also fixed, which in turn sets the DIO background rate for a given momentum threshold.
All signal acceptance parameters were held fixed, except for the efficiency of the momentum threshold, which when combined with the number of stopped muons determines the \ac{ses}.
The \ac{ses} is indicated in the number along the lines in units of \num{1e-17}.
}
\end{figure}
}

\newcommand{\FigDIOEndPointComparison}{
\begin{figure}[tbp]
\centering
\includegraphics[width=0.8\textwidth,trim=0 0 0 0,clip]{figs/backgrounds/CompareDIOEndpoints.pdf}
\caption{\figlabel{bg:dio:spectra}
Comparison of the various available end-point expansions.
The red and blue lines show the parametrisations reported in the literature, whilst the black shows the digitisation of the spectrum used in SimG4.
For this study, the more conservative parametrisation from the 2011 Czarnecki paper~\cite{Czarnecki2011} has been used.
}
\end{figure}
}

\newcommand{\FigRMCExperiments}{
\begin{figure}[tbp]
\centering
%\fbox{
\includegraphics[width=0.5\textwidth]{figs/backgrounds/RMC_Gorringe_ExperimentSummary.pdf}
%}
\caption{\figlabel{bg:rmc:experiments}
Summary of experimental values of the rate of \ac{RMC} producing photons with energy grater than 57~MeV, and the observed end-point, reproduced from~\cite{RevModPhys.76.31}.
The column lablled `$\alpha$' is the neutron excess for the element, determined by: $\alpha=(A-2Z)/Z$.
}
\end{figure}
}

\newcommand{\TabRMCEndPoints}{%
\begin{table}[tb]%
%\centering
\begin{tabular}{lSSS}%
\hline
Reaction & \multicolumn{1}{m{3cm}}{Atomic Mass of Daughter (u)} & \multicolumn{1}{m{2cm}}{$\Delta{}M$ (MeV/c$^{2}$)}&  \multicolumn{1}{m{2cm}}{$\max(E_e^\textrm{RMC})$ (MeV/c$^{2}$)} \\
\hline
${}^{27}$Al$(\mu,\gamma\nu){}^{27}  $Mg     & 26.984341 &  3.12  & 101.85 \\
${}^{27}$Al$(\mu,\gamma\nu2n){}^{26}$Mg     & 25.982593 &  9.56  &  95.41 \\
${}^{27}$Al$(\mu,\gamma\nu2n){}^{25}$Mg     & 24.985837 & 20.66  &  84.31 \\
${}^{27}$Al$(\mu,\gamma\nu{}p){}^{26}$Na    & 25.992633 & 18.13  &  87.37 \\
${}^{27}$Al$(\mu,\gamma\nu{}np){}^{25}$Na   & 24.989954 & 23.71  &  81.77 \\
${}^{27}$Al$(\mu,\gamma\nu{}d){}^{25}$Na    & 24.989954 & 21.49  &  84.00 \\
${}^{27}$Al$(\mu,\gamma\nu\alpha){}^{23}$Na & 22.994467 & 15.49  &  91.01 \\
\hline
\end{tabular}
\caption{\tablabel{bg:rmc:massDifferences}%
Several potential daughter nuclei of nuclear muon capture in \textsuperscript{27}Al.
The mass of \textsuperscript{27}Al is 26.98153863~$u$, and one $u$ is taken as 931.494061~MeV/c$^2$~\cite{PDG2014}.
All masses come from~\cite{AUDI20033}.}\end{table}%
\xspace}%

\newcommand{\FigRMCSimResults}{
\begin{figure}[tbp]
\centering
%\fbox{
\includegraphics[width=0.85\textwidth]{figs/backgrounds/RMC_simResults.pdf}
%}
\caption{\figlabel{bg:rmc:simulation}
Observed electrons from a simulation of \num{6e7} \ac{RMC} photons.
The overlaid spectrum is normalised arbitrarily to fit on the plot.
}
\end{figure}
}

\newcommand{\FigRPCData}{
\begin{figure}[btp]
\centering
\subfloat[][\figlabel{bg:rpc:data:ca}Calcium]  {\includegraphics[width=0.43\textwidth]{figs/backgrounds/RPC-data-calcium.png}}\hspace{0.2cm}%
\subfloat[][\figlabel{bg:rpc:data:mg}Magnesium]{\includegraphics[width=0.53\textwidth]{figs/backgrounds/RPC-data-magnesium.png}}
\caption{\figlabel{bg:rpc:data}
Spectrum of photons coming from \acf{RPC}~\cite{Bistirlich:1972jy}.
The spectrum of manesium, which is adjacent to aluminium on the periodic table, was used as the basis of these studies.
}
\end{figure}
}

\newcommand{\FigRPCSimulatedSpectrum}{
\begin{figure}[btp]
\centering
%\fbox{%
\includegraphics[width=0.73\textwidth,trim=1cm 0.5cm 2cm 1cm,clip]{figs/backgrounds/RPC_simulated_spectrum.pdf}%
%}
\caption{\figlabel{bg:rpc:spectrum}
Digitised (red) and smoothed (blue) spectrum of \ac{RPC} from magnesium (see \fig{bg:rpc:data:mg}) used as input to the Monte Carlo simulation.
}
\end{figure}
}

\newcommand{\FigPionStopDist}{
\begin{figure}[btp]
\centering
\subfloat[][\figlabel{bg:piStop:dist:x}X-direction]{\includegraphics[width=0.32\textwidth,trim=0.2cm 0 1cm 0.7cm,clip]{figs/backgrounds/Tidied_StoppedPi-X.pdf}}\hspace{0.1cm}%
\subfloat[][\figlabel{bg:piStop:dist:y}Y-direction]{\includegraphics[width=0.32\textwidth,trim=0.2cm 0 1cm 0.7cm,clip]{figs/backgrounds/Tidied_StoppedPi-Y.pdf}}\hspace{0.1cm}%
\subfloat[][\figlabel{bg:piStop:dist:z}Z-direction]{\includegraphics[width=0.32\textwidth,trim=0.2cm 0 1cm 0.7cm,clip]{figs/backgrounds/Tidied_StoppedPi-Z.pdf}}
\caption{\figlabel{bg:piStop:dist}
Stopping distributions of pions in the target.
These distributions have considerably different forms to the muon stopping distributions shown in \fig{sense:stops2D}, mostly due to the different momenta of muons and pions.
}
\end{figure}
}

\newcommand{\FigPiVsMuMomenta}{
\begin{figure}[btp]
\centering
%\fbox{%
\includegraphics[width=0.9\textwidth,trim=0 0.5cm 1.3cm 0.4cm,clip]{figs/backgrounds/Tidied_MuVsPiMomentum.pdf}%
%}
\caption{\figlabel{bg:piVsMu:momenta}
The momentum of muons and pions for those that reach the target area and those that actually stop.
It is clear how the pion momenta are in general higher, including those that stop, although the maximum stopping momentum for pions is similar to that of muons.
}
\end{figure}
}

\newcommand{\FigAntiprotonEndpoint}{
\begin{figure}[btp]
\centering
\subfloat[][\figlabel{bg:antiprotons:end-point:tungsten}Tungsten]{\includegraphics[width=0.49\textwidth,clip=true,trim=0 0 1cm 2cm]{figs/backgrounds/Antiproton_Tungsten_theta_lab.pdf}}%\hspace{0.5cm}%
\subfloat[][\figlabel{bg:antiprotons:end-point:carbon}Carbon    ]{\includegraphics[width=0.49\textwidth,clip=true,trim=0 0 1cm 2cm]{figs/backgrounds/Antiproton_Carbon_theta_lab}}
\caption{\figlabel{bg:antiprotons:end-point}
The kinematic end-point for antiproton production as a function of the out-going antiproton direction with respect to the incoming proton in the frame of the target nucleus (the lab frame).
The absolute end-point is only achieved when the nucleus and out-going protons recoils coherently.
}
\end{figure}
}

\newcommand{\FigAntiprotonFits}{
\begin{figure}[tbp]
\centering
\includegraphics[width=1.0\textwidth,trim=0 0 0.45cm 0,clip]{figs/backgrounds/AntiprotonFits.pdf}
\caption{\figlabel{bg:antiprotons:fits}
Piecewise fitting to experimental data and kinematic end-points.
Inlays show a zoom around the experimental data points.
}
\end{figure}
}

\newcommand{\FigAntiprotonAngularDependence}{
\begin{figure}[tbp]
\centering
\includegraphics[width=0.8\textwidth,trim=0 0 1.4cm 1cm,clip]{figs/backgrounds/AntiprotonAngularDependence.pdf}
\caption{\figlabel{bg:antiprotons:angular}
The angular dependence of the rate of antiproton emission, integrated over all momenta.
The different lines represent the different fits to the high momentum part of the spectrum.
The relationship given in~\cite{Boyarinov:1994tp} would suggest the data here should fit a straight line.
The dashed lines represent instead a quadratic fit to these points, which looks like a better fit.
For reweighting events the interpolated (straight solid) lines were used to be conservative.
}
\end{figure}
}

\newcommand{\TabAntiprotonRegions}{
\begin{table}[bp]
\centering
	\begin{tabular}{rccl}
		Region & Data source & Fitted Momentum Function & Total $\bar{p}$ per POT \\
\hline
                $0 \le \theta< 59\degree$ & 10\degree \cite{Kiselev:2012sj} &         & $5.32\times10^{-4}  $\\
                $59 \le \theta< 97\degree$ & 59\degree \cite{Kiselev:2012sj} &        & $2.80\times10^{-8 } $\\
                $97 \le \theta< 119\degree$ & 97\degree \cite{Boyarinov:1994tp} &     & $2.39\times10^{-12} $\\
                $119 \le \theta< 180\degree$ & 119\degree \cite{Boyarinov:1994tp} &   & $1.22\times10^{-12} $\\

\hline
\end{tabular}
\caption{\tablabel{bg:antiprotons:regions}
Regions and fits used to simulate antiproton production.  
The values in the final column are result of converting to rates per POT and integrating the differential cross-sections measured in \cite{Boyarinov:1994tp,Kiselev:2012sj}.
%integrated the fitted and extrapolated spectra and then integrates over the fitted angular dependence.
}
\end{table}
}

