\newcommand{\FigMuecCreation}{
\begin{figure}[bt]
\centering 
%\fbox{
%\input{figs/feynman/mu_to_e_gamma_via_SM-Wgamma.tex}
\subfloat[][\figlabel{muec:underlying}Underlying Process]{%
\includegraphics[width=0.32\textwidth,trim=-0.6cm 0 -0.6cm 0,clip]{figs/feynman/pdfs/mu_e_conversion.pdf}}\hspace{0.01\textwidth}%
\subfloat[][\figlabel{muec:atomSketch}Conversion from Ground State]{%
\includegraphics[width=0.30\textwidth]{figs/mueconv/MuEConversion-atom-sketch.pdf}}\hspace{0.01\textwidth}%
\subfloat[][\figlabel{muec:beamOnTgt}Muon Beam Stopped in Target]
\caption{\figlabel{muec:creation}
The underlying \mueconv process \protect\subref{fig:muec:underlying} occurs from the ground state of a muonic atom~\protect\subref{fig:muec:atomSketch}.
To produce the muonic atoms a beam of muons has to be stopped in a target~\protect\subref{fig:muec:beamOnTgt}.
}
%\footnote{though the author has failed to reproduce the stereoscopic effect with his own eyes}
\end{figure}
}

