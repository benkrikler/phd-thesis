\newcommand{\FigMuecCreation}{
\begin{figure}[bt]
\centering 
%\fbox{
\includegraphics[width=0.92\textwidth]{figs/mueconv/MuEConvSearchOverview.pdf}
%\input{figs/feynman/mu_to_e_gamma_via_SM-Wgamma.tex}
%\subfloat[][\figlabel{muec:underlying}Underlying Process]{%
%\includegraphics[width=0.32\textwidth,trim=-0.6cm 0 -0.6cm 0,clip]{figs/feynman/pdfs/mu_e_conversion.pdf}}\hspace{0.01\textwidth}%
%\subfloat[][\figlabel{muec:atomSketch}Conversion from Ground State]{%
%\includegraphics[width=0.30\textwidth]{figs/mueconv/MuEConversion-atom-sketch.pdf}}\hspace{0.01\textwidth}%
%\subfloat[][\figlabel{muec:beamOnTgt}Muon Beam Stopped in Target]{%
%\includegraphics[width=0.32\textwidth,trim=2.5cm 0.5cm 2.5cm 0.5cm,clip]{figs/mueconv/BeamOnTarget.png}}
%}
\caption{\figlabel{muec:creation}
A New Model introducing \ac{CLFV} generates \mueconv when connected to a nucleus.
Observing this requires many muonic atoms be formed and so \mueconv experiments progress by stopping an intense muon beam in a target and looking for electrons leaving with a specific energy.
}
%\footnote{though the author has failed to reproduce the stereoscopic effect with his own eyes}
\end{figure}
}

\newcommand{\FigMuecSindrumII}{
\begin{figure}[bt]
\centering 
%\fbox{
\includegraphics[width=0.99\textwidth]{figs/mueconv/SindrumII.pdf}
%}
\caption{\figlabel{muec:sindrum}
The \sindrumII experiment, which holds the current limit on \mueconv.
Left: the detector and target, with the muon beam produced from decay of a pion beam created by protons striking a target.
Right: the observed electron and positron energies and expected background and signal spectra.
Reproduced from~\cite{sindrum2006}.
}
%\footnote{though the author has failed to reproduce the stereoscopic effect with his own eyes}
\end{figure}
}

\newcommand{\FigMuonicXrays}{
\begin{figure}[t]
\centering
%\fbox{
\includegraphics[width=0.55\textwidth]{figs/mueconv/MuonicXrays_Hartmann.png}
%}
\caption{
The most intense line of the muonic atom atomic cascade, the $2p-1s$ transition, surrounded by the peaks of the muonic-magnesium Lyman series.
Reproduced from~\cite{Hartmann:1982uk}.
}
\figlabel{muec:alXrays}
\end{figure}
}

\newcommand{\FigDecayInOrbitSpectrum}{
\begin{figure}[t]
\centering
%\fbox{
\includegraphics[width=0.98\textwidth,trim=2.6cm 3.2cm 2.6cm 19.7cm,clip=true]{figs/detector/Czarnecki_2011_spectrum}
%}
\caption{
The spectrum of electrons produced by muon decay-in-orbit, according to Czarnecki \etal~\cite{Czarnecki2011}.
The two plots show the same data, but left is on a linear-linear scale whilst the right plot is on a log-lin scale which shows clearly the high-energy tail reaching up to the \mueconv signal energy of 104.97~MeV.
}
\figlabel{muec:dio}
\figlabel{detector:DIOSpectrum}
\end{figure}
}

\newcommand{\FigMuecMuCapture}{
\begin{figure}[tb]
\centering 
%\fbox{
\subfloat[][\figlabel{muec:mucap:}Protons from Al~\cite{Krane:1979wt}]{%
\includegraphics[height=0.23\textheight]{figs/mueconv/MuCap-Krane.pdf}}\hspace{0.01\textwidth}%
\subfloat[][\figlabel{muec:mucap:}Charged Particles from Si~\cite{Sobottka1968}]{%
\includegraphics[height=0.24\textheight,trim=0 1cm 0 0,clip]{figs/mueconv/MuCap-Sobottka.pdf}}\hspace{0.01\textwidth}%
\subfloat[][\figlabel{muec:mucap:}Total Rates from Al~\cite{Wyttenbach:1978rp}]
\caption{\figlabel{muec:mucap}
Selected experimental measurements of charged particle emission following muon capture, from the late 1960s to 1970s.
}
%\footnote{though the author has failed to reproduce the stereoscopic effect with his own eyes}
\end{figure}
}

