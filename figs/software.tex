\newcommand{\FigICEDUSTOverview}{
\begin{figure}[t]
\centering
%\fbox{
\includegraphics[width=1.00\textwidth,trim=0.85cm 0.5cm 0.5cm 0.5cm,clip=true]{figs/software/ICEDUST_structure}
%}
\caption{
Overview diagram for the ICEDUST framework.
Data produced from simulation or taken in the real experiment are treated identically through the calibration and onwards up to analysis.
}
\figlabel{software:ICEDUSTOverview}
\end{figure}
}

\newcommand{\FigNDTwoEighty}{
\begin{figure}[t]
\centering
\includegraphics[width=0.95\textwidth]{figs/software/ND280SoftwareDiagram}
\caption{
Overview diagram for the ND280 framework.
}
\figlabel{software:ND280}
\end{figure}
}

\newcommand{\FigSimulationOverview}{
\begin{figure}[b]
\centering
%\fbox{
\includegraphics[width=1.00\textwidth]{figs/software/Simulation_structure}
%}
\caption{
Diagram showing the stages used to simulate COMET.
The timing schematics on the right show how a simulated event is built up, firstly by producing many individual proton interactions with the production target,
then by transporting the secondary particles to produce energy deposits in the detector, which are then combined with the truth hits from other proton events to produce a realistic bunch structure.
Finally these bunch events are processed through the detector response simulation to produce fake waveforms and other detector read-outs.
}
\figlabel{software:SimulationOverview}
\end{figure}
}

