\newcommand{\TabOptimisationParameters}{
\begin{table}[tb]
\begin{tabular}{ll}
	\hline
	Region for optimisation & Approx.\ No.\  of parameters \\
	\hline
	Production target dimensions and location & $3+3$ \\
	Torus1 dipole field strength & 1 \\
	Torus2 dipole field strength & 1 \\
	Muon beam collimator shapes, position, and material & $3+1+1$ \\
	Stopping target shape and location & $4+3$ \\
	Beam blocker position, form, and material& $3+3+1$ \\
	Electron spectrometer dipole field strength& 1 \\
	DIO blockers in the spectrometer & $4$ \\
	\hline
	Approx. total number of parameters& 32 \\
	\hline
\end{tabular}
	\caption{\tablabel{optimisation:possible-parameters}%
	The list of parameters that can be optimised and an estimate for the number of parameters that this represents.
	In the case of the target, beam blocker and collimator shapes the number of parameters is only approximate; crudely speaking there is at least a width, length and height but in principle one could have a very irregular shape that cannot be parametrised by only three numbers, for example shapes that change as a function of distance along the beamline.
}
\end{table}
}

\newcommand{\FigOptimProdTgtLength}{
\begin{figure}[t]
\centering
\subfloat[][\figlabel{optimisation:ProdTgtSec:Length:Momentum:Muons}Muons]{\includegraphics[width=0.45\textwidth,trim=0 0 0 1.5cm,clip]{figs/optimisation/ProdTgtGeom/Length_mu-minus_momentum}}
\subfloat[][\figlabel{optimisation:ProdTgtSec:Length:Momentum:Pions}Pions]{\includegraphics[width=0.45\textwidth,trim=0 0 0 1.5cm,clip]{figs/optimisation/ProdTgtGeom/Length_pi-minus_momentum}}
\caption{
Change to momentum distributions at the entrance to the first 90 degrees of the bent muon beam solenoid for different target lengths.
Target length is given as half-length which is the Geant4 convention.  
}
\label{optimisation:ProdTgtSec:Length:Momentum}
\end{figure}

\begin{figure}[t]
\centering
\subfloat[][\figlabel{optimisation:ProdTgtSec:Length:Integral:Muons}Muons]{\includegraphics[width=0.45\textwidth,trim=0 0 0 1.5cm,clip]{figs/optimisation/ProdTgtGeom/Length_mu-minus_integral_toZero}}
\subfloat[][\figlabel{optimisation:ProdTgtSec:Length:Integral:Pions}Pions]{\includegraphics[width=0.45\textwidth,trim=0 0 0 1.5cm,clip]{figs/optimisation/ProdTgtGeom/Length_pi-minus_integral_toZero}}
\caption{\figlabel{optimisation:ProdTgtSec:Length:Integral}
Integrated muon and pion yields up to a certain momentum at the entrance to the first 90 degrees of the bent muon beam solenoid as a function of target length.
}
\end{figure}
\begin{figure}[t]
\centering
\subfloat[][\figlabel{optimisation:ProdTgtSec:Length:IntegralRatio:Muons}Muons]{\includegraphics[width=0.45\textwidth,trim=0 0 0 1.5cm,clip]{figs/optimisation/ProdTgtGeom/Length_mu-minus_integral_ratios}}
\subfloat[][\figlabel{optimisation:ProdTgtSec:Length:IntegralRatio:Pions}Pions]{\includegraphics[width=0.45\textwidth,trim=0 0 0 1.5cm,clip]{figs/optimisation/ProdTgtGeom/Length_pi-minus_integral_ratios}}
\caption{\figlabel{optimisation:ProdTgtSec:Length:IntegralRatio}
Change in the momentum distribution of muons and pions at the entrance to the first 90 degrees of the bent muon beam solenoid as a function of target length.
}
\end{figure}
}

\newcommand{\FigOptimProdTgtRad}{
\begin{figure}[t]
\centering
\subfloat[][\figlabel{optimisation:ProdTgtSec:Radius:Momentum:Muons}Muons]{\includegraphics[width=0.45\textwidth,trim=0 0 0 1.5cm,clip]{figs/optimisation/ProdTgtGeom/Radius_mu-minus_momentum}}
\subfloat[][\figlabel{optimisation:ProdTgtSec:Radius:Momentum:Pions}Pions]{\includegraphics[width=0.45\textwidth,trim=0 0 0 1.5cm,clip]{figs/optimisation/ProdTgtGeom/Radius_pi-minus_momentum}}
\caption{
Change to momentum distributions at the entrance to the first 90 degrees of the bent muon beam solenoid for different target radii.
}
\label{optimisation:ProdTgtSec:Radius:Momentum}
\end{figure}

\begin{figure}[t]
\centering
\subfloat[][\figlabel{optimisation:ProdTgtSec:Radius:Integral:Muons}Muons]{\includegraphics[width=0.45\textwidth,trim=0 0 0 1.5cm,clip]{figs/optimisation/ProdTgtGeom/Radius_mu-minus_integral_toZero}}
\subfloat[][\figlabel{optimisation:ProdTgtSec:Radius:Integral:Pions}Pions]{\includegraphics[width=0.45\textwidth,trim=0 0 0 1.5cm,clip]{figs/optimisation/ProdTgtGeom/Radius_pi-minus_integral_toZero}}
\caption{\figlabel{optimisation:ProdTgtSec:Radius:Integral}
Integrated muon and pion yields up to a certain momentum at the entrance to the first 90 degrees of the bent muon beam solenoid as a function of target radius.
}
\end{figure}
\begin{figure}[t]
\centering
\subfloat[][\figlabel{optimisation:ProdTgtSec:Radius:IntegralRatio:Muons}Muons]{\includegraphics[width=0.45\textwidth,trim=0 0 0 1.5cm,clip]{figs/optimisation/ProdTgtGeom/Radius_mu-minus_integral_ratios}}
\subfloat[][\figlabel{optimisation:ProdTgtSec:Radius:IntegralRatio:Pions}Pions]{\includegraphics[width=0.45\textwidth,trim=0 0 0 1.5cm,clip]{figs/optimisation/ProdTgtGeom/Radius_pi-minus_integral_ratios}}
\caption{\figlabel{optimisation:ProdTgtSec:Radius:IntegralRatio}
Change in the momentum distribution of muons and pions at the entrance to the first 90 degrees of the bent muon beam solenoid as a function of target radius.
}
\end{figure}
}

\newcommand{\FigOptimMuBeamDipoleMuStops}{
\begin{figure}[b]
\centering
\includegraphics[width=0.85\textwidth,trim=0 0.5cm 0 1.5cm,clip]{figs/optimisation/MuonBeamDipoles/Tidied_stopped_muons.pdf}
\caption{\figlabel{optim:muBeamDipole:stoppedMu}
	Muon stopping rate as a function of the two dipole field strengths (given relative to the \phaseI design specification).
	A clear anti-correlation is visible which is discussed in the text.
}
\end{figure}
}

\newcommand{\FigOptimMuBeamDipolePiStops}{
\begin{figure}[t]
\centering
\includegraphics[width=0.85\textwidth,trim=0 0.5cm 0 1.5cm,clip]{figs/optimisation/MuonBeamDipoles/Tidied_stopped_pions.pdf}
\caption{\figlabel{optim:muBeamDipole:stoppedPi}
	Pion stopping rate as a function of the two dipole field strengths (given relative to the \phaseI design specification).
At the level of statistics used to generate each point, no clear trend is obvious.
Empty squares are those where no pions stopped in the run.
}
\end{figure}
}
