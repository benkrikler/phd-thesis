\newcommand{\FigDIOBackground}{
\begin{figure}[tbp]
\centering
%\fbox{
\includegraphics[width=1.0\textwidth,trim=0 0 1cm 0.93cm,clip]{figs/backgrounds/Dio_BackgroundRateVsRuntime.pdf}
%}
\caption{\figlabel{bg:dio:rates}
The DIO background rate as a function of momentum threshold for different total running times.
Given a fixed running time, the total number of stopped muons is also fixed, which in turn sets the signal sensitivity and the DIO background rate.
All signal acceptance parameters were held fixed, except for the efficiency of the momentum threshold, which, when combined with the number of stopped muons, determines the \ac{ses}.
The \ac{ses} is indicated in the number along the lines in units of \num{1e-17}.
}
\end{figure}
}

\newcommand{\FigDIOEndPointComparison}{
\begin{figure}[tbp]
\centering
\includegraphics[width=0.8\textwidth,trim=0 0 0 0,clip]{figs/backgrounds/CompareDIOEndpoints.pdf}
\caption{\figlabel{bg:dio:spectra}
Comparison of the various available end-point expansions.
The red and blue lines show the parametrisations reported in the literature, whilst the black shows the digitisation of the spectrum used in SimG4.
For this study, the more conservative parametrisation from the 2011 Czarnecki paper~\cite{Czarnecki2011} has been used.
}
\end{figure}
}

\newcommand{\FigRMCExperiments}{
\begin{figure}[tbp]
\centering
%\fbox{
\includegraphics[width=0.5\textwidth]{figs/backgrounds/RMC_Gorringe_ExperimentSummary.pdf}
%}
\caption{\figlabel{bg:rmc:experiments}
Summary of experimental values of the rate of \ac{RMC} producing photons with energy greater than 57~MeV, $R_\gamma$, and the observed end-point, $k_\textrm{max}$, redacted from~\cite{RevModPhys.76.31}.
The column lablled `$\alpha$' is the neutron excess for the element, determined by: $\alpha=(A-2Z)/Z$.
}
\end{figure}
}

\newcommand{\TabRMCEndPoints}{%
\begin{table}[tb]%
%\centering
\begin{tabular}{lS[table-format=2.6]SS}%
\hline
Reaction & \multicolumn{1}{C{3cm}}{Atomic Mass of Daughter (u)} & \multicolumn{1}{C{2cm}}{$\Delta{}M$ (MeV/c$^{2}$)}&\multicolumn{1}{C{2cm}}{$\max(E_e^\textrm{RMC})$ (MeV/c$^{2}$)}\\
\hline
${}^{27}$Al$(\mu,\gamma\nu){}^{27}  $Mg     & 26.984341 &  3.12  & 101.85 \\
${}^{27}$Al$(\mu,\gamma\nu2n){}^{26}$Mg     & 25.982593 &  9.56  &  95.41 \\
${}^{27}$Al$(\mu,\gamma\nu2n){}^{25}$Mg     & 24.985837 & 20.66  &  84.31 \\
${}^{27}$Al$(\mu,\gamma\nu{}p){}^{26}$Na    & 25.992633 & 18.13  &  87.37 \\
${}^{27}$Al$(\mu,\gamma\nu{}np){}^{25}$Na   & 24.989954 & 23.71  &  81.77 \\
${}^{27}$Al$(\mu,\gamma\nu{}d){}^{25}$Na    & 24.989954 & 21.49  &  84.00 \\
${}^{27}$Al$(\mu,\gamma\nu\alpha){}^{23}$Na & 22.994467 & 15.49  &  91.01 \\
\hline
\end{tabular}
\caption{\tablabel{bg:rmc:massDifferences}%
Several potential daughter nuclei of nuclear muon capture in \textsuperscript{27}Al.
The mass of \textsuperscript{27}Al is 26.98153863~$u$, and one $u$ is taken as 931.494061~MeV/c$^2$~\cite{PDG2014}.
All masses come from~\cite{AUDI20033}.}\end{table}%
\xspace}%

\newcommand{\FigRMCSimResults}{
\begin{figure}[tbp]
\centering
%\fbox{
\includegraphics[width=0.85\textwidth]{figs/backgrounds/RMC_simResults.pdf}
%}
\caption{\figlabel{bg:rmc:simulation}
Observed electrons from a simulation of \num{6e7} \ac{RMC} photons.
The overlaid spectrum is normalised arbitrarily to fit on the plot.
}
\end{figure}
}

\newcommand{\FigRPCData}{
\begin{figure}[btp]
\centering
\subfloat[][\figlabel{bg:rpc:data:ca}Calcium]  {\includegraphics[width=0.43\textwidth]{figs/backgrounds/RPC-data-calcium.png}}\hspace{0.2cm}%
\subfloat[][\figlabel{bg:rpc:data:mg}Magnesium]{\includegraphics[width=0.53\textwidth]{figs/backgrounds/RPC-data-magnesium.png}}
\caption{\figlabel{bg:rpc:data}
Spectrum of photons coming from \acf{RPC}~\cite{Bistirlich:1972jy}.
The spectrum of manesium, which is adjacent to aluminium on the periodic table, was used as the basis of these studies.
}
\end{figure}
}

\newcommand{\FigRPCSimulatedSpectrum}{
\begin{figure}[btp]
\centering
%\fbox{%
\includegraphics[width=0.73\textwidth,trim=1cm 0.5cm 2cm 1cm,clip]{figs/backgrounds/RPC_simulated_spectrum.pdf}%
%}
\caption{\figlabel{bg:rpc:spectrum}
Digitised (red) and smoothed (blue) spectrum of \ac{RPC} from magnesium (see \fig{bg:rpc:data:mg}) used as input to the Monte Carlo simulation.
}
\end{figure}
}

\newcommand{\FigPionStopDist}{
\begin{figure}[btp]
\centering
\subfloat[][\figlabel{bg:piStop:dist:x}X-direction]{\includegraphics[width=0.32\textwidth,trim=0.2cm 0 1cm 0.7cm,clip]{figs/backgrounds/Tidied_StoppedPi-X.pdf}}\hspace{0.1cm}%
\subfloat[][\figlabel{bg:piStop:dist:y}Y-direction]{\includegraphics[width=0.32\textwidth,trim=0.2cm 0 1cm 0.7cm,clip]{figs/backgrounds/Tidied_StoppedPi-Y.pdf}}\hspace{0.1cm}%
\subfloat[][\figlabel{bg:piStop:dist:z}Z-direction]{\includegraphics[width=0.32\textwidth,trim=0.2cm 0 1cm 0.7cm,clip]{figs/backgrounds/Tidied_StoppedPi-Z.pdf}}
\caption{\figlabel{bg:piStop:dist}
Stopping distributions of pions in the target.
These distributions have considerably different forms to the muon stopping distributions shown in \fig{sense:stops}, mostly due to the different momenta of muons and pions.
}
\end{figure}
}

\newcommand{\FigPiVsMuMomenta}{
\begin{figure}[btp]
\centering
%\fbox{%
\includegraphics[width=0.9\textwidth,trim=0 0.5cm 1.3cm 0.4cm,clip]{figs/backgrounds/Tidied_MuVsPiMomentum.pdf}%
%}
\caption{\figlabel{bg:piVsMu:momenta}
The momentum of muons and pions for those that reach the target area and those that actually stop.
It is clear how the pion momenta are in general higher, including those that stop, although the maximum stopping momentum for pions is similar to that of muons.
}
\end{figure}
}

\newcommand{\FigRPCSimResults}{
\begin{figure}[btp]
\centering
%\fbox{
\subfloat[][\figlabel{bg:rpc:sim:momVtime}Momentum Vs.\ Time]
%\begin{minipage}[b]{0.45\textwidth}
%\subfloat[][\figlabel{bg:rpc:sim:time}Arrival Time]{%
%\includegraphics[width=\textwidth,trim=0.9cm 0.3cm 1cm 0.5cm,clip]{figs/backgrounds/Tidied_RPC_sim_time.pdf}}\\
%\subfloat[][\figlabel{bg:rpc:sim:mom}Momentum]{%
%\includegraphics[width=\textwidth,trim=0.9cm 0.3cm 1cm 0.5cm,clip]{figs/backgrounds/Tidied_RPC_sim_mom.pdf}}%
%\end{minipage}\hspace{1ex}
\hspace{1em}%
\subfloat[][\figlabel{bg:rpc:sim:time}Arrival Time of High-$p$ Electrons]{%
\includegraphics[width=0.49\textwidth,trim=0 0 0 2.8cm,clip]{figs/backgrounds/RPC_lifetime.png}}%
%\fbox{
%}
\caption{\figlabel{bg:rpc:sim}
Detection of secondaries from RPC photons in the target.
Although many high-momentum electrons are detected \protect\subref{fig:bg:rpc:sim:momVtime}, they are all well before the time-gated detected window \protect\subref{fig:bg:rpc:sim:time}.
}
\end{figure}
}

\newcommand{\FigAntiprotonMeco}{
\begin{figure}[tbp]
\centering
\includegraphics[width=0.6\textwidth]{figs/backgrounds/Antiproton_Meco24_energy.pdf}
\caption{\figlabel{bg:antiprotons:meco24}
Variation in the antiproton production rate as a function of incident proton energy, according to Meco note 24~\cite{Meco024} and used in the COMET TDR~\cite{TDR2016}.
For reference, protons with 8~GeV kinetic energy have 8.89~GeV/c momentum, whilst with 10.14~GeV kinetic energy their momentum is 11.038~GeV/c.
The vertical coloured lines have been added to indicate these energies, whilst the horizontal bands show the range of predicted cross sections for the models of proton-nucleon and proton-nucleus interaction.
}
\end{figure}
}

\newcommand{\FigAntiprotonData}{
\begin{figure}[tbp]
\centering
\includegraphics[width=1.0\textwidth,trim=0 0 0 0,clip]{figs/backgrounds/Antiproton_RatePerPOT_data.pdf}
\caption{\figlabel{bg:antiprotons:data}
Experimental data for antiproton production rates for 10~GeV protons~\cite{Boyarinov:1994tp,Kiselev:2012sj}.
Each line represents the cross section obtained for the four different target materials covered in those papers, scaled to match the number of nucleons of tungsten and with the additional factors of \eq{bg:antiprotons:rate} included.
}
\end{figure}
}

\newcommand{\FigAntiprotonEndpoint}{
\begin{figure}[btp]
\centering
\subfloat[][\figlabel{bg:antiprotons:end-point:tungsten}Tungsten]{\includegraphics[width=0.49\textwidth,clip=true,trim=0 0 1cm 1.7cm]{figs/backgrounds/Antiproton_Tungsten_theta_lab.pdf}}%\hspace{0.5cm}%
\subfloat[][\figlabel{bg:antiprotons:end-point:carbon}Carbon    ]{\includegraphics[width=0.49\textwidth,clip=true,trim=0 0 1cm 1.7cm]{figs/backgrounds/Antiproton_Carbon_theta_lab}}
\caption{\figlabel{bg:antiprotons:end-point}
The kinematic end-point for antiproton production as a function of the outgoing antiproton direction with respect to the incoming proton in the frame of the target nucleus (the lab frame).
The absolute end-point is only achieved when the nucleus and outgoing protons recoils coherently.
}
\end{figure}
}

\newcommand{\FigAntiprotonFits}{
\begin{figure}[tbp]
\centering
%	\fbox{
\includegraphics[width=1.0\textwidth]{figs/backgrounds/AntiprotonFits.png}
%}
\caption{\figlabel{bg:antiprotons:fits}
Piecewise fitting to experimental data and kinematic end-points.
Inlays show a zoom around the experimental data points.
}
\end{figure}
}

\newcommand{\FigAntiprotonAngularDependence}{
\begin{figure}[b]
\centering
\includegraphics[width=0.8\textwidth,trim=0 0 1.4cm 1cm,clip]{figs/backgrounds/AntiprotonAngularDependence.pdf}
\caption{\figlabel{bg:antiprotons:angular}
The angular dependence of the rate of antiproton emission, integrated over all momenta.
The different lines represent the different fits to the high momentum part of the spectrum.
The relationship given in~\cite{Boyarinov:1994tp} would suggest the data here should fit a straight line.
The dashed lines represent instead a quadratic fit to these points, which looks like a better fit.
For reweighting events the interpolated (straight solid) lines were used to be conservative.
}
\end{figure}
}

\newcommand{\TabAntiprotonRegions}{
\begin{table}[t]
\centering
\sisetup{table-number-alignment=right,table-format=1.2e3}%
\begin{tabular}{a|r|SS}
\multicolumn{1}{c|}{\multirow{2}{*}{Region}} & \multirow{2}{*}{Data Source} & \multicolumn{2}{c}{Total $\bar{p}$ per POT in this region} \\ 
                                             &                              & {Linear Tail}       &  {Exponential Tail} \\ 
\hline
                  0-59\degree    & 10\degree \cite{Kiselev:2012sj}    & 9.13e-05 & 5.26e-05 \\ 
                  59-97\degree   & 59\degree \cite{Kiselev:2012sj}    & 2.64e-08 & 4.17e-09 \\ 
                  97-119\degree  & 97\degree \cite{Boyarinov:1994tp}  & 3.40e-12 & 1.74e-12 \\ 
                  119-180\degree & 119\degree \cite{Boyarinov:1994tp} & 2.58e-12 & 5.71e-13 \\ 
\hline
\end{tabular}
\caption{\tablabel{bg:antiprotons:regions}
Angular regions and the source of the data used to build the momentum spectrum for that region.
The integrated rate for the two different high-momentum tail descriptions are also given.
Note that these values do \emph{not} contain the correction for the different incident proton energies;
for the COMET proton beam the antiproton yield is expected to be a factor 0.12 times those given here.
%The values in the final column are result of converting to rates per POT and integrating the differential cross-sections measured in \cite{Boyarinov:1994tp,Kiselev:2012sj}.
%integrated the fitted and extrapolated spectra and then integrates over the fitted angular dependence.
}
\end{table}
}

\newcommand{\FigAntiprotonSimHeightsTwoDPbar}{%
\begin{figure}[ph]%
\centering %
\subfloat[][\figlabel{bg:antiprotons:sim:2D-antip:10}Production between 0 and 59\degree]{%
\includegraphics[width=1\textwidth,trim=3.7cm 0.3cm 1.8cm 0.8cm,clip]{figs/backgrounds/Antiproton_height2D_antiproton_10.png}}\\%
\subfloat[][\figlabel{bg:antiprotons:sim:2D-antip:59}Production between 59 and 97\degree]{%
\includegraphics[width=1\textwidth,trim=3.7cm 0.3cm 1.8cm 0.8cm,clip]{figs/backgrounds/Antiproton_height2D_antiproton_59.png}}\\%
\subfloat[][\figlabel{bg:antiprotons:sim:2D-antip:97}Production between 97 and 119\degree]{%
\includegraphics[width=1\textwidth,trim=3.7cm 0.3cm 1.8cm 0.8cm,clip]{figs/backgrounds/Antiproton_height2D_antiproton_97.png}}\\%
\subfloat[][\figlabel{bg:antiprotons:sim:2D-antip:119}Production between 119 and 180\degree]{%
\includegraphics[width=1\textwidth,trim=3.7cm 0.3cm 1.8cm 0.8cm,clip]{figs/backgrounds/Antiproton_height2D_antiproton_119.png}}%
\caption{
The heights of antiprotons passing along the beamline for the four different angular regions of productions.
Each antiproton trajectory is weighted by the probability of producing an antiproton at this angle.
The colour scale on all these plots is the same.%
\figlabel{bg:antiprotons:sim:2D-antip}}%
\end{figure}%
\xspace}

\newcommand{\FigAntiprotonSimHeightsTwoDPiMin}{%
\begin{figure}[ph]%
\centering %
\subfloat[][\figlabel{bg:antiprotons:sim:2D-pi:10}Production between 0 and 59\degree]{%
\includegraphics[width=1\textwidth,trim=3.7cm 0.3cm 1.8cm 0.8cm,clip]{figs/backgrounds/Antiproton_height2D_pi-_10.png}}\\%
\subfloat[][\figlabel{bg:antiprotons:sim:2D-pi:59}Production between 59 and 97\degree]{%
\includegraphics[width=1\textwidth,trim=3.7cm 0.3cm 1.8cm 0.8cm,clip]{figs/backgrounds/Antiproton_height2D_pi-_59.png}}\\%
\subfloat[][\figlabel{bg:antiprotons:sim:2D-pi:97}Production between 97 and 119\degree]{%
\includegraphics[width=1\textwidth,trim=3.7cm 0.3cm 1.8cm 0.8cm,clip]{figs/backgrounds/Antiproton_height2D_pi-_97.png}}\\%
\subfloat[][\figlabel{bg:antiprotons:sim:2D-pi:119}Production between 119 and 180\degree]{%
\includegraphics[width=1\textwidth,trim=3.7cm 0.3cm 1.8cm 0.8cm,clip]{figs/backgrounds/Antiproton_height2D_pi-_119.png}}%
\caption{
The heights of secondary pions passing along the beamline produced from antiprotons in each of the four different angular regions of productions.
Each trajectory is weighted by the probability of producing the parent antiproton in its initial direction at the target.
\figlabel{bg:antiprotons:sim:2D-pi}}%
\end{figure}%
\xspace}%

\newcommand{\FigAntiprotonSimFluxes}{
\begin{figure}[b]
\centering 
\subfloat[][\figlabel{bg:antiprotons:sim:fluxes:antip}Unweighted Antiproton Survival Probability]{
\includegraphics[width=1\textwidth,trim=0.7cm 0 1.9cm 0.2cm,clip]{figs/backgrounds/Antiproton_fluxes.pdf}}\\
\subfloat[][\figlabel{bg:antiprotons:sim:fluxes:pion}Unweighted Secondary Pion Transport Probability]{
\includegraphics[width=1\textwidth,trim=0.7cm 0 1.9cm 0.2cm,clip]{figs/backgrounds/Antiproton_fluxes-pions.pdf}}
\caption{\figlabel{bg:antiprotons:sim:fluxes}
The surivival probability of antiprotons and secondaries pions per antiproton produced in the target as a function of distance along the beamline.
These plots are not weighted by the probability that an antiproton is produced at a particular angle.
From left to right the vertical gray lines indicate the production target, Torus1 entrance, and the Torus2 exit.
}
\end{figure}
}

\newcommand{\FigAntiprotonSimTime}{
\begin{figure}[b!]
\centering 
\subfloat[][\figlabel{bg:antiprotons:sim:time:antip}Timing of Antiprotons]{
\includegraphics[width=0.485\textwidth,trim=0.5cm 0.9cm 0.5cm 0.9cm,clip]{figs/backgrounds/Antiproton_timing_antiprotons.pdf}}
\hspace{1ex}\subfloat[][\figlabel{bg:antiprotons:sim:time:pion}Timing of Secondary Pions]{
\includegraphics[width=0.485\textwidth,trim=0.5cm 0.9cm 0.5cm 0.9cm,clip]{figs/backgrounds/Antiproton_timing_pions.pdf}}
\caption{\figlabel{bg:antiprotons:sim:time}
%\CHECK{Add pion stopping timing to RPC section to be able to compare to it here}
	The arrival time of antiprotons~\protect\subref{fig:bg:antiprotons:sim:time:antip} and pions~\protect\subref{fig:bg:antiprotons:sim:time:pion}
	at various points along the beamline and for the different initial antiproton directions.
Note that the x-axis scales are different.
Whilst the timing of antiprotons themselves is very delayed, the timing of secondary pions, which are produced predominantly at the production target, is relatively prompt and will be effective at suppressing the induced backgrounds.
}
\end{figure}
}

%\newcommand{\FigAntiprotonSimFluxes}{
%\begin{figure}[ph]
%\centering 
%\subfloat[][\figlabel{bg:antiprotons:sim:fluxes:10}Production between 0 and 59\degree]{
%\includegraphics[width=1\textwidth,trim=0.7cm 0 1.9cm 0.2cm,clip]{figs/backgrounds/Antiproton_flux_10.pdf}}\\
%\subfloat[][\figlabel{bg:antiprotons:sim:fluxes:59}Production between 59 and 97\degree]{
%\includegraphics[width=1\textwidth,trim=0.7cm 0 1.9cm 0.2cm,clip]{figs/backgrounds/Antiproton_flux_59.pdf}}\\
%\subfloat[][\figlabel{bg:antiprotons:sim:fluxes:97}Production between 97 and 119\degree]{
%\includegraphics[width=1\textwidth,trim=0.7cm 0 1.9cm 0.2cm,clip]{figs/backgrounds/Antiproton_flux_97.pdf}}\\
%\subfloat[][\figlabel{bg:antiprotons:sim:fluxes:119}Production between 119 and 180\degree]{
%\includegraphics[width=1\textwidth,trim=0.7cm 0 1.9cm 0.2cm,clip]{figs/backgrounds/Antiproton_flux_119.pdf}}
%\caption{\figlabel{bg:antiprotons:sim:fluxes}
%The surivival probability of antiprotons and their secondaries per antiproton produced in the target as a function of distance along the beamline.
%From left to right the vertical magenta lines indicate the production target, Torus1 entrance, and the Torus2 exit.
%}
%\end{figure}
%}

\newcommand{\FigAntiprotonSimPiMom}{
\begin{figure}[tb]
\centering 
\includegraphics[width=0.9\textwidth,trim=2.0cm 0 0 0,clip]{figs/backgrounds/Antiproton_pion_momentum.pdf}
\caption{\figlabel{bg:antiprotons:sim:piMom}
Momentum of pions passing the exit of Torus1 (90\degree around the bent muon beam transport solenoid) compared to pions in the main muon beam (which is arbitrarily normalised).
}
\end{figure}
}

\newcommand{\HeaderPi}[1]{\multicolumn{1}{#1}{Torus1 $\pi^-$}}
\newcommand{\HeaderPBar}[1]{\multicolumn{1}{#1}{$\bar{p}$ Stop}}
%\newcommand{\TabAntiprotonResults}{
%\begin{table}[tb]
%\centering
%\begin{tabular}{a|SS|SS|SS|}
%\multicolumn{1}{c|}{Region} & \multicolumn{2}{c|}{Observed Events} & \multicolumn{2}{c|}{Weighted Mean per $\bar{p}$}& \multicolumn{2}{c}{Rate per \ac{POT}}  \\
%\multicolumn{1}{c|}{}       & \HeaderPi{r}    & \HeaderPBar{r|}    & \HeaderPi{r}      & \HeaderPBar{r|}      &\HeaderPi{r}     & \HeaderPBar{r}                  \\
%\hline
%   0-59\degree &50&0&2.5e-4&& \\
%  59-97\degree &31&0&1.6e-4&& \\
% 97-119\degree &36&0&1.8e-4&& \\
%119-180\degree &64&9&3.2e-4&& \\
%\hline
%\multicolumn{1}{c|}{Total} & & & & & \\
%\hline
%\end{tabular}
%\caption{\tablabel{bg:antiprotons:results}
%Results of the antiproton simulation.
%`Torus1 $\pi^-$' are the pions that pass the exit of Torus1, which is 90\degree round the muon beamline.
%`$\bar{p}$ Stop' refers to the number of antiprotons that stop in the muon Stopping Target.
%The weighted mean is the sum of the observed events weighted by the production probability given the initial antiproton direction, divided by the total number of input antiprotons.
%Finally the Rate per \ac{POT} is weighted mean scaled by the number of antiprotons produced for this region per POT (last column of \tab{bg:antiprotons:regions}).
%}
%\end{table}
%}

%\newcommand{\TabAntiprotonResultsPiSecond}{
%\begin{table}[tb]
%\centering
%\begin{tabular}{a|SSS}
%\multicolumn{1}{c|}{Region} & \multicolumn{1}{c}{Observed Events} & \multicolumn{1}{c}{Weighted Mean per $\bar{p}$}& \multicolumn{1}{c}{Rate per \ac{POT}}  \\
%\hline
%   0-59\degree &50&2.5e-4&& \\
%  59-97\degree &31&1.6e-4&& \\
% 97-119\degree &36&1.8e-4&& \\
%119-180\degree &64&3.2e-4&& \\
%\hline
%\multicolumn{1}{c|}{Total} & & & & & \\
%\hline
%\end{tabular}
%\caption{\tablabel{bg:antiprotons:results}
%Results of the antiproton simulation.
%`Torus1 $\pi^-$' are the pions that pass the exit of Torus1, which is 90\degree round the muon beamline.
%`$\bar{p}$ Stop' refers to the number of antiprotons that stop in the muon Stopping Target.
%The weighted mean is the sum of the observed events weighted by the production probability given the initial antiproton direction, divided by the total number of input antiprotons.
%Finally the Rate per \ac{POT} is weighted mean scaled by the number of antiprotons produced for this region per POT (last column of \tab{bg:antiprotons:regions}).
%}
%\end{table}
%}

%\newcommand{\TabAntiprotonResultsTrans}{
%\begin{table}[p]
%\centering
%\begin{tabular}{a|SSS|SSS|S|}
% &\multicolumn{3|}{c}{Raw count} &\multicolumn{3|}{c}{Weighted Probability} & \\
% &\multicolumn{1}{p{0.6cm}}{Entr.}&\multicolumn{1}{p{0.6cm}}{Mid.}&\multicolumn{1}{p{0.6cm}|}{Target}&\multicolumn{1}{p{0.6cm}}{Entr.}&\multicolumn{1}{p{0.6cm}}{Mid.}&\multicolumn{1}{p{0.6cm}}{Target}&\multicolumn{1}{p{0.6cm}}{$P(\textrm{Target}|\textrm{Entr.}$}\\
%\hline
%\multicolumn{1}{l}{Pions} & \multicolumn{7}{p{6cm}}{}\\
%   0-59\degree    & 16943  & 1452  & 1  & 5.77E-06 & 5.10E-07 & 9.98E-11 & 1.73E-05 \\ 
%   59-97\degree   & 87230  & 7157  & 18 & 4.64E-09 & 4.06E-10 & 2.99E-13 & 6.45E-05 \\ 
%   97-119\degree  & 157385 & 13041 & 25 & 4.10E-14 & 3.59E-15 & 5.02E-18 & 1.22E-04 \\ 
%   119-180\degree & 222486 & 12997 & 30 & 1.25E-17 & 9.59E-19 & 1.91E-21 & 1.53E-04 \\ 
%\hline
%\multicolumn{1}{l}{Antiprotons} & \multicolumn{7}{p{6cm}}{}\\
%0-59\degree    & 7      & 3     & 0    & 1.40E-09 & 5.14E-11 & 3.23E-12                  &          \\ 
%59-97\degree   & 270    & 58    & 8    & 1.70E-12 & 4.81E-13 & 1.16E-14                  & 2.41E-02 \\ 
%97-119\degree  & 2907   & 830   & 114  & 1.10E-16 & 3.31E-17 & 5.17E-19                  & 1.56E-02 \\ 
%119-180\degree & 278105 & 20787 & 2237 & 2.27E-19 & 5.20E-20 & 7.74E-21                  & 1.49E-01 \\ 
%               &        &       &      &          &          & \multicolumn{1}{c}{Mean=} & 6.29E-02 \\ 
%%\hline
%\end{tabular}
%\end{table}
%}

\newcommand{\TabAntiprotonResultsPiSecond}{%
\sisetup{table-number-alignment=right, table-format=2.2e2}%
\begin{table}[p]%
\centering%
%\begin{adjustwidth}{-0.7cm}{}%
\begin{tabular}{lr|S!{\VRule}S!{\VRule}S!{\VRule}S|}%
\multicolumn{2}{c|}{\multirow{2}{3cm}{Rates for Secondary $\pi^-$}} & \multicolumn{4}{c|}{Secondary $\pi^-$ from Angular Region for Antiproton Production} \\ 
                              &  & \multicolumn{1}{a!{\VRule}}{0-59\degree} & \multicolumn{1}{a!{\VRule}}{59-97\degree} & \multicolumn{1}{a!{\VRule}}{97-119\degree} & \multicolumn{1}{a|}{119-180\degree} \\ 
\hline
\multirow{3}{1.6cm}{Raw Counts} & Torus1 & \multicolumn{1}{r!{\VRule}}{16943} & \multicolumn{1}{r!{\VRule}}{87230} & \multicolumn{1}{r!{\VRule}}{157385} & \multicolumn{1}{r|}{222486} \\ 
                               % & TS3 & \multicolumn{1}{r!{\VRule}}{1452}  & \multicolumn{1}{r!{\VRule}}{7157}  & \multicolumn{1}{r!{\VRule}}{13041}  & \multicolumn{1}{r|}{12997}  \\ 
                                & Torus2 & \multicolumn{1}{r!{\VRule}}{208}   & \multicolumn{1}{r!{\VRule}}{1106}  & \multicolumn{1}{r!{\VRule}}{1995}   & \multicolumn{1}{r|}{1847}   \\ 
                                & Target & \multicolumn{1}{r!{\VRule}}{1}     & \multicolumn{1}{r!{\VRule}}{18}    & \multicolumn{1}{r!{\VRule}}{25}     & \multicolumn{1}{r|}{30}     \\ 
\hline
\multirow{3}{1.6cm}{Weighted Mean} & Torus1  &6.93E-07 & 5.57E-10 & 4.92E-15 & 1.50E-18\\ 
                                %  & TS3  &6.12E-08 & 4.87E-11 & 4.31E-16 & 1.15E-19 \\
                                  & Torus2  &1.03E-08 & 7.11E-12 & 6.95E-17 & 1.47E-20\\ 
                                  & Target  &1.20E-11 & 3.59E-14 & 6.02E-19 & 2.29E-22\\ 
%\hline
%\multicolumn{2}{r|}{$P(\textrm{Target}|\textrm{Torus2})$}  & 0.039 & 0.040 &  0.022 & 0.017 \\
\hline
\end{tabular}%
%\end{adjustwidth}%
\caption{
Secondary pion fluxes from antiprotons observed at key points along the beamline.
See caption to \tab{bg:antiprotons:sim:antip} for a description of the column and row contents.
%The raw counts are the total observed events from the simulation of 80M antiprotons for each of the four angular regions.
%The weighted sum then shows the weighted sum over every particle passing the point, where the weight is determined as described by equation~\eq{bg:antiprotons:factorisation}.
\tablabel{bg:antiprotons:sim:pi}}
\end{table}\xspace}

\newcommand{\TabAntiprotonResultsAntip}{%
\sisetup{table-number-alignment=right, table-format=2.2e2}%
\begin{table}[p]%
%\begin{adjustwidth}{-0.7cm}{}%
\begin{tabular}{lr|S!{\VRule}S!{\VRule}S!{\VRule}S|}%
\multicolumn{2}{c|}{\multirow{2}{3cm}{Rates for $\bar{p}$ Transport}} & \multicolumn{4}{c|}{Angular Region for Antiproton Production} \\ 
                              &  & \multicolumn{1}{a!{\VRule}}{0-59\degree} & \multicolumn{1}{a!{\VRule}}{59-97\degree} & \multicolumn{1}{a!{\VRule}}{97-119\degree} & \multicolumn{1}{a|}{119-180\degree} \\ 
\hline
\multirow{3}{1.6cm}{Raw Counts} & Torus1 &\multicolumn{1}{r!{\VRule}}{7} & \multicolumn{1}{r!{\VRule}}{270} & \multicolumn{1}{r!{\VRule}}{2907} & \multicolumn{1}{r|}{278105} \\ 
                                & Torus2 &\multicolumn{1}{r!{\VRule}}{3} & \multicolumn{1}{r!{\VRule}}{58 } & \multicolumn{1}{r!{\VRule}}{830 } & \multicolumn{1}{r|}{20787 } \\ 
                                %& Torus2 Collim. & \multicolumn{1}{r!{\VRule}}{1} & \multicolumn{1}{r!{\VRule}}{39 } & \multicolumn{1}{r!{\VRule}}{431 } & \multicolumn{1}{r|}{9024  } \\ 
                                & Target &\multicolumn{1}{r!{\VRule}}{0} & \multicolumn{1}{r!{\VRule}}{8  } & \multicolumn{1}{r!{\VRule}}{114 } & \multicolumn{1}{r|}{2237  } \\ 
\hline
\multirow{3}{1.6cm}{Weighted Mean} & Torus1 & 1.68E-10    & 2.04E-13 & 1.32E-17 & 2.72E-20 \\ 
                                   & Torus2 & 6.16E-12 & 5.77E-14 & 3.97E-18 & 6.23E-21\\
                                   %& Torus2 Collim. & 1.95E-15    & 6.18E-15 & 1.85E-18 & 3.15E-21 \\ 
                                   & Target & {*}4.39E-16 & 1.39E-15 & 6.20E-20 & 9.29E-22 \\ 
\hline                                                           
\multicolumn{2}{r|}{$P(\textrm{Target}|\textrm{Torus2})$} &{*}0.225& 0.225  & 0.034  & 0.295\\
\hline
\end{tabular}%
%\end{adjustwidth}%
\caption{
Antiproton stopping rates and fluxes at key points: at the entrance to the bent
solenoids (Torus1), just before the final beam collimator (Torus2), and in front of the
stopping target (Target).
Raw counts are the total number of particles seen in each simulation of 80
million antiprotons. The weighted mean is the average of all particle
weights, given by the energy correction factor $\xi$
and the antiproton production angle, $\Phi(\theta,\phi)$.
$P(\textrm{Target}|\textrm{Torus2})$, gives the survival
probability for an antiproton to reach the target, given that it reached the Torus2 collimator.
Since no antiprotons were seen stopping for the angular region from 0 to
59\degree, the value for the final entry in that column---indicated
with an asterisk (*)---are obtained by multiplying the antiproton rate
expected at the Torus2 collimator with the median collimator acceptance from
the other three angular regions.%
\tablabel{bg:antiprotons:sim:antip}}
\end{table}\xspace}

%\newcommand{\TabAntiprotonFactors}{
%\begin{table}[tb]
%\centering
%        \begin{tabular}{llm{0.5\textwidth}}
%	\hline
%        Parameter & \multicolumn{1}{l}{Value} & Description \\
%	\hline
%        $R_{\pi/p}$                & \VarPiStopsPerPOT & Pion stopping rate per \ac{POT}  \\ 
%        $\mathcal{B}_\textrm{RPC}$ & \num{2.27e-2} & Branching ratio of \ac{RPC} \\ 
%	$f_{e,\textrm{RPC}}$           & \VarDetectedEsPerRPC & Probability of an RPC photon producing signal-like electrons in the detector \\ 
%	$A_\textrm{time}$              & \VarRPCTimingEfficiency & Acceptance of timing window to secondary electrons from RPC \\ 
%        $\epsilon_\textrm{extinction}$ & \VarExtinctionFactor[2] &  Extinction factor\\ 
%	\hline
%\end{tabular}
%\caption{\tablabel{bg:antiprotons:factors}
%Parameters and their values in the determination of the \ac{RPC} background rate.
%}
%\end{table}
%}

\newcommand{\TabAntiprotonEstimates}{
\sisetup{table-number-alignment=right,table-format=2.2e3,table-comparator=true}%
\begin{table}[b]
%\begin{adjustwidth}{-0.9cm}{}
\begin{tabular}{la|S|SSS|}
            &         & \multicolumn{1}{c|}{Stopping Rates} & \multicolumn{3}{c|}{Background Rate per POT} \\ 
                           &          &            & {No Timing} & {Prompt} & {Delayed} \\%& {Unweighted}
\hline                                                                              %              
\multirow{4}{*}{$\bar{p}$} & 0-59    & 4.39E-16 & 2.03E-22 & {-}      & 1.04E-22 \\ % & <6.58E-13
                           & 59-97   & 1.39E-15 & 6.43E-22 & {-}      & 3.30E-22 \\ % & 4.17E-16
                           & 97-119  & 6.20E-20 & 2.87E-26 & {-}      & 1.47E-26 \\ % & 2.48E-18
                           & 119-180 & 9.29E-22 & 4.29E-28 & {-}      & 2.20E-28 \\ % & 1.60E-17
\hline
\multirow{4}{*}{$\pi^-$}   & 0-59    & 1.20E-11 & 3.54E-18 & 3.54E-29 & 1.95E-30 \\ % & 6.58E-13
                           & 59-97   & 3.59E-14 & 1.06E-20 & 1.06E-31 & 5.83E-33 \\ % & 9.38E-16
                           & 97-119  & 6.02E-19 & 1.78E-25 & 1.78E-36 & 9.78E-38 \\ % & 5.44E-19
                           & 119-180 & 2.29E-22 & 6.78E-29 & 6.78E-40 & 3.72E-41 \\ % & 2.14E-19
\hline                                            
 \multicolumn{3}{r|}{Sum (per POT)}              & 3.56E-18 & 3.56E-29 & 4.34E-22\\           
\hline
\end{tabular}
%\end{adjustwidth}
\caption{\tablabel{bg:antiprotons:estimates}
Final estimated antiproton-induced background rates.
See text for column definitions.
}
\end{table}\xspace}

\newcommand{\FigBgBeamMomVsTime}{
\begin{figure}[t]
\centering 
\subfloat[][\figlabel{bg:beam:MomVsTime:strawTrk}$e^-$ at the Straw Tracker]{
\includegraphics[width=0.45\textwidth,trim=0 0 0 1.7cm,clip]{figs/backgrounds/Beam_TimeVsMomentum_StrawTkt.pdf}}
\subfloat[][\figlabel{bg:beam:MomVsTime:stopTgt}$e^-$ after the Beam Blocker]{
\includegraphics[width=0.45\textwidth,trim=0 0 0 1.7cm,clip]{figs/backgrounds/Beam_TimeVsMomentum_StopTgt.pdf}}
\caption{\figlabel{bg:beam:MomVsTime}
The timing and momentum of electrons detected at the straw tracker and passing a plane immediately after the beam blocker.
}
\end{figure}
}

\newcommand{\TabBgBeamFactors}{
\begin{table}[tb]
\centering
        \begin{tabular}{llm{0.5\textwidth}}
	\hline
        Parameter & \multicolumn{1}{l}{Value} & Description \\
	\hline
        $A_\textrm{geom}$ & \VarBeamBgGeometric &  Geometric acceptance to high-momentum electrons per \ac{POT}\\ 
        $A_\textrm{mom}$ & \VarBeamBgMomentum &  Fraction of beam electrons within the signal momentum cuts \\ 
        $A_\textrm{time}^\textrm{delayed}$ & \VarBeamBgTiming &  Fraction of beam electrons delayed until the detector window \\ 
	\hline
\end{tabular}
\caption{\tablabel{bg:beam:factors}
Parameters and their values in the determination of the background rate due to high-energy particles in the beam.
}
\end{table}
}

\newcommand{\FigBgBeamExtrapolate}{
\begin{figure}[t]
\centering 
\subfloat[][\figlabel{bg:beam:acceptance:momentum}Momentum of electrons]{
\includegraphics[width=0.49\textwidth,trim=0 1.5cm 0 0.7cm,clip]{figs/backgrounds/Beam_Acceptance_momentum.pdf}}
\subfloat[][\figlabel{bg:beam:acceptance:time}Time of high-$p$ electrons]{
\includegraphics[width=0.49\textwidth,trim=0 1.5cm 0 0.7cm,clip]{figs/backgrounds/Beam_Acceptance_time.pdf}}
\caption{\figlabel{bg:beam:acceptance}
Projections of the momentum and arrival time of electrons.
\protect\subref{fig:bg:beam:acceptance:momentum} shows the momentum for electrons that arrive at any time, 
whereas \protect\subref{fig:bg:beam:acceptance:time} only shows the arrival time for electrons with momentum above 65~MeV/c.
Magenta lines in \protect\subref{fig:bg:beam:acceptance:time} indicate the mean arrival time after the beam blocker and the mean and maximum arrival time
at the straw tracker.
The black line is an exponential fit to the momentum and time of electrons after the beam blocker, with the equation and constants shown next to the fit.
%Below this momentum, electrons from muon decay-in-orbit dominate the timing.
}
\end{figure}
}

\newcommand{\FigBgCosmicPrimary}{
\begin{figure}[t]
\centering
\subfloat[][\figlabel{bg:cosmics:primary:momentum}Momentum]{
\includegraphics[width=0.47\textwidth,trim=0 0 0 0,clip]{figs/backgrounds/Cosmics_primary_momentum.pdf}}
\subfloat[][\figlabel{bg:cosmics:primary:transverse}Transverse Direction Cosine]{
\includegraphics[width=0.47\textwidth,trim=0 0 0 0,clip]{figs/backgrounds/Cosmics_primary_transverse.pdf}}
\caption{\figlabel{bg:cosmics:primary}
Distributions of momenta and the transverse direction for the cosmic muon and antimuon fluxes provided by the ND280 collaboration.
}\end{figure}\xspace}

\newcommand{\FigBgCosmicBeam}{
\begin{figure}[t]
\centering
\includegraphics[width=0.98\textwidth,trim=2.5cm 0 2.3cm 0,clip]{figs/backgrounds/Cosmics_highP_electrons.png}
\caption{\figlabel{bg:cosmics:beam}
Projection onto beamline coordinate system of electrons from cosmic rays with momenta greater than 100~MeV/c, from a simulation of about 170 million cosmic muons.
Three tracks can be seen to enter the electron spectrometer and pass along, although their momenta are all above 105~MeV/c as visible from the upwards drift of the trajectories.
}\end{figure}\xspace}

\newcommand{\FigBgCosmicMomenta}{
\begin{figure}[t]
\centering
\includegraphics[width=0.9\textwidth,trim=0 0 1.6cm 0,clip]{figs/backgrounds/Cosmics_strawTrackHits.pdf}
\caption{\figlabel{bg:cosmics:momenta}
Momentum of electrons that hit one and two layers (planes) of the straw tracker.
The vertical axis shows the raw number of observed events based on the simulation of 170 million cosmic muons described in the text.
}\end{figure}\xspace}

\newcommand{\TabBgCosmicParameters}{
\begin{table}[t]
\centering
\begin{tabular}{llm{0.7\textwidth}}
	\hline
        Parameter & \multicolumn{1}{l}{Value} & Description \\
	\hline
        $R_\mu$                     & \num{4.17e5} & Cosmic ray flux through entire hall per second         \\ 
        $n_e^\textrm{obs}$          & 0.46         & Observed electrons per MeV/c                           \\ 
        $N_\textrm{sim}$            & \num{1.76e8} & Number of simulated events                             \\ 
        $(1-\epsilon_\textrm{CRV})$ & \num{1e-4}   & Miss rate of the \ac{CRV}                              \\ 
        $A_\textrm{duty}$           & 0.36         & Accelerator duty factor                                \\ 
        $A_\textrm{timing}$         & 0.51         & Fraction of time within gated-time detection window \\ 
        $A_\textrm{TDAQ,Recon}$     & 0.70         & Combined acceptance of TDAQ and Reconstruction         \\ 
	\hline
\end{tabular}
\caption{\tablabel{bg:cosmics:params}
Parameters and their values for the determination of the cosmic ray background rate.
}
\end{table}
}

\newcommand{\TabBackgroundFinalVals}{
\begin{sidewaystable}[b]
\centering
\sisetup{table-number-alignment=right,table-format=1.2e3}%
\renewcommand{\arraystretch}{1.5} % Default value: 1
%\begin{adjustwidth}{-0.7cm}{}%
\begin{tabular}{ll|SSS|S|L{7cm}}
\toprule
Type                         & Source                   & \multicolumn{3}{c|}{Background Rate}                                       & \multicolumn{1}{c|}{Total Events} & Comment \\ 
                             &                          & {per $\mu^-$ stop} & {per POT}              & {per second}            &            \\ 
\toprule
\multirow{2}{*}{Intrinsic}   & DIO                      & \NumDIOPerMuStop & \NumDIOPerPOT                    & \NumDIOPerSecond                 & \NumDIOTotal                     & \\ 
			     & RMC                      & \NumRMCPerMuStop & \NumRMCPerPOT                    & \NumRMCPerSecond                 & \NumRMCTotal                     & \\ 
\cmidrule{2-7}
\multirow{4}{*}{Delayed}     & RPC                      & {--}             & \NumRPCDelayedPerPOT             & \NumRPCDelayedPerSec             & \NumRPCDelayedTotal              &  \\ 
			     & Beam                     & {--}             & \NumBeamBgDelayedPerPOT          & \NumBeamBgDelayedPerSec          & \NumBeamBgDelayedTotal           & \multirow{1}{7cm}{\footnotesize Beam includes high-energy electrons from $\pi$, $\mu$, and $n$ capture.} \\ 
			     & Stopped $\bar{p}$        & {--}             & \NumBgAntiprotonsPerPOT          & \NumBgAntiprotonsPerSec          & \NumBgAntiprotonsTotal           & \multirow{2}{7cm}{\footnotesize Based on conservative interpolation and extrapolation of limited experimental $\bar{p}$ data.}\\ 
                             & $\pi^{-}$ from $\bar{p}$ & {--}             & \NumBgAntiprotonsPiDelayedPerPOT & \NumBgAntiprotonsPiDelayedPerSec & \NumBgAntiprotonsPiDelayedTotal  & \\ 
\cmidrule{2-7}
\multirow{3}{*}{Prompt}      & RPC                      & {--}             & \NumRPCPromptPerPOT              & \NumRPCPromptPerSec              & \NumRPCPromptTotal               &  \\ 
                             & Beam                     & {--}             & \NumBeamBgPromptPerPOT           & \NumBeamBgPromptPerSec           & \NumBeamBgPromptTotal            &  \\ 
                             & $\pi^{-}$ from $\bar{p}$ & {--}             & \NumBgAntiprotonsPiPromptPerPOT  & \NumBgAntiprotonsPiPromptPerSec  & \NumBgAntiprotonsPiPromptTotal   & \\ 
\cmidrule{2-7}
\multicolumn{2}{c|}{Cosmics}                            & {--}             & {--}                             & \NumCosmicRatePerSecond          & \NumCosmicRateTotal              & \footnotesize Dominated by conservative miss-rate. \\ 
\bottomrule                                                                                                   
\multicolumn{2}{c|}{Total}                              & {--}             & {--}                             & \NumTotalBgPerSecond             & \NumTotalBgPhasII                &  \\ 
\bottomrule
\end{tabular}
%\end{adjustwidth}
\caption{ \tablabel{bg:summary}%
Final predicted background rates and events.
The dominant backgrounds are cosmic ray electrons and stopping antiprotons, which are predicted to have produce almost equal rates.
Assumes an extinction factor of \VarExtinctionFactor, a double-Gaussian momentum resolution with a core resolution of $\sigma=200$~keV/c and a tail probability of \VarResolutionProbFromRMCToSignal that events are misreconstructed with at least an additional 2.4~MeV/c. 
All other parameters are the same as stated in chapter~\sect{sense}.
}
\end{sidewaystable}%
}

\newcommand{\FigBgVsResolution}{
\begin{figure}[t]
\centering
\includegraphics[width=0.9\textwidth,clip=true,trim=0 0 0 1cm]{figs/backgrounds/BG_vs_resolution.pdf}
\caption{\figlabel{bg:resolution}
Dependence of the intrinsic background rates on $\sigma_2$, the width of the high/low momentum tail in the reconstruction distribution.
From this it can be seen that the background rates from DIO will always dominate, compared to RMC.
}\end{figure}\xspace}

%% Important variables:
% Number of POT
% Number of muons stopped
% Total run time
\chapter{COMET \phaseII: Backgrounds}
Having optimised and evaluated the signal sensitivity, it is now important to check the expected background rates.
Interpreting the final result to declaring an observation or produce a confidence limit is only possible if the number of background events has been predicted.
The certainty of any observation or the stringency of a final confidence limit is determined by the relative background rate and signal sensitivity; ideally the background rate should be small, well below a single expected event.

The types of background we must consider for COMET were outlined in \sect{detector:background}, but in this section they are evaluated in the optimised experiment design using the improved simulation.

\FigDIOBackground
\section{Muon Decay in Orbit (\acs{DIO})}
\sectlabel{bg:dio}
%\begin{easylist}
%# Discussed in previous section \sect{sense:momentum}
%# DIO spectrum near end-point: which expansion is correct?
%# Final DIO rate above momentum threshold with resolution
%\end{easylist}
In order to decide the optimal threshold for the momentum cut, it was necessary, in section \sect{sense:momentum}, to study the rate of \ac{DIO} events.
At the selected momentum threshold of \VarMomThreshold, the number of expected DIO events per muon stop is \VarDIOPerMuStop{}.
Given the \VarTotalMuStops~muon stops that should take place during \phaseII, the total expected number of background events due to DIO is \VarDIOTotal.

However, it is important to note just how steeply falling the DIO rate is in this region.
\Fig{bg:dio:rates} shows how, given a fixed signal sensitivity, the number of DIO background events is affected.
Changing the momentum threshold also affects the signal acceptance, so that for a fixed \ac{ses} one must increase the running time.
Simultaneously, with more muons stopped the \ac{DIO} rate will increase for a fixed momentum threshold.
\Fig{bg:dio:rates} was produced using the same DIO and signal spectra shown in \fig{sense:spectra:resolution}, which included energy losses in the material of the target, beamline, and detector, as well as a 200~keV/c Gaussian resolution function. 
In addition, all the acceptance parameters of the sensitivity chapter were held fixed, except for the momentum cut efficiency.

\FigDIOEndPointComparison
The fact that the run-time, signal acceptance, and background rate depend so strongly on the momentum threshold makes the theoretical prediction for the \ac{DIO} spectrum particularly important.
The two most recent calculations of the high energy tail of the DIO spectrum are shown in \fig{bg:dio:spectra}, as well as the cruder function that fits the whole spectrum range, used in SimG4.
The more conservative spectrum from the 2011 paper has been used in this study.
However, if one were to use the more recent 2015 spectrum, background rates due to DIO would fall to 0.072 events at the same momentum threshold.
This would agree with that papers statement that the inclusion of radiative corrections suppresses the DIO background by 15\% at the end-point.

In total, based on the 2011 DIO spectrum, \sci{14}{-17} electrons would be produced per muon stop with momentum greater than 104~MeV/c.
Therefore, the DIO rejection efficiency, including both geometric effects and the threshold on the detected momentum, suppress the detection of DIO end-point electrons by about 98.4\%.

\section{Radiative Muon Capture (\acs{RMC})}
\TabRMCEndPoints%
\sectlabel{bg:rmc}
During the process of nuclear muon capture, there is a finite probability of a hard photon being radiated from the muon, nucleus, or the exchanged $W$-boson.
This is known as \acf{RMC} and is distinguished from radioactive gamma-ray production during nuclear de-excitation or decay of the daughter nucleus.
The maximum energy this photon can take, $\max(E_e^\textrm{RMC})$, differs from the \mueconv signal and \ac{DIO} end-point by the minimum energy needed to change the nucleus from $N(A,Z)$ to $N(A,Z-1)$:
\begin{align}
\max(E_e^\textrm{RMC})=&\big(M_\mu - B_{\mu,\textrm{binding}} - E_\textrm{rec}\big)-\big(M_{N_2}+\sum_iM_{h_i}-M_{N_1}-\Delta{}Z\cdot{}M_e\big),\\
                      =&E^\textrm{Conversion}_e  - \Delta{}M,
\end{align}
with $M_{N_1}$ and $M_{N_2}$ the mass of the parent and daughter nuclei respectively, and $M_{h_i}$ the mass of the $i$-th hadron (proton, neutron, alpha, \etc).
The mass of the electron, $M_e$, must also be included since for ever proton removed from the nucleus, given by $\Delta{}Z$, a free electron must be ejected from the atomic orbitals.
$\Delta{}M$ is then the total energy lost to changes in the atomic mass%
\footnote{
One has to think of the atomic mass difference, rather than the nuclear mass difference.
The distinction is important since consideration of the nuclear masses alone would ignore the effect on the atomic electrons.
Under nuclear muon capture (radiative or otherwise), the number of protons in the nucleus is reduced by at least one, and accordingly atomic electrons become unbound.
The notation here follows that of the COMET TDR~\cite{TDR2016} and CDR~\cite{CDRphase2}, where the electron mass is absorbed into the value of $\Delta{}M$ (although there it is called $\Delta_{Z-1}$).
}
of the nuclei and other emitted hadrons.

When an aluminium-27 nucleus captures a muon various daughter nuclei are possible.
If no other particles are emitted, as part of a prompt process or direct nucleon capture, the daughter nucleus will be magnesium-27.
In general, this could be left in an excited state, but to reach the end-point of the \ac{RMC} spectrum it will be left in the ground-state configuration.
The atomic mass difference between these two nuclei is shown in \tab{bg:rmc:massDifferences}, where it can be seen the RMC end-point is separated from the \mueconv signal energy by around 3~MeV.

\FigRMCExperiments
%\newcolumntype{d}[1]{D{.}{\cdot}{#1} }

\Fig{bg:rmc:experiments} shows the summary table of experimental data on Aluminium, taken from the summary by Gorringe~\cite{RevModPhys.76.31}.
It is interesting to note that in all experiments to date none of the emprical fits to RMC used have suggested an end-point above 90.1~MeV~\cite{PhysRevC.37.1633,PhysRevC.46.1094,PhysRevC.59.2853}.
This is close to the end-point predicted by the various transitions besides ${}^{27}$Al$(\mu,\gamma\nu){}^{27}$Mg.
The most recent of these two experiments both measure the branching ratio for \ac{RMC} producing photons with $E>57~$MeV to ordinary muon capture to be \sci{1.43}{-5}. 

To produce a background event in the detector, the high energy photons produced from RMC must be produce to a high energy electron.
Asymmetric pair production from this photon is one such process, although producing the positron at rest is highly suppressed.
In addition, around 1~MeV is consumed my the mass of the electron-positron pair.
Compton scattering provides another mechanism by which the high-energy photon can convert to a high-energy electron.
In the limit where the photon is reflected directly back, the resultant electron is around $M_e/2$ more energy than the incoming photon.
As such it is Compton scattering which is more of a concern for COMET.

Finally, given the maximum photon energy is below the momentum threshold of \VarMomThreshold, the energy needs to be mis-reconstructed by around 2.4~MeV.
If the final resolution function were Gaussian with a 200~keV/c width (at signal energies), then mis-reconstruction by such an amount would be an 11~$\sigma$ event, \ie $P(p_\textrm{recon}-p_\textrm{true}>\SI{2.4}{MeV/c})=\num{1.9e-28}$.
However, the resolution function is not likely to be a pure Gaussian.
On-going studies for reconstruction with the  \phaseI StrECAL currently suggest that $P(p_\textrm{recon}-p_\textrm{true}>\SI{2.4}{MeV/c})=\num{7.5e-3}$~\cite{YFujiiStrECALRecon}.
Although this is considerably larger than the pure Gaussian, this is based on the most preliminary results of reconstruction algorithms.
If need be, cuts on fit quality can be improved over the current \phaseI values; the sensitivity estimate presented here should be robust against this since it includes the CDR estimate for the signal efficiency of such a cut.

\subsection{Calculation and Simulation of RMC}
Conversion of the RMC photons can take place in any of the material around the target, such as the beam blocker, solenoids, or cryostat.
To estimate the acceptance of the electrons produced from this conversion a simulation was performed where RMC photons were input at the stopping target, using the realistic muon stopping distribution obtained previously.
Geant4 implements the process of photo conversion via both of the above methods.

To build a realistic end-point spectrum, the same recipe as used in the \phaseI TDR was applied~\cite{TDR2016}.
The spectrum shape near the end-point is modelled by the equation~\cite{CHRISTILLIN1980331}:
\begin{equation}
	\eqlabel{bg:rmc:spect}
	\Gamma(\textrm{RMC})\propto(1-2x+2x^2)(x)(1-x)^2,
\end{equation}
where $x=E_\gamma/\max(E_e^\textrm{RMC})$.
The largest observed branching ratio for \ac{RMC} -- \num{1.83e-5} compared to ordinary muon capture --  is used to set the normalisation.
The relative rate of \ac{RMC} resulting in photons with energies above 90~MeV is, based on the spectrum in equation~\eq{bg:rmc:spect}, \num{7.7e-3} of the experimentally observed rate.
\FigRMCSimResults

Given the total number of muon stops during \phaseII, the probability of ordinary muon capture, and the relative rates for \ac{RMC}, one expects some \num{1.2e11} photons to be produced in the target with energies above 95~MeV.
Since this is an intrinsic background, the lifetime of this process will be the same as for signal events. 
Assuming, therefore, the same timing window acceptance, and applying the same reconstruction and TDAQ efficiencies as for signal, this number is reduced to \num{3.3e10}.
If the resolution function were truly a 200~keV/c Gaussian, then this would be no issue, regardless of the geometric acceptance or probability of Compton scattering.
If, however, the current \phaseI high-energy reconstruction tail were present, we would expect some \num{2.4e8} events to be considered dangerous if they were all accepted.

To check the geometric acceptance of the beamline, some \num{6e7} RMC events were generated in the stopping target, with initial photon energies greater than 90~MeV and distributed according to the spectrum in equation~\eq{bg:rmc:spect}.
\Fig{bg:rmc:simulation} shows the momentum and rate at which these were detected.
Based on this, the fraction of events reaching the detector with momentum larger than 98~MeV/c was \num{6.2e-7}, whilst above 100~MeV/c only \num{3.3e-8} electrons per \ac{RMC}$|_{p>90}$ event were detected.

As a result, if the current \phaseI resolution were applied, we would expect around 8 background events during the entire run.
However, since the resolution function at this stage is so poorly known, it seems to premature to define a value.
Instead, the predicted RMC rate provides a constraint on the high-energy tail of the resolution function: to do better than 0.1 background events during \phaseII from RMC, fewer than 1 in \num{1e4} high-energy electrons can be reconstructed with a momentum more than 2.4~MeV/c larger than the true value.

\subsection{Aluminium-26 and \ac{RMC}}
Based on the above energy calculation, the end-point for \ac{RMC} against ${}^{26}$Al (to ${}^{26}$Mg with no other particles emitted) would be 108.5~MeV.
Clearly such photons would be extremely dangerous to COMET if they are produced.

Aluminium-26 comes in two isomers, one with a half-life of around 6 seconds~\cite{PhysRevLett.106.032501}, the other lasting around 700 thousand years~\cite{AUDI20033}.
Since Al-26 is unstable its abundance in natural aluminium is low.
However, it can be produced by various methods, such as proton and deuteron bombardment of magnesium and sodium, or photoneutron emission of the aluminium-27 isotope~\cite{THOMPSON1965486}.
With such production mechanisms, it is likely that alumininium-26 will be produced in the COMET stopping target, via the interaction of daughter nuclei of muon capture (which are typically magnesium and sodium)
with protons, deuterons, gammas, and neutrons coming from either the beam or as products of muon capture.
The exact rate of Al-26 production, however, is a complicated value to estimate and one that unfortunately cannot be estimated here.

One can, however, set an acceptable rate of production if the induced background event rate is to be kept at the level of 0.1 events.
Based on the previous simulation of Al-27 \ac{RMC}, the probability of an electron being detected within 5~MeV of the \ac{RMC} end-point is \num{5.6e-18} per nuclear muon capture.
With \VarTotalMuStops muon stops during \phaseII, and assuming that the branching ratios for both ordinary and radiative muon capture are the same for Al-26 and Al-27 (61\%), then the
concentration of Al-26 in the stopping target must be less than 1\% (by number density) on average during the entire \phaseII run.
The branching ratio for radiative muon capture is, in reality, likely to be slightly more than in Al-27, based on the fact that the neutron excess of a given isotope seems a better indicator of the branching ratio than atomic number, and that the two are anti-correlated~\cite{RevModPhys.76.31}.

Clearly though, there is more work to be done on this, including a better understanding of \ac{RMC} events coming from Al-26, as well as understanding the rate of Al-26 production from muon beams.
Data from the \alcap experiment might be able to help with this, however, since the production rate could depend a lot on the exact beam conditions, measuring this directly should be an important goal of \phaseI.

Finally, if this does produce a sizeable background contribution, one can imagine several techniques to mitigate or reduce the challenge this poses.
Provided one can measure the concentration of Al-26 in the stopping target at the end of the run, and ideally at various stages whilst running, then the number of backgrounds can be predicted and possibly subtracted.
Additionally, it could be possible to remove the stopping target and replace it with a fresh one such that the Al-26 concentration never rises beyond an unsafe level.

\CHECK{Mention that this could be a useful source of calibration}

\section{Radiative Pion Capture (\acs{RPC})}
\sectlabel{bg:rpc}
When low-energy negative pions are stopped in material they behave similarly to negative muons and form pionic atoms.
The probability that the pion is then captured by the nucleus rather than decays in orbit is, however, considerably larger than for a muon.
Furthermore, given the extra 30~MeV/c$^2$ of the pion mass and the lack of an outgoing neutrino, the end-point for \acf{RPC} by the nucleus is well above the \mueconv signal energy.
As for photons of \ac{RMC}, the \ac{RPC} photons can then be converted via Compton scattering or pair production to signal-like electrons.
Pion capture could, therefore, be a dangerous source of backgrounds and was likely the dominant source of background events at \sindrumII.

Since the pion interacts via the strong force, negative pions capture almost immediately in the nucleus, on the order of picoseconds~\cite{Engelhardt:1975ct}.
The timing of backgrounds caused from pion capture are therefore determined predominantly by the time when the pion was produced.
If a background arises from pion capture and the pion was produced in the main muon pulse, then the pion or resultant background electron must have been significantly delayed.
Delayed \ac{RPC} backgrounds are therefore suppressed by the time-gated detector window.
If however a background arises because the pion was produced outside of the main proton pulse, due to late-arriving protons (or from antiprotons, but we will treat these separately below), then this background is considered prompt.
These prompt \ac{RPC} background events are, therefore, suppressed by the extinction factor.

The background rate per \ac{POT} for prompt and delayed pions is therefore:
\begin{align}
	R_\textrm{prompt}=&R_{\pi/p}\mathcal{B}_\textrm{RMC}f_{e,\textrm{RMC}}A_\textrm{time}, \\
	R_\textrm{prompt}=&R_{\pi/p}\mathcal{B}_\textrm{RMC}f_{e,\textrm{RMC}}\epsilon_\textrm{extinction},
\end{align}
where $R_{\pi/p}$ is the pion stopping rate per \ac{POT} and $\mathcal{B}_\textrm{RMC}$ is the branching ratio of \ac{RMC} for stopped pions.
$\epsilon_\textrm{extinction}$ is the extinction factor, whereas $A_\textrm{time}$ is the acceptance of the time-gated detector window to RMC electrons.
$f_{e,\textrm{RMC}}$ is the probability that an RMC photon converts to an electron which reaches the detector with signal-like momentum.
To a reasonable approximation, this can be factorised as:
\begin{equation}
f_{e,\textrm{RMC}}=f_{\gamma\rightarrow{}e^-}A_\textrm{geom}A_\textrm{mom}
\end{equation}
where $f_{\gamma\rightarrow{}e^-}$ is the conversion rate of RMC photons to an electron, $A_\textrm{geom}$ and $A_\textrm{mom}$
are, respectively, the geometric acceptance and momentum cut efficiency for such electrons.
However, such a factorisation misses out various correlations, such as where in the experiment the conversion takes place and at what momentum the secondary electron is produced, so that only a single value for $f_{e,\textrm{RMC}}$ will be reported here.

\FigRPCData
\subsection{Photons from Radiative Pion Capture (\acs{RPC})}
%In much the same way as \ac{RMC}, when a pion is captured in a nucleus there is a chance that it will produce a prompt, hard photon.
There is a range of experimental and theoretical data on \ac{RPC}. 
\Fig{bg:rpc:spectra} shows what is perhaps the most useful data currently available: the observed spectrum of photons coming from RPC for magnesium and calcium.
Magnesium being adjacent to aluminium on the periodic table, this spectrum is a reasonable proxy for the spectrum of \ac{RPC} on aluminium.
The relative rate for \ac{RPC} compared to ordinary pion capture is discussed in Amaro \etal~\cite{Amaro:1997ed}.
For experimental and theoretical studies of the three isotopes summarised in that paper --- carbon, oxygen, and calcium --- the measured and predicted branching ratios are all within 1.19\% and 2.27\%.
To be conservative, we take here the branching ratio for RPC on aluminium to be the largest of these as 2.27\%.

\subsection{Pion Stopping Rate}
\FigPionStopDist
\FigPiVsMuMomenta
To simulate the pion stopping rate, the pions from the main production simulation were resample multiple times to build up a large number of pion stops.
\Fig{bg:piStop:dist} shows the distribution of pions stopping in the target, in one-dimensional projections to the ICEDUST global coordinate system.
In that coordinate system, pions arrive from large values of Z at the target, so the pion beam in \fig{bg:piStop:dist:z} is going from right to left.
By comparison with the plots in \fig{sense:stops2D}, pions tend to stop further downstream in the target.
This is readily understood by the fact that pions reaching the stopping target tend to have much higher momentum than muons reaching the target, as shown in \fig{bg:piVsMu:momenta}.

In total, the number of pions stopping in the target per \ac{POT} is $R_{\pi/p}=$\VarPiStopsPerPOT.

\subsection{Simulating \acs{RPC}}
\FigRPCSimulatedSpectrum
\FigRPCSimResults
Using the realistic pion stopping distribution shown in \fig{bg:piStop:dist}, \ac{RPC} photons were generated in the target and Geant4 used to convert and track electrons resulting from this process.
To model the distribution of photon energies from \ac{RPC}, the experimentally obtained spectrum from magnesium of ref.~\cite{Bistirlich:1972jy} was used.
To build the model, the raw spectrum was first digitised and then smoothed, using `TGraphSmooth::SuperSmooth()' from the  ROOT library~\cite{ROOT}.
These steps are shown in \fig{bg:rpc:spectrum}.

Based on a simulation of \num{4e6} RPC photons, the distribution of electrons and positrons reaching the detector was obtained.
The timing and momentum of such electrons is shown in \fig{bg:rpc:sim}, where it is clear that although many signal-like electrons are detected, they all arrive well before the gated-time threshold of 600~ns.

The probability of an \ac{RPC} photon producing an electron that reaches the detector with momentum between \VarMomThreshold and 105~MeV/c is: $f_{e,\textrm{RMC}}=$\VarDetectedEsPerRPC.
To estimate the rate of delayed \ac{RPC} backgrounds, we also need to know the value of $A_\textrm{time}$.
As can be seen in \fig{bg:rpc:sim:momVtime}, the timing for electrons originating from RPC photons and detected with momentum greater than 30~MeV/c is independent of the momentum.
By fitting the tail of this distribution with a single exponential, the lifetime of the high energy electrons is found to be 18.6~ns, such that with the timing window between 600 and 1200~ns we find the acceptance of the timing window to be: $A_\textrm{time}=\VarRPCTimingEfficiency$.

\Tab{bg:rpc:estimates} summarizes these numbers, from which we find that the rate of backgrounds for delayed \ac{RPC} is \VarRPCDelayedPerPOT per POT, whilst the prompt form occurs at \VarRPCPromptPerPOT.
\begin{table}[tb]
\centering
        \begin{tabular}{llm{0.5\textwidth}}
	\hline
        Parameter & \multicolumn{1}{l}{Value} & Description \\
	\hline
        $R_{\pi/p}$                    & \VarPiStopsPerPOT & Pion stopping rate per \ac{POT}  \\ 
        $\mathcal{B}_\textrm{RPC}$     & \num{2.27e-2} & Branching ratio of \ac{RPC} \\ 
	$f_{e,\textrm{RPC}}$           & \VarDetectedEsPerRPC & Probability of an RPC photon producing signal-like electrons in the detector \\ 
	$A_\textrm{time}$              & \VarRPCTimingEfficiency & Acceptance of timing window to seondary electrons from RPC \\ 
        $\epsilon_\textrm{extinction}$ & \VarExtinctionFactor[2] &  Extinction factor\\ 
	\hline
\end{tabular}
\caption{\tablabel{bg:rpc:estimates}
Parameters and their values in the determination of the \ac{RPC} background rate.
}
\end{table}

\section{Antiprotons in the Beam}
\sectlabel{bg:antiprotons}
Antiprotons can be produced when the primary 8~GeV proton beam interacts with the production target, creating a proton--antiproton pair:
\begin{equation}
p + N(A,Z) \rightarrow p + N^*(A,Z) + p+\bar{p}
\end{equation}
Given their relatively large mass, antiprotons travel much more slowly than other products of the same momentum which results in a smearing of the beam's time structure for antiprotons and their secondaries.
The pulsed beam and time-gated detector window are, therefore, less effective at suppressing the induced backgrounds.

%Secondaries of antiproton interactions with matter include X-rays and pions, though pion production dominates~\cite{BobBernsteinExperimentersGuide,Mishustin:2004xa}.
%
The interaction of antiprotons with matter has a strong analogy with muons and negative pions, particularly at low energies, when the antiprotons stop.
%There is a strong analogy between antiproton stopping in matter and the phenomona around negative muon and pion stopping.
In matter, antiprotons with energies of a few tens of keV --- similar to that of atomic electrons --- can become bound in the Coulomb potential of the atom's nucleus.
X-rays emitted in the ensuing electromagnetic cascade are typically not more than a 100~keV~\cite{Aramaki201352}.
Unlike for muons, the antiproton will only rarely reach the atomic ground state, before the interaction with the nucleus takes over.
This interaction can take the form of an immediate annihilation or the formation of a composite nucleus where the antiproton is now bound within the nuclear potential~\cite{Wong:1984fy,Mishustin:2004xa}.
The binding energy of some of these nuclear levels can reach up to hundreds of MeV, and so in the transition, pions and other hadrons can be readily produced.
Eventually though the antiproton will annihilate in the nucleus, producing large multiplicities of pions and other mesons.

As such, although antiprotons themselves are not an immediate source of high energy photons or electrons, they are an additional source of pions, which can produce backgrounds via \acf{RPC}.

\subsection{Antiproton Production Rate and Spectrum}
The literature on antiproton production with 8~GeV protons on a tungsten target and at large angles is somewhat lacking.
%There is really very little literature on the production of antiprotons from a tungsten target for a range of angles.
Accordingly most hadron models are particularly under-constrained when it comes to antiproton production.
The QGSP_BERT_HP model used as the basis for SimG4 is, in fact, completely unable to produce antiprotons.

In the COMET TDR~\cite{TDR2016}, the yield of antiprotons per \ac{POT} is given as \num{4e-5}, based on the results of a MARS simulation performed for the MECO experiment~\cite{Meco024}.
Based on this, simulations of antiprotons showed that absorber foils would be needed along the beamline.  
Since the \phaseII geometry in ICEDUST re-uses most of the implementation for the Production Target Capture and Torus1 sections, the upstream absorbers are also contained in the geometry used to study antiprotons here,
although the absorber near 90\degree has been removed.

\FigAntiprotonData
Whilst tungsten targets have not been studied at the relevant angles and proton energies, a set of papers \cite{Boyarinov:1994tp,Kiselev:2012sj} do exist covering antiproton production up to 2 radians for tantulum (which is adjacent to tungsten on the periodic table), copper, aluminium, and beryllium targets and using 10 GeV protons.
Between them, these two papers provide the invariant triple-differential cross section as a function of antiproton momentum for production angles of 10, 59, 97 and 119\degree, defined as:
\begin{equation}
\eqlabel{bg:antiprotons:lit}
	F(p,\theta,\phi)=E \frac{d^3\sigma}{dp^3}=E\frac{d^3\sigma(p,\theta,\phi)}{p^2dpd\Omega}
\end{equation}
where $E$ and $p$ are the antiproton energy and momentum.  The earlier Boyarinov paper actually reports $f=F/A$, where $A$ is the relative atomic mass of the nucleus.
\Fig{bg:antiprotons:data} sHows the data from these papers converted to the relative rate such a cross section would imply for the \phaseII tungsten target.
From the measured cross sections, it is clear that as you move to larger angles, the spectrum becomes considerably softer whilst the overall rate falls quickly, in much the same way as for pion production.
To convert the differential invariant cross sections given in the literature (equation~\eq{bg:antiprotons:lit}) into a differential production rate per \ac{POT}, $d^3R(\theta)/dpd\Omega$, the following formula is used:
\begin{align}
	\eqlabel{bg:antiprotons:rate}
	\frac{d^3R(p,\theta,\phi)}{dp^3}&=\frac{F(p,\theta,\phi)}{E}\frac{\rho N_A l}{m_N}
\end{align}
where $\rho$ is the mass density of the target, 19.25~g/cm$^3$ for Tungsten, $m_N$ the atomic mass, 183.86~g/mol for tungsten, $l$ the length of the target, and $N_A$ is Avodadro's number.

\FigAntiprotonEndpoint
To build the limited number of data-points from this paper into a complete but conservative spectrum, empirical fits to the data were performed after adding two data points for the minimum and maximum momentum.
The minimum momentum only added the constraint that at zero momentum the cross section also be zero.
On the other hand, to calculate the maximum momentum the kinematic end-point was found by considering the entire nucleus and the two out-going protons to recoil directly against the antiproton.
The value of the end-point kinetic energy and longitudinal and transverse momenta for tungsten and carbon are shown in \fig{bg:antiprotons:end-point} using the formulae derived in appendix~\sect{appendix:antiprotonEndpoint}.
This end-point will be a highly conservative estimate, since in reality not all the nucleus will recoil coherently;  the de Broglie wavelength for a proton with 8~GeV kinetic energy is about 0.15~fm, compared to the 7~fm or so of a tungsten nucleus%
\footnote{Antiproton production from protons in this energy regime was historically referred to as `sub-threshold production', since it is below the threshold for single nucleon interactions, around 5.6~GeV\CHECK{Is that the single Nucleon threshold}.
The fact antiprotons are observed is therefore proof that it is not a single, stationary nucleon that interacts with the proton and the nuclear environment as a whole must be considered.
  The older literature on this topic in fact refers to a parameter called the `cumulative number' which was related to the number of nucleons that must be involved to produce secondary hadrons (\eg antiprotons or pions) with such out-going energies, given the incoming proton energy.  
These sub-threshold secondary particles were themselves sometimes called cumulative particles.}.
Additionally, achieving this end-point configuration would be highly phase-space suppressed.

\FigAntiprotonFits
With the addition of these two end-points, to interpolate and extrapolate the data, a polynomial of order 4 to 6 (depending on the number of available data points) was used to fit from zero up to the last experimental datum.
For the high-momentum tails of each spectrum, two fits were tried: a straight line fit between the last experimental datum and the kinematic end-point described above; and an exponential fitted to the kinematic end-point and the last two experimental data points.  
\Fig{bg:antiprotons:fits} shows the results of this fitting procedure, where it can be seen that a low momentum peak is visible and a high momentum tail well described.
Whilst these spectra are likely a poor representation of the true production spectrum, they serve as useful upper bounds which can be used as inputs for the antiproton background rate estimation.

\TabAntiprotonRegions
\subsection{Simulating Antiprotons}
To study the resultant backgrounds, antiprotons were generated uniformly in the production target.
Four separate simulations were run, corresponding to the four angular regions provided by the data, as shown in \tab{bg:antiprotons:regions}.
For each simulation, antiprotons were generated isotropically within a cone parallel to the incoming proton beam, with minimum and maximum values of theta defined by the angular region being studied.
The momentum distribution for each simlation used the fit to the data for the angle at the lower edge of the region, $\theta_\textrm{min}$ (and for the region from 0 -- 59\degree, the fit to the 10\degree data was used).

\FigAntiprotonAngularDependence
To account for the strong angular dependence visible from the measurements, each input antiproton event was reweighted during analysis based on the angle between the initial antiproton direction and the proton beam.
The combined procedure of generating isotropically with a momentum distribution based on the fit to the data, then reweighting based on the angle, amounts to the following factorisation of the measured differential cross section:
%It is clear from the four measured directions that the cross section for antiproton prodcution has a strong dependence on the angle of production.
%To extrapolate this dependence to angles other than those measured, the differential cross section was factorised into a momentum and angular dependence:
\begin{align}
\eqlabel{bg:antiprotons:factorisation}
	F(p,\theta,\phi)=&\left(\frac{E}{p^2}\frac{d\rho(p)}{dp}\right)~\left(\frac{1}{2\pi}\frac{d\Theta(\theta)}{d\theta}\right),\\
	                =&P(p,\theta_\mathrm{min)}\Phi(\theta),
\end{align}
where $P(p,\theta_\textrm{min})$ is the fitted momentum spectrum for the differential rate at the lower angle of the current region, $\theta_\textrm{min}$.
$P(p,\theta_\textrm{min})$, therefore, is the probability density function used to generate the momentum for each antiproton in the simulation, and as such the normalisation of $P(p,\theta_\textrm{min})$ is fixed to unity.

Instead, $\Phi(\theta)$ is used to provide the event weight.
It encapsulates the total probability of an antiproton being produced at the given angle with respect to the incident proton, and is found by integrating the fits shown in \fig{bg:antiprotons:fits}.

The results of these integrations is shown in \fig{bg:antiprotons:angular}.
%When these distributions are used to generate particles they are essentially normalised so their integral is unity.
%For a given value of $\theta$ the momentum distribution $dP/dp$ is chosen to match the first spectrum measured at a larger value of $\theta$.
%This keeps the model conservative since the true momentum spectrum should soften as one moves to higher angles which this will not do.
%$d\Phi/d\theta$ is then used to vary the overall normalisation as a function of theta.
Boyarinov et al. state~\cite{Boyarinov:1994tp} that the angular dependence of antiproton production should take the form:
\begin{equation}
%\frac{d\Phi(\theta)}{d\theta}
\Phi(\theta)=\alpha e^{\beta\cos\theta}
\end{equation}
where $\alpha$ and $\beta$ are constants. 
It is for this reason that the axes of \fig{bg:antiprotons:angular} are $\cos\theta$ and $\ln(R)$: if such a function were valid, the points would form a straight line.
In fact it seems like a quadratic function might be more appropriate, although the fitted quadratic functions all reach a minimum above $\cos\theta=-1$.
To keep things conservative then, instead of a linear or quadratic function fitted to all data points, we use the linear interpolation between each adjacent datum to provide the function, $\Phi(\theta)$.
Of the three possible fits to the high momentum tail, the angular dependence obtained by the linear tail is used for the reweighting, again to ensure this simulation be conservative.

Finally, it should be noted that the factorisation in equation \eq{bg:antiprotons:factorisation} injects the assumption that the momentum distribution at a given angle is independent of the angle.
Whilst this should not be the case for a realistic spectrum, to produce limits on the background rate it is a valid assumption --- the spectrum becomes softer with increasing production angle so the spectrum at the smaller angle should be an overestimate.
In addition, the $\phi$ dependence of $\Phi(\theta,\phi)$ is assumed constant, since the target and proton beam in COMET are not polarised.

\FigAntiprotonSimHeightsTwoDPbar
\FigAntiprotonSimHeightsTwoDPiMin
\subsection{Results of simulation}
Using the procedures described above, \num{2e5} antiprotons were fed through each angular region.
The transport of these antiprotons through the beamline is shown in \fig{bg:antiprotons:sim:2D-antip}.
The secondary pions produced from antiprotons interacting with the beam pipe and material in the beamline are then shown in \fig{bg:antiprotons:sim:2D-pi}.
From these it can be seen how antiprotons in the small angle region are not of significant concern, since they nearly all leave the capture solenoids and pass into the beam dump.
Antiprotons from the two large angles regions, however, penetrate the foil at the entrance to the Torus1 region (which is indicated by the vertical magenta line second from the left).

There are two principle regions of pion production: at the production target itelf, and the foil at the entrance to Torus1.
Many of the pions produced before the bent solenoids are of high energies and are removed quickly by the magnetic field in the bent solenoid.

For both particles, there is a finite probability that pions and antiprotons reach the stopping target region although at the present level of simulated statistics none have been seen to pass this point.

\Fig{bg:antiprotons:sim:1D} shows the mean probability per secondary antiproton that a particular particle type passes a given point.
From this the above conclusions are confirmed: most pions originate from the production target and the foil at the entrance to the Torus1, and antiprotons from the largest angles are able to penetrate this foil.
It can also be seen that a number of muons are produced and even stop in the stopping target --- antiprotons might produce background events, but at some level they can produce signal events.
\FigAntiprotonSimFluxes
\FigAntiprotonSimPiMom
\TabAntiprotonResults
\TabAntiprotonEstimates

The number of pions passing the 90\degree point along the bent solenoid and the number of antiprotons stopping in the target are summarised in \tab{bg:antiprotons:results}.
Since for all regions except for angles larger than 119\degree, only an upper limit can be set on the number of antiprotons that stop in the target.

To convert these numbers into estimates for the background rates, we use the study for \acf{RPC}.
\Fig{bg:antiprotons:sim:piMom} shows the momenta of pions that pass the 90\degree point and the momentum distribution from the \ac{RPC} study. 
Since the momentum distributions are approximately similar (albeit with different statistical significance), we use the rate of \ac{RPC} backgrounds per pion passing this point based on the RPC study in section~\sect{bg:rpc}.
The probability that a pion stops in the target given that it reached the 90\degree point is \num{3.22e-4}, whilst the probability that a stopped pion produces a signal-like electron in the detector (without any cut on the timing) is \num{4.20e-7}.
Finally, we apply the same TDAQ and reconstruction efficiencies, as well as the fraction of time occupied by the gated-time detector window, amounting to a combined factor of 0.36.
Since pions in the main beam simulation originate at the production target, their pitch angles at the entrance of the bent solenoid are relatively small compared to pions produced at the entrance to the vacuum window from antiprotons.  
Since large pitch angles implies a greater amount of drift, pions coming from antiprotons hitting the vacuum window tend to be lower down in the beampipe. 
Accordingly the transmission of pions coming from antiprotons is likely smaller than for those coming from normal proton events, and accordingly the above probability of an RPC event being produced from a secondary pion of an antiprotons is probably an overestimate.

To convert the rate of stopped antiprotons into a background rate, we assume every negative pion produced from antiproton capture in the target produces an RPC event.
Based on branching ratios for pion production from antiproton annihilation reported in \cite{Mishustin:2004xa} (including the probability of subsequent $\pi^-$ production from heavier mesons produced during annihilation), the average multiplicity of negative pions arising from antiproton annihilation is 1.56.
These pions are assumed to stop immediately in the target and the additional factors for RPC backgrounds described above are then also applied.

Finally, pulling all of these numbers together, the total predicted background rate due to antiprotons produced in the production target is found, and the values are summarised in \tab{bg:antiprotons:predictions}.
The total background rate, summed over all four angular regions, is estimated to be less than XXXX.  
Although the model that has been developed throughout this section aims to be particularly conservative, the final results are statistically limited, particularly for the small angle regions, where most of the antiprotons are produced.

\begin{easylist}
# Timing cut on pions
# Antiproton transmission per antiproton passing the vacuum flange based on those produced above 119 degrees
\end{easylist}

\subsection{Reducing Antiproton Backgrounds Further}
\begin{easylist}
# Ways to improve this / further optimisations:
## Antiprotons close to the beam axis after the production target: on-axis blocker (activation?)
## Absorber material in lower half of TS3
\end{easylist}

\subsection{Antiproton Transmission}
\Fig{bg:antiprotons:transmission} shows how antiprotons are transmitted along the beamline.
Material at the join between the production target capture region and the first bent muon transport solenoid designed to keep the vacuum seal and cryogenic conditions is also acting to remove antiprotons.
\Fig{bg:antiprotons:ap-timing-tor1-entrance} shows the timing of antiprotons reaching the entrance of the bent muon transport solenoid.

Based on these simulations, the probability that an antiproton will enter the bent muon transport solenoid is well less than XXXX per POT.


\subsection{Delayed Pion Production}
\sectlabel{bg:rpc}

\begin{easylist}
	# Pion stops per pion passing Torus2 monitor = 3.21826657700288808e-04 %based on file: ~/comet/1603w12_Phase-II_estimates/pions_in_beam/160429_Pions-1_BeamTags_analysis.root
	# RPC background rates:
	## Prob. of RPC = 2.15\% per stopped pion
	## Prob. of RPC producing signal-like electron in detector = 1.05e-5 per RPC event
	# Total probability of RPC event per pion passing Torus2 monitor = 7.26523679758401863e-11
	# Number of pions at Torus2 monitor: 
	## For region 1 (0 -- 59\degree): 2.1e-9 pions per antiproton in this range
	# Total backgrounds from pions from antiprotons:
	## For region 1: 5.32e-4 * 2.1e-9 * 7.3e-11 = 8.5e-23 per POT
	# Assumes no timing cut, but actually timing cut will help reduce this
\end{easylist}

\section{Direct Beam-Related Backgrounds}
\sectlabel{bg:beam}
%\begin{easylist}
%	# Number of POT simulated
%	# Number of observed high momentum electrons or muons
%	# Lifetime of muons at the target (need to smear this a bit ?): Fit the projection of high momentum particles to the time with an exponential and extrapolate
%	# Survival probability per high momentum particle before the target to reach the straw tracker
%\end{easylist}
Although neither a muon or pion in their rest frame can decay to electrons greater than 55~MeV, once these particles are boosted sufficiently the outgoing electron can also become boosted into the signal region.
For muons, at 78~MeV/c the decay electron could be boosted to 105~MeV/c if it decays in the direction of the muon.
For pions, the threshold for signal-like electrons to be produced during decay is 58~MeV/c, although the branching rate for this process is helicity suppressed to around \num{1e-4}.

The bent solenoids and the pulsed beam help to suppress this backgrounds as does the beam blocker after the target disks.
The inner radius of the collimator at the exit of Torus2 is set to 12~cm, whilst the beam blocker has a radius of 25~cm. 
Even though the beam aperture grows due to the reduction in the field strength between the exit of Torus2 and the beam blocker, these values prevent a direct line of sight between the muon beam and the spectrometer.
As such, the only way for beam particles to reach the detector are by a hard scatter off the target or beam blocker, or to be produced via decay of another particle close to the target itself.

To check the probability that signal-like electrons are produced in the beam and arrive at the detector in the time window,
the output of the large production target simulation were resampled five times, so that around \num{1.4e9}~\ac{POT} events were studied, equivalent to 7.5~\phaseII bunches \CHECK{Is this the right number of simulated events?}.
\begin{easylist}
	# Number of POT simulated
	# Number of observed high momentum electrons or muons
	# Lifetime of muons at the target (need to smear this a bit ?):
        ## Fit the projection of high momentum particles to the time with an exponential and extrapolate
        ## Use a double exponential fit, need to high momentum
	# Survival probability per high momentum particle before the target to reach the straw tracker (geometric x momentum)
	# Product of all factors
	# Statistically limited result
	# Beamline in Phase-II is about 3 times longer than Phase-I, 
	# Need to move away from brute-force Monte Carlo and develop reweighted algorithms, such as disabling particle decay and weighting by the rest frame time divide by the lifetime
	# High energy electron acceptance vs. beamline distance 
\end{easylist}

\section{Cosmic Ray Background}
\sectlabel{bg:cosmics}

\section{Neutrons from the Production Target}
\sectlabel{bg:neutrons}

\section{Summary of Background Rates}
\sectlabel{bg:summary}

\section{Further Studies and Improvements}
\sectlabel{bg:improvements}
\begin{easylist}
# Neutron background with different hadron codes and \alcap neutron spectrum from mu capture
# Improved cosmic ray veto geometry
# High-energy electron acceptance
# Pion stops elsewhere in the beamline
# StrECAL resolution function
# Improved proton beam timing structure 
# RMC from Al-26 that is produced from Al27 via neutrons. Photons can be made up to 108 MeV
\end{easylist}
