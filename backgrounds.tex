\newcommand{\FigDIOBackground}{
\begin{figure}[tbp]
\centering
%\fbox{
\includegraphics[width=1.0\textwidth,trim=0 0 1cm 0.93cm,clip]{figs/backgrounds/Dio_BackgroundRateVsRuntime.pdf}
%}
\caption{\figlabel{bg:dio:rates}
The DIO background rate as a function of momentum threshold for different total running times.
Given a fixed running time, the total number of stopped muons is also fixed, which in turn sets the signal sensitivity and the DIO background rate.
All signal acceptance parameters were held fixed, except for the efficiency of the momentum threshold, which, when combined with the number of stopped muons, determines the \ac{ses}.
The \ac{ses} is indicated in the number along the lines in units of \num{1e-17}.
}
\end{figure}
}

\newcommand{\FigDIOEndPointComparison}{
\begin{figure}[tbp]
\centering
\includegraphics[width=0.8\textwidth,trim=0 0 0 0,clip]{figs/backgrounds/CompareDIOEndpoints.pdf}
\caption{\figlabel{bg:dio:spectra}
Comparison of the various available end-point expansions.
The red and blue lines show the parametrisations reported in the literature, whilst the black shows the digitisation of the spectrum used in SimG4.
For this study, the more conservative parametrisation from the 2011 Czarnecki paper~\cite{Czarnecki2011} has been used.
}
\end{figure}
}

\newcommand{\FigRMCExperiments}{
\begin{figure}[tbp]
\centering
%\fbox{
\includegraphics[width=0.5\textwidth]{figs/backgrounds/RMC_Gorringe_ExperimentSummary.pdf}
%}
\caption{\figlabel{bg:rmc:experiments}
Summary of experimental values of the rate of \ac{RMC} producing photons with energy greater than 57~MeV, $R_\gamma$, and the observed end-point, $k_\textrm{max}$, redacted from~\cite{RevModPhys.76.31}.
The column lablled `$\alpha$' is the neutron excess for the element, determined by: $\alpha=(A-2Z)/Z$.
}
\end{figure}
}

\newcommand{\TabRMCEndPoints}{%
\begin{table}[tb]%
%\centering
\begin{tabular}{lS[table-format=2.6]SS}%
\hline
Reaction & \multicolumn{1}{C{3cm}}{Atomic Mass of Daughter (u)} & \multicolumn{1}{C{2cm}}{$\Delta{}M$ (MeV/c$^{2}$)}&\multicolumn{1}{C{2cm}}{$\max(E_e^\textrm{RMC})$ (MeV/c$^{2}$)}\\
\hline
${}^{27}$Al$(\mu,\gamma\nu){}^{27}  $Mg     & 26.984341 &  3.12  & 101.85 \\
${}^{27}$Al$(\mu,\gamma\nu2n){}^{26}$Mg     & 25.982593 &  9.56  &  95.41 \\
${}^{27}$Al$(\mu,\gamma\nu2n){}^{25}$Mg     & 24.985837 & 20.66  &  84.31 \\
${}^{27}$Al$(\mu,\gamma\nu{}p){}^{26}$Na    & 25.992633 & 18.13  &  87.37 \\
${}^{27}$Al$(\mu,\gamma\nu{}np){}^{25}$Na   & 24.989954 & 23.71  &  81.77 \\
${}^{27}$Al$(\mu,\gamma\nu{}d){}^{25}$Na    & 24.989954 & 21.49  &  84.00 \\
${}^{27}$Al$(\mu,\gamma\nu\alpha){}^{23}$Na & 22.994467 & 15.49  &  91.01 \\
\hline
\end{tabular}
\caption{\tablabel{bg:rmc:massDifferences}%
Several potential daughter nuclei of nuclear muon capture in \textsuperscript{27}Al.
The mass of \textsuperscript{27}Al is 26.98153863~$u$, and one $u$ is taken as 931.494061~MeV/c$^2$~\cite{PDG2014}.
All masses come from~\cite{AUDI20033}.}\end{table}%
\xspace}%

\newcommand{\FigRMCSimResults}{
\begin{figure}[tbp]
\centering
%\fbox{
\includegraphics[width=0.85\textwidth]{figs/backgrounds/RMC_simResults.pdf}
%}
\caption{\figlabel{bg:rmc:simulation}
Observed electrons from a simulation of \num{6e7} \ac{RMC} photons.
The overlaid spectrum is normalised arbitrarily to fit on the plot.
}
\end{figure}
}

\newcommand{\FigRPCData}{
\begin{figure}[btp]
\centering
\subfloat[][\figlabel{bg:rpc:data:ca}Calcium]  {\includegraphics[width=0.43\textwidth]{figs/backgrounds/RPC-data-calcium.png}}\hspace{0.2cm}%
\subfloat[][\figlabel{bg:rpc:data:mg}Magnesium]{\includegraphics[width=0.53\textwidth]{figs/backgrounds/RPC-data-magnesium.png}}
\caption{\figlabel{bg:rpc:data}
Spectrum of photons coming from \acf{RPC}~\cite{Bistirlich:1972jy}.
The spectrum of manesium, which is adjacent to aluminium on the periodic table, was used as the basis of these studies.
}
\end{figure}
}

\newcommand{\FigRPCSimulatedSpectrum}{
\begin{figure}[btp]
\centering
%\fbox{%
\includegraphics[width=0.73\textwidth,trim=1cm 0.5cm 2cm 1cm,clip]{figs/backgrounds/RPC_simulated_spectrum.pdf}%
%}
\caption{\figlabel{bg:rpc:spectrum}
Digitised (red) and smoothed (blue) spectrum of \ac{RPC} from magnesium (see \fig{bg:rpc:data:mg}) used as input to the Monte Carlo simulation.
}
\end{figure}
}

\newcommand{\FigPionStopDist}{
\begin{figure}[btp]
\centering
\subfloat[][\figlabel{bg:piStop:dist:x}X-direction]{\includegraphics[width=0.32\textwidth,trim=0.2cm 0 1cm 0.7cm,clip]{figs/backgrounds/Tidied_StoppedPi-X.pdf}}\hspace{0.1cm}%
\subfloat[][\figlabel{bg:piStop:dist:y}Y-direction]{\includegraphics[width=0.32\textwidth,trim=0.2cm 0 1cm 0.7cm,clip]{figs/backgrounds/Tidied_StoppedPi-Y.pdf}}\hspace{0.1cm}%
\subfloat[][\figlabel{bg:piStop:dist:z}Z-direction]{\includegraphics[width=0.32\textwidth,trim=0.2cm 0 1cm 0.7cm,clip]{figs/backgrounds/Tidied_StoppedPi-Z.pdf}}
\caption{\figlabel{bg:piStop:dist}
Stopping distributions of pions in the target.
These distributions have considerably different forms to the muon stopping distributions shown in \fig{sense:stops}, mostly due to the different momenta of muons and pions.
}
\end{figure}
}

\newcommand{\FigPiVsMuMomenta}{
\begin{figure}[btp]
\centering
%\fbox{%
\includegraphics[width=0.9\textwidth,trim=0 0.5cm 1.3cm 0.4cm,clip]{figs/backgrounds/Tidied_MuVsPiMomentum.pdf}%
%}
\caption{\figlabel{bg:piVsMu:momenta}
The momentum of muons and pions for those that reach the target area and those that actually stop.
It is clear how the pion momenta are in general higher, including those that stop, although the maximum stopping momentum for pions is similar to that of muons.
}
\end{figure}
}

\newcommand{\FigRPCSimResults}{
\begin{figure}[btp]
\centering
%\fbox{
\subfloat[][\figlabel{bg:rpc:sim:momVtime}Momentum Vs.\ Time]
%\begin{minipage}[b]{0.45\textwidth}
%\subfloat[][\figlabel{bg:rpc:sim:time}Arrival Time]{%
%\includegraphics[width=\textwidth,trim=0.9cm 0.3cm 1cm 0.5cm,clip]{figs/backgrounds/Tidied_RPC_sim_time.pdf}}\\
%\subfloat[][\figlabel{bg:rpc:sim:mom}Momentum]{%
%\includegraphics[width=\textwidth,trim=0.9cm 0.3cm 1cm 0.5cm,clip]{figs/backgrounds/Tidied_RPC_sim_mom.pdf}}%
%\end{minipage}\hspace{1ex}
\hspace{1em}%
\subfloat[][\figlabel{bg:rpc:sim:time}Arrival Time of High-$p$ Electrons]{%
\includegraphics[width=0.49\textwidth,trim=0 0 0 2.8cm,clip]{figs/backgrounds/RPC_lifetime.png}}%
%\fbox{
%}
\caption{\figlabel{bg:rpc:sim}
Detection of secondaries from RPC photons in the target.
Although many high-momentum electrons are detected \protect\subref{fig:bg:rpc:sim:momVtime}, they are all well before the time-gated detected window \protect\subref{fig:bg:rpc:sim:time}.
}
\end{figure}
}

\newcommand{\FigAntiprotonMeco}{
\begin{figure}[tbp]
\centering
\includegraphics[width=0.6\textwidth]{figs/backgrounds/Antiproton_Meco24_energy.pdf}
\caption{\figlabel{bg:antiprotons:meco24}
Variation in the antiproton production rate as a function of incident proton energy, according to Meco note 24~\cite{Meco024} and used in the COMET TDR~\cite{TDR2016}.
For reference, protons with 8~GeV kinetic energy have 8.89~GeV/c momentum, whilst with 10.14~GeV kinetic energy their momentum is 11.038~GeV/c.
The vertical coloured lines have been added to indicate these energies, whilst the horizontal bands show the range of predicted cross sections for the models of proton-nucleon and proton-nucleus interaction.
}
\end{figure}
}

\newcommand{\FigAntiprotonData}{
\begin{figure}[tbp]
\centering
\includegraphics[width=1.0\textwidth,trim=0 0 0 0,clip]{figs/backgrounds/Antiproton_RatePerPOT_data.pdf}
\caption{\figlabel{bg:antiprotons:data}
Experimental data for antiproton production rates for 10~GeV protons~\cite{Boyarinov:1994tp,Kiselev:2012sj}.
Each line represents the cross section obtained for the four different target materials covered in those papers, scaled to match the number of nucleons of tungsten and with the additional factors of \eq{bg:antiprotons:rate} included.
}
\end{figure}
}

\newcommand{\FigAntiprotonEndpoint}{
\begin{figure}[btp]
\centering
\subfloat[][\figlabel{bg:antiprotons:end-point:tungsten}Tungsten]{\includegraphics[width=0.49\textwidth,clip=true,trim=0 0 1cm 1.7cm]{figs/backgrounds/Antiproton_Tungsten_theta_lab.pdf}}%\hspace{0.5cm}%
\subfloat[][\figlabel{bg:antiprotons:end-point:carbon}Carbon    ]{\includegraphics[width=0.49\textwidth,clip=true,trim=0 0 1cm 1.7cm]{figs/backgrounds/Antiproton_Carbon_theta_lab}}
\caption{\figlabel{bg:antiprotons:end-point}
The kinematic end-point for antiproton production as a function of the outgoing antiproton direction with respect to the incoming proton in the frame of the target nucleus (the lab frame).
The absolute end-point is only achieved when the nucleus and outgoing protons recoils coherently.
}
\end{figure}
}

\newcommand{\FigAntiprotonFits}{
\begin{figure}[tbp]
\centering
%	\fbox{
\includegraphics[width=1.0\textwidth]{figs/backgrounds/AntiprotonFits.png}
%}
\caption{\figlabel{bg:antiprotons:fits}
Piecewise fitting to experimental data and kinematic end-points.
Inlays show a zoom around the experimental data points.
}
\end{figure}
}

\newcommand{\FigAntiprotonAngularDependence}{
\begin{figure}[b]
\centering
\includegraphics[width=0.8\textwidth,trim=0 0 1.4cm 1cm,clip]{figs/backgrounds/AntiprotonAngularDependence.pdf}
\caption{\figlabel{bg:antiprotons:angular}
The angular dependence of the rate of antiproton emission, integrated over all momenta.
The different lines represent the different fits to the high momentum part of the spectrum.
The relationship given in~\cite{Boyarinov:1994tp} would suggest the data here should fit a straight line.
The dashed lines represent instead a quadratic fit to these points, which looks like a better fit.
For reweighting events the interpolated (straight solid) lines were used to be conservative.
}
\end{figure}
}

\newcommand{\TabAntiprotonRegions}{
\begin{table}[t]
\centering
\sisetup{table-number-alignment=right,table-format=1.2e3}%
\begin{tabular}{a|r|SS}
\multicolumn{1}{c|}{\multirow{2}{*}{Region}} & \multirow{2}{*}{Data Source} & \multicolumn{2}{c}{Total $\bar{p}$ per POT in this region} \\ 
                                             &                              & {Linear Tail}       &  {Exponential Tail} \\ 
\hline
                  0-59\degree    & 10\degree \cite{Kiselev:2012sj}    & 9.13e-05 & 5.26e-05 \\ 
                  59-97\degree   & 59\degree \cite{Kiselev:2012sj}    & 2.64e-08 & 4.17e-09 \\ 
                  97-119\degree  & 97\degree \cite{Boyarinov:1994tp}  & 3.40e-12 & 1.74e-12 \\ 
                  119-180\degree & 119\degree \cite{Boyarinov:1994tp} & 2.58e-12 & 5.71e-13 \\ 
\hline
\end{tabular}
\caption{\tablabel{bg:antiprotons:regions}
Angular regions and the source of the data used to build the momentum spectrum for that region.
The integrated rate for the two different high-momentum tail descriptions are also given.
Note that these values do \emph{not} contain the correction for the different incident proton energies;
for the COMET proton beam the antiproton yield is expected to be a factor 0.12 times those given here.
%The values in the final column are result of converting to rates per POT and integrating the differential cross-sections measured in \cite{Boyarinov:1994tp,Kiselev:2012sj}.
%integrated the fitted and extrapolated spectra and then integrates over the fitted angular dependence.
}
\end{table}
}

\newcommand{\FigAntiprotonSimHeightsTwoDPbar}{%
\begin{figure}[ph]%
\centering %
\subfloat[][\figlabel{bg:antiprotons:sim:2D-antip:10}Production between 0 and 59\degree]{%
\includegraphics[width=1\textwidth,trim=3.7cm 0.3cm 1.8cm 0.8cm,clip]{figs/backgrounds/Antiproton_height2D_antiproton_10.png}}\\%
\subfloat[][\figlabel{bg:antiprotons:sim:2D-antip:59}Production between 59 and 97\degree]{%
\includegraphics[width=1\textwidth,trim=3.7cm 0.3cm 1.8cm 0.8cm,clip]{figs/backgrounds/Antiproton_height2D_antiproton_59.png}}\\%
\subfloat[][\figlabel{bg:antiprotons:sim:2D-antip:97}Production between 97 and 119\degree]{%
\includegraphics[width=1\textwidth,trim=3.7cm 0.3cm 1.8cm 0.8cm,clip]{figs/backgrounds/Antiproton_height2D_antiproton_97.png}}\\%
\subfloat[][\figlabel{bg:antiprotons:sim:2D-antip:119}Production between 119 and 180\degree]{%
\includegraphics[width=1\textwidth,trim=3.7cm 0.3cm 1.8cm 0.8cm,clip]{figs/backgrounds/Antiproton_height2D_antiproton_119.png}}%
\caption{
The heights of antiprotons passing along the beamline for the four different angular regions of productions.
Each antiproton trajectory is weighted by the probability of producing an antiproton at this angle.
The colour scale on all these plots is the same.%
\figlabel{bg:antiprotons:sim:2D-antip}}%
\end{figure}%
\xspace}

\newcommand{\FigAntiprotonSimHeightsTwoDPiMin}{%
\begin{figure}[ph]%
\centering %
\subfloat[][\figlabel{bg:antiprotons:sim:2D-pi:10}Production between 0 and 59\degree]{%
\includegraphics[width=1\textwidth,trim=3.7cm 0.3cm 1.8cm 0.8cm,clip]{figs/backgrounds/Antiproton_height2D_pi-_10.png}}\\%
\subfloat[][\figlabel{bg:antiprotons:sim:2D-pi:59}Production between 59 and 97\degree]{%
\includegraphics[width=1\textwidth,trim=3.7cm 0.3cm 1.8cm 0.8cm,clip]{figs/backgrounds/Antiproton_height2D_pi-_59.png}}\\%
\subfloat[][\figlabel{bg:antiprotons:sim:2D-pi:97}Production between 97 and 119\degree]{%
\includegraphics[width=1\textwidth,trim=3.7cm 0.3cm 1.8cm 0.8cm,clip]{figs/backgrounds/Antiproton_height2D_pi-_97.png}}\\%
\subfloat[][\figlabel{bg:antiprotons:sim:2D-pi:119}Production between 119 and 180\degree]{%
\includegraphics[width=1\textwidth,trim=3.7cm 0.3cm 1.8cm 0.8cm,clip]{figs/backgrounds/Antiproton_height2D_pi-_119.png}}%
\caption{
The heights of secondary pions passing along the beamline produced from antiprotons in each of the four different angular regions of productions.
Each trajectory is weighted by the probability of producing the parent antiproton in its initial direction at the target.
\figlabel{bg:antiprotons:sim:2D-pi}}%
\end{figure}%
\xspace}%

\newcommand{\FigAntiprotonSimFluxes}{
\begin{figure}[b]
\centering 
\subfloat[][\figlabel{bg:antiprotons:sim:fluxes:antip}Unweighted Antiproton Survival Probability]{
\includegraphics[width=1\textwidth,trim=0.7cm 0 1.9cm 0.2cm,clip]{figs/backgrounds/Antiproton_fluxes.pdf}}\\
\subfloat[][\figlabel{bg:antiprotons:sim:fluxes:pion}Unweighted Secondary Pion Transport Probability]{
\includegraphics[width=1\textwidth,trim=0.7cm 0 1.9cm 0.2cm,clip]{figs/backgrounds/Antiproton_fluxes-pions.pdf}}
\caption{\figlabel{bg:antiprotons:sim:fluxes}
The surivival probability of antiprotons and secondaries pions per antiproton produced in the target as a function of distance along the beamline.
These plots are not weighted by the probability that an antiproton is produced at a particular angle.
From left to right the vertical gray lines indicate the production target, Torus1 entrance, and the Torus2 exit.
}
\end{figure}
}

\newcommand{\FigAntiprotonSimTime}{
\begin{figure}[b!]
\centering 
\subfloat[][\figlabel{bg:antiprotons:sim:time:antip}Timing of Antiprotons]{
\includegraphics[width=0.485\textwidth,trim=0.5cm 0.9cm 0.5cm 0.9cm,clip]{figs/backgrounds/Antiproton_timing_antiprotons.pdf}}
\hspace{1ex}\subfloat[][\figlabel{bg:antiprotons:sim:time:pion}Timing of Secondary Pions]{
\includegraphics[width=0.485\textwidth,trim=0.5cm 0.9cm 0.5cm 0.9cm,clip]{figs/backgrounds/Antiproton_timing_pions.pdf}}
\caption{\figlabel{bg:antiprotons:sim:time}
%\CHECK{Add pion stopping timing to RPC section to be able to compare to it here}
	The arrival time of antiprotons~\protect\subref{fig:bg:antiprotons:sim:time:antip} and pions~\protect\subref{fig:bg:antiprotons:sim:time:pion}
	at various points along the beamline and for the different initial antiproton directions.
Note that the x-axis scales are different.
Whilst the timing of antiprotons themselves is very delayed, the timing of secondary pions, which are produced predominantly at the production target, is relatively prompt and will be effective at suppressing the induced backgrounds.
}
\end{figure}
}

%\newcommand{\FigAntiprotonSimFluxes}{
%\begin{figure}[ph]
%\centering 
%\subfloat[][\figlabel{bg:antiprotons:sim:fluxes:10}Production between 0 and 59\degree]{
%\includegraphics[width=1\textwidth,trim=0.7cm 0 1.9cm 0.2cm,clip]{figs/backgrounds/Antiproton_flux_10.pdf}}\\
%\subfloat[][\figlabel{bg:antiprotons:sim:fluxes:59}Production between 59 and 97\degree]{
%\includegraphics[width=1\textwidth,trim=0.7cm 0 1.9cm 0.2cm,clip]{figs/backgrounds/Antiproton_flux_59.pdf}}\\
%\subfloat[][\figlabel{bg:antiprotons:sim:fluxes:97}Production between 97 and 119\degree]{
%\includegraphics[width=1\textwidth,trim=0.7cm 0 1.9cm 0.2cm,clip]{figs/backgrounds/Antiproton_flux_97.pdf}}\\
%\subfloat[][\figlabel{bg:antiprotons:sim:fluxes:119}Production between 119 and 180\degree]{
%\includegraphics[width=1\textwidth,trim=0.7cm 0 1.9cm 0.2cm,clip]{figs/backgrounds/Antiproton_flux_119.pdf}}
%\caption{\figlabel{bg:antiprotons:sim:fluxes}
%The surivival probability of antiprotons and their secondaries per antiproton produced in the target as a function of distance along the beamline.
%From left to right the vertical magenta lines indicate the production target, Torus1 entrance, and the Torus2 exit.
%}
%\end{figure}
%}

\newcommand{\FigAntiprotonSimPiMom}{
\begin{figure}[tb]
\centering 
\includegraphics[width=0.9\textwidth,trim=2.0cm 0 0 0,clip]{figs/backgrounds/Antiproton_pion_momentum.pdf}
\caption{\figlabel{bg:antiprotons:sim:piMom}
Momentum of pions passing the exit of Torus1 (90\degree around the bent muon beam transport solenoid) compared to pions in the main muon beam (which is arbitrarily normalised).
}
\end{figure}
}

\newcommand{\HeaderPi}[1]{\multicolumn{1}{#1}{Torus1 $\pi^-$}}
\newcommand{\HeaderPBar}[1]{\multicolumn{1}{#1}{$\bar{p}$ Stop}}
%\newcommand{\TabAntiprotonResults}{
%\begin{table}[tb]
%\centering
%\begin{tabular}{a|SS|SS|SS|}
%\multicolumn{1}{c|}{Region} & \multicolumn{2}{c|}{Observed Events} & \multicolumn{2}{c|}{Weighted Mean per $\bar{p}$}& \multicolumn{2}{c}{Rate per \ac{POT}}  \\
%\multicolumn{1}{c|}{}       & \HeaderPi{r}    & \HeaderPBar{r|}    & \HeaderPi{r}      & \HeaderPBar{r|}      &\HeaderPi{r}     & \HeaderPBar{r}                  \\
%\hline
%   0-59\degree &50&0&2.5e-4&& \\
%  59-97\degree &31&0&1.6e-4&& \\
% 97-119\degree &36&0&1.8e-4&& \\
%119-180\degree &64&9&3.2e-4&& \\
%\hline
%\multicolumn{1}{c|}{Total} & & & & & \\
%\hline
%\end{tabular}
%\caption{\tablabel{bg:antiprotons:results}
%Results of the antiproton simulation.
%`Torus1 $\pi^-$' are the pions that pass the exit of Torus1, which is 90\degree round the muon beamline.
%`$\bar{p}$ Stop' refers to the number of antiprotons that stop in the muon Stopping Target.
%The weighted mean is the sum of the observed events weighted by the production probability given the initial antiproton direction, divided by the total number of input antiprotons.
%Finally the Rate per \ac{POT} is weighted mean scaled by the number of antiprotons produced for this region per POT (last column of \tab{bg:antiprotons:regions}).
%}
%\end{table}
%}

%\newcommand{\TabAntiprotonResultsPiSecond}{
%\begin{table}[tb]
%\centering
%\begin{tabular}{a|SSS}
%\multicolumn{1}{c|}{Region} & \multicolumn{1}{c}{Observed Events} & \multicolumn{1}{c}{Weighted Mean per $\bar{p}$}& \multicolumn{1}{c}{Rate per \ac{POT}}  \\
%\hline
%   0-59\degree &50&2.5e-4&& \\
%  59-97\degree &31&1.6e-4&& \\
% 97-119\degree &36&1.8e-4&& \\
%119-180\degree &64&3.2e-4&& \\
%\hline
%\multicolumn{1}{c|}{Total} & & & & & \\
%\hline
%\end{tabular}
%\caption{\tablabel{bg:antiprotons:results}
%Results of the antiproton simulation.
%`Torus1 $\pi^-$' are the pions that pass the exit of Torus1, which is 90\degree round the muon beamline.
%`$\bar{p}$ Stop' refers to the number of antiprotons that stop in the muon Stopping Target.
%The weighted mean is the sum of the observed events weighted by the production probability given the initial antiproton direction, divided by the total number of input antiprotons.
%Finally the Rate per \ac{POT} is weighted mean scaled by the number of antiprotons produced for this region per POT (last column of \tab{bg:antiprotons:regions}).
%}
%\end{table}
%}

%\newcommand{\TabAntiprotonResultsTrans}{
%\begin{table}[p]
%\centering
%\begin{tabular}{a|SSS|SSS|S|}
% &\multicolumn{3|}{c}{Raw count} &\multicolumn{3|}{c}{Weighted Probability} & \\
% &\multicolumn{1}{p{0.6cm}}{Entr.}&\multicolumn{1}{p{0.6cm}}{Mid.}&\multicolumn{1}{p{0.6cm}|}{Target}&\multicolumn{1}{p{0.6cm}}{Entr.}&\multicolumn{1}{p{0.6cm}}{Mid.}&\multicolumn{1}{p{0.6cm}}{Target}&\multicolumn{1}{p{0.6cm}}{$P(\textrm{Target}|\textrm{Entr.}$}\\
%\hline
%\multicolumn{1}{l}{Pions} & \multicolumn{7}{p{6cm}}{}\\
%   0-59\degree    & 16943  & 1452  & 1  & 5.77E-06 & 5.10E-07 & 9.98E-11 & 1.73E-05 \\ 
%   59-97\degree   & 87230  & 7157  & 18 & 4.64E-09 & 4.06E-10 & 2.99E-13 & 6.45E-05 \\ 
%   97-119\degree  & 157385 & 13041 & 25 & 4.10E-14 & 3.59E-15 & 5.02E-18 & 1.22E-04 \\ 
%   119-180\degree & 222486 & 12997 & 30 & 1.25E-17 & 9.59E-19 & 1.91E-21 & 1.53E-04 \\ 
%\hline
%\multicolumn{1}{l}{Antiprotons} & \multicolumn{7}{p{6cm}}{}\\
%0-59\degree    & 7      & 3     & 0    & 1.40E-09 & 5.14E-11 & 3.23E-12                  &          \\ 
%59-97\degree   & 270    & 58    & 8    & 1.70E-12 & 4.81E-13 & 1.16E-14                  & 2.41E-02 \\ 
%97-119\degree  & 2907   & 830   & 114  & 1.10E-16 & 3.31E-17 & 5.17E-19                  & 1.56E-02 \\ 
%119-180\degree & 278105 & 20787 & 2237 & 2.27E-19 & 5.20E-20 & 7.74E-21                  & 1.49E-01 \\ 
%               &        &       &      &          &          & \multicolumn{1}{c}{Mean=} & 6.29E-02 \\ 
%%\hline
%\end{tabular}
%\end{table}
%}

\newcommand{\TabAntiprotonResultsPiSecond}{%
\sisetup{table-number-alignment=right, table-format=2.2e2}%
\begin{table}[p]%
\centering%
%\begin{adjustwidth}{-0.7cm}{}%
\begin{tabular}{lr|S!{\VRule}S!{\VRule}S!{\VRule}S|}%
\multicolumn{2}{c|}{\multirow{2}{3cm}{Rates for Secondary $\pi^-$}} & \multicolumn{4}{c|}{Secondary $\pi^-$ from Angular Region for Antiproton Production} \\ 
                              &  & \multicolumn{1}{a!{\VRule}}{0-59\degree} & \multicolumn{1}{a!{\VRule}}{59-97\degree} & \multicolumn{1}{a!{\VRule}}{97-119\degree} & \multicolumn{1}{a|}{119-180\degree} \\ 
\hline
\multirow{3}{1.6cm}{Raw Counts} & Torus1 & \multicolumn{1}{r!{\VRule}}{16943} & \multicolumn{1}{r!{\VRule}}{87230} & \multicolumn{1}{r!{\VRule}}{157385} & \multicolumn{1}{r|}{222486} \\ 
                               % & TS3 & \multicolumn{1}{r!{\VRule}}{1452}  & \multicolumn{1}{r!{\VRule}}{7157}  & \multicolumn{1}{r!{\VRule}}{13041}  & \multicolumn{1}{r|}{12997}  \\ 
                                & Torus2 & \multicolumn{1}{r!{\VRule}}{208}   & \multicolumn{1}{r!{\VRule}}{1106}  & \multicolumn{1}{r!{\VRule}}{1995}   & \multicolumn{1}{r|}{1847}   \\ 
                                & Target & \multicolumn{1}{r!{\VRule}}{1}     & \multicolumn{1}{r!{\VRule}}{18}    & \multicolumn{1}{r!{\VRule}}{25}     & \multicolumn{1}{r|}{30}     \\ 
\hline
\multirow{3}{1.6cm}{Weighted Mean} & Torus1  &6.93E-07 & 5.57E-10 & 4.92E-15 & 1.50E-18\\ 
                                %  & TS3  &6.12E-08 & 4.87E-11 & 4.31E-16 & 1.15E-19 \\
                                  & Torus2  &1.03E-08 & 7.11E-12 & 6.95E-17 & 1.47E-20\\ 
                                  & Target  &1.20E-11 & 3.59E-14 & 6.02E-19 & 2.29E-22\\ 
%\hline
%\multicolumn{2}{r|}{$P(\textrm{Target}|\textrm{Torus2})$}  & 0.039 & 0.040 &  0.022 & 0.017 \\
\hline
\end{tabular}%
%\end{adjustwidth}%
\caption{
Secondary pion fluxes from antiprotons observed at key points along the beamline.
See caption to \tab{bg:antiprotons:sim:antip} for a description of the column and row contents.
%The raw counts are the total observed events from the simulation of 80M antiprotons for each of the four angular regions.
%The weighted sum then shows the weighted sum over every particle passing the point, where the weight is determined as described by equation~\eq{bg:antiprotons:factorisation}.
\tablabel{bg:antiprotons:sim:pi}}
\end{table}\xspace}

\newcommand{\TabAntiprotonResultsAntip}{%
\sisetup{table-number-alignment=right, table-format=2.2e2}%
\begin{table}[p]%
%\begin{adjustwidth}{-0.7cm}{}%
\begin{tabular}{lr|S!{\VRule}S!{\VRule}S!{\VRule}S|}%
\multicolumn{2}{c|}{\multirow{2}{3cm}{Rates for $\bar{p}$ Transport}} & \multicolumn{4}{c|}{Angular Region for Antiproton Production} \\ 
                              &  & \multicolumn{1}{a!{\VRule}}{0-59\degree} & \multicolumn{1}{a!{\VRule}}{59-97\degree} & \multicolumn{1}{a!{\VRule}}{97-119\degree} & \multicolumn{1}{a|}{119-180\degree} \\ 
\hline
\multirow{3}{1.6cm}{Raw Counts} & Torus1 &\multicolumn{1}{r!{\VRule}}{7} & \multicolumn{1}{r!{\VRule}}{270} & \multicolumn{1}{r!{\VRule}}{2907} & \multicolumn{1}{r|}{278105} \\ 
                                & Torus2 &\multicolumn{1}{r!{\VRule}}{3} & \multicolumn{1}{r!{\VRule}}{58 } & \multicolumn{1}{r!{\VRule}}{830 } & \multicolumn{1}{r|}{20787 } \\ 
                                %& Torus2 Collim. & \multicolumn{1}{r!{\VRule}}{1} & \multicolumn{1}{r!{\VRule}}{39 } & \multicolumn{1}{r!{\VRule}}{431 } & \multicolumn{1}{r|}{9024  } \\ 
                                & Target &\multicolumn{1}{r!{\VRule}}{0} & \multicolumn{1}{r!{\VRule}}{8  } & \multicolumn{1}{r!{\VRule}}{114 } & \multicolumn{1}{r|}{2237  } \\ 
\hline
\multirow{3}{1.6cm}{Weighted Mean} & Torus1 & 1.68E-10    & 2.04E-13 & 1.32E-17 & 2.72E-20 \\ 
                                   & Torus2 & 6.16E-12 & 5.77E-14 & 3.97E-18 & 6.23E-21\\
                                   %& Torus2 Collim. & 1.95E-15    & 6.18E-15 & 1.85E-18 & 3.15E-21 \\ 
                                   & Target & {*}4.39E-16 & 1.39E-15 & 6.20E-20 & 9.29E-22 \\ 
\hline                                                           
\multicolumn{2}{r|}{$P(\textrm{Target}|\textrm{Torus2})$} &{*}0.225& 0.225  & 0.034  & 0.295\\
\hline
\end{tabular}%
%\end{adjustwidth}%
\caption{
Antiproton stopping rates and fluxes at key points: at the entrance to the bent
solenoids (Torus1), just before the final beam collimator (Torus2), and in front of the
stopping target (Target).
Raw counts are the total number of particles seen in each simulation of 80
million antiprotons. The weighted mean is the average of all particle
weights, given by the energy correction factor $\xi$
and the antiproton production angle, $\Phi(\theta,\phi)$.
$P(\textrm{Target}|\textrm{Torus2})$, gives the survival
probability for an antiproton to reach the target, given that it reached the Torus2 collimator.
Since no antiprotons were seen stopping for the angular region from 0 to
59\degree, the value for the final entry in that column---indicated
with an asterisk (*)---are obtained by multiplying the antiproton rate
expected at the Torus2 collimator with the median collimator acceptance from
the other three angular regions.%
\tablabel{bg:antiprotons:sim:antip}}
\end{table}\xspace}

%\newcommand{\TabAntiprotonFactors}{
%\begin{table}[tb]
%\centering
%        \begin{tabular}{llm{0.5\textwidth}}
%	\hline
%        Parameter & \multicolumn{1}{l}{Value} & Description \\
%	\hline
%        $R_{\pi/p}$                & \VarPiStopsPerPOT & Pion stopping rate per \ac{POT}  \\ 
%        $\mathcal{B}_\textrm{RPC}$ & \num{2.27e-2} & Branching ratio of \ac{RPC} \\ 
%	$f_{e,\textrm{RPC}}$           & \VarDetectedEsPerRPC & Probability of an RPC photon producing signal-like electrons in the detector \\ 
%	$A_\textrm{time}$              & \VarRPCTimingEfficiency & Acceptance of timing window to secondary electrons from RPC \\ 
%        $\epsilon_\textrm{extinction}$ & \VarExtinctionFactor[2] &  Extinction factor\\ 
%	\hline
%\end{tabular}
%\caption{\tablabel{bg:antiprotons:factors}
%Parameters and their values in the determination of the \ac{RPC} background rate.
%}
%\end{table}
%}

\newcommand{\TabAntiprotonEstimates}{
\sisetup{table-number-alignment=right,table-format=2.2e3,table-comparator=true}%
\begin{table}[b]
%\begin{adjustwidth}{-0.9cm}{}
\begin{tabular}{la|S|SSS|}
            &         & \multicolumn{1}{c|}{Stopping Rates} & \multicolumn{3}{c|}{Background Rate per POT} \\ 
                           &          &            & {No Timing} & {Prompt} & {Delayed} \\%& {Unweighted}
\hline                                                                              %              
\multirow{4}{*}{$\bar{p}$} & 0-59    & 4.39E-16 & 2.03E-22 & {-}      & 1.04E-22 \\ % & <6.58E-13
                           & 59-97   & 1.39E-15 & 6.43E-22 & {-}      & 3.30E-22 \\ % & 4.17E-16
                           & 97-119  & 6.20E-20 & 2.87E-26 & {-}      & 1.47E-26 \\ % & 2.48E-18
                           & 119-180 & 9.29E-22 & 4.29E-28 & {-}      & 2.20E-28 \\ % & 1.60E-17
\hline
\multirow{4}{*}{$\pi^-$}   & 0-59    & 1.20E-11 & 3.54E-18 & 3.54E-29 & 1.95E-30 \\ % & 6.58E-13
                           & 59-97   & 3.59E-14 & 1.06E-20 & 1.06E-31 & 5.83E-33 \\ % & 9.38E-16
                           & 97-119  & 6.02E-19 & 1.78E-25 & 1.78E-36 & 9.78E-38 \\ % & 5.44E-19
                           & 119-180 & 2.29E-22 & 6.78E-29 & 6.78E-40 & 3.72E-41 \\ % & 2.14E-19
\hline                                            
 \multicolumn{3}{r|}{Sum (per POT)}              & 3.56E-18 & 3.56E-29 & 4.34E-22\\           
\hline
\end{tabular}
%\end{adjustwidth}
\caption{\tablabel{bg:antiprotons:estimates}
Final estimated antiproton-induced background rates.
See text for column definitions.
}
\end{table}\xspace}

\newcommand{\FigBgBeamMomVsTime}{
\begin{figure}[t]
\centering 
\subfloat[][\figlabel{bg:beam:MomVsTime:strawTrk}$e^-$ at the Straw Tracker]{
\includegraphics[width=0.45\textwidth,trim=0 0 0 1.7cm,clip]{figs/backgrounds/Beam_TimeVsMomentum_StrawTkt.pdf}}
\subfloat[][\figlabel{bg:beam:MomVsTime:stopTgt}$e^-$ after the Beam Blocker]{
\includegraphics[width=0.45\textwidth,trim=0 0 0 1.7cm,clip]{figs/backgrounds/Beam_TimeVsMomentum_StopTgt.pdf}}
\caption{\figlabel{bg:beam:MomVsTime}
The timing and momentum of electrons detected at the straw tracker and passing a plane immediately after the beam blocker.
}
\end{figure}
}

\newcommand{\TabBgBeamFactors}{
\begin{table}[tb]
\centering
        \begin{tabular}{llm{0.5\textwidth}}
	\hline
        Parameter & \multicolumn{1}{l}{Value} & Description \\
	\hline
        $A_\textrm{geom}$ & \VarBeamBgGeometric &  Geometric acceptance to high-momentum electrons per \ac{POT}\\ 
        $A_\textrm{mom}$ & \VarBeamBgMomentum &  Fraction of beam electrons within the signal momentum cuts \\ 
        $A_\textrm{time}^\textrm{delayed}$ & \VarBeamBgTiming &  Fraction of beam electrons delayed until the detector window \\ 
	\hline
\end{tabular}
\caption{\tablabel{bg:beam:factors}
Parameters and their values in the determination of the background rate due to high-energy particles in the beam.
}
\end{table}
}

\newcommand{\FigBgBeamExtrapolate}{
\begin{figure}[t]
\centering 
\subfloat[][\figlabel{bg:beam:acceptance:momentum}Momentum of electrons]{
\includegraphics[width=0.49\textwidth,trim=0 1.5cm 0 0.7cm,clip]{figs/backgrounds/Beam_Acceptance_momentum.pdf}}
\subfloat[][\figlabel{bg:beam:acceptance:time}Time of high-$p$ electrons]{
\includegraphics[width=0.49\textwidth,trim=0 1.5cm 0 0.7cm,clip]{figs/backgrounds/Beam_Acceptance_time.pdf}}
\caption{\figlabel{bg:beam:acceptance}
Projections of the momentum and arrival time of electrons.
\protect\subref{fig:bg:beam:acceptance:momentum} shows the momentum for electrons that arrive at any time, 
whereas \protect\subref{fig:bg:beam:acceptance:time} only shows the arrival time for electrons with momentum above 65~MeV/c.
Magenta lines in \protect\subref{fig:bg:beam:acceptance:time} indicate the mean arrival time after the beam blocker and the mean and maximum arrival time
at the straw tracker.
The black line is an exponential fit to the momentum and time of electrons after the beam blocker, with the equation and constants shown next to the fit.
%Below this momentum, electrons from muon decay-in-orbit dominate the timing.
}
\end{figure}
}

\newcommand{\FigBgCosmicPrimary}{
\begin{figure}[t]
\centering
\subfloat[][\figlabel{bg:cosmics:primary:momentum}Momentum]{
\includegraphics[width=0.47\textwidth,trim=0 0 0 0,clip]{figs/backgrounds/Cosmics_primary_momentum.pdf}}
\subfloat[][\figlabel{bg:cosmics:primary:transverse}Transverse Direction Cosine]{
\includegraphics[width=0.47\textwidth,trim=0 0 0 0,clip]{figs/backgrounds/Cosmics_primary_transverse.pdf}}
\caption{\figlabel{bg:cosmics:primary}
Distributions of momenta and the transverse direction for the cosmic muon and antimuon fluxes provided by the ND280 collaboration.
}\end{figure}\xspace}

\newcommand{\FigBgCosmicBeam}{
\begin{figure}[t]
\centering
\includegraphics[width=0.98\textwidth,trim=2.5cm 0 2.3cm 0,clip]{figs/backgrounds/Cosmics_highP_electrons.png}
\caption{\figlabel{bg:cosmics:beam}
Projection onto beamline coordinate system of electrons from cosmic rays with momenta greater than 100~MeV/c, from a simulation of about 170 million cosmic muons.
Three tracks can be seen to enter the electron spectrometer and pass along, although their momenta are all above 105~MeV/c as visible from the upwards drift of the trajectories.
}\end{figure}\xspace}

\newcommand{\FigBgCosmicMomenta}{
\begin{figure}[t]
\centering
\includegraphics[width=0.9\textwidth,trim=0 0 1.6cm 0,clip]{figs/backgrounds/Cosmics_strawTrackHits.pdf}
\caption{\figlabel{bg:cosmics:momenta}
Momentum of electrons that hit one and two layers (planes) of the straw tracker.
The vertical axis shows the raw number of observed events based on the simulation of 170 million cosmic muons described in the text.
}\end{figure}\xspace}

\newcommand{\TabBgCosmicParameters}{
\begin{table}[t]
\centering
\begin{tabular}{llm{0.7\textwidth}}
	\hline
        Parameter & \multicolumn{1}{l}{Value} & Description \\
	\hline
        $R_\mu$                     & \num{4.17e5} & Cosmic ray flux through entire hall per second         \\ 
        $n_e^\textrm{obs}$          & 0.46         & Observed electrons per MeV/c                           \\ 
        $N_\textrm{sim}$            & \num{1.76e8} & Number of simulated events                             \\ 
        $(1-\epsilon_\textrm{CRV})$ & \num{1e-4}   & Miss rate of the \ac{CRV}                              \\ 
        $A_\textrm{duty}$           & 0.36         & Accelerator duty factor                                \\ 
        $A_\textrm{timing}$         & 0.51         & Fraction of time within gated-time detection window \\ 
        $A_\textrm{TDAQ,Recon}$     & 0.70         & Combined acceptance of TDAQ and Reconstruction         \\ 
	\hline
\end{tabular}
\caption{\tablabel{bg:cosmics:params}
Parameters and their values for the determination of the cosmic ray background rate.
}
\end{table}
}

\newcommand{\TabBackgroundFinalVals}{
\begin{sidewaystable}[b]
\centering
\sisetup{table-number-alignment=right,table-format=1.2e3}%
\renewcommand{\arraystretch}{1.5} % Default value: 1
%\begin{adjustwidth}{-0.7cm}{}%
\begin{tabular}{ll|SSS|S|L{7cm}}
\toprule
Type                         & Source                   & \multicolumn{3}{c|}{Background Rate}                                       & \multicolumn{1}{c|}{Total Events} & Comment \\ 
                             &                          & {per $\mu^-$ stop} & {per POT}              & {per second}            &            \\ 
\toprule
\multirow{2}{*}{Intrinsic}   & DIO                      & \NumDIOPerMuStop & \NumDIOPerPOT                    & \NumDIOPerSecond                 & \NumDIOTotal                     & \\ 
			     & RMC                      & \NumRMCPerMuStop & \NumRMCPerPOT                    & \NumRMCPerSecond                 & \NumRMCTotal                     & \\ 
\cmidrule{2-7}
\multirow{4}{*}{Delayed}     & RPC                      & {--}             & \NumRPCDelayedPerPOT             & \NumRPCDelayedPerSec             & \NumRPCDelayedTotal              &  \\ 
			     & Beam                     & {--}             & \NumBeamBgDelayedPerPOT          & \NumBeamBgDelayedPerSec          & \NumBeamBgDelayedTotal           & \multirow{1}{7cm}{\footnotesize Beam includes high-energy electrons from $\pi$, $\mu$, and $n$ capture.} \\ 
			     & Stopped $\bar{p}$        & {--}             & \NumBgAntiprotonsPerPOT          & \NumBgAntiprotonsPerSec          & \NumBgAntiprotonsTotal           & \multirow{2}{7cm}{\footnotesize Based on conservative interpolation and extrapolation of limited experimental $\bar{p}$ data.}\\ 
                             & $\pi^{-}$ from $\bar{p}$ & {--}             & \NumBgAntiprotonsPiDelayedPerPOT & \NumBgAntiprotonsPiDelayedPerSec & \NumBgAntiprotonsPiDelayedTotal  & \\ 
\cmidrule{2-7}
\multirow{3}{*}{Prompt}      & RPC                      & {--}             & \NumRPCPromptPerPOT              & \NumRPCPromptPerSec              & \NumRPCPromptTotal               &  \\ 
                             & Beam                     & {--}             & \NumBeamBgPromptPerPOT           & \NumBeamBgPromptPerSec           & \NumBeamBgPromptTotal            &  \\ 
                             & $\pi^{-}$ from $\bar{p}$ & {--}             & \NumBgAntiprotonsPiPromptPerPOT  & \NumBgAntiprotonsPiPromptPerSec  & \NumBgAntiprotonsPiPromptTotal   & \\ 
\cmidrule{2-7}
\multicolumn{2}{c|}{Cosmics}                            & {--}             & {--}                             & \NumCosmicRatePerSecond          & \NumCosmicRateTotal              & \footnotesize Dominated by conservative miss-rate. \\ 
\bottomrule                                                                                                   
\multicolumn{2}{c|}{Total}                              & {--}             & {--}                             & \NumTotalBgPerSecond             & \NumTotalBgPhasII                &  \\ 
\bottomrule
\end{tabular}
%\end{adjustwidth}
\caption{ \tablabel{bg:summary}%
Final predicted background rates and events.
The dominant backgrounds are cosmic ray electrons and stopping antiprotons, which are predicted to have produce almost equal rates.
Assumes an extinction factor of \VarExtinctionFactor, a double-Gaussian momentum resolution with a core resolution of $\sigma=200$~keV/c and a tail probability of \VarResolutionProbFromRMCToSignal that events are misreconstructed with at least an additional 2.4~MeV/c. 
All other parameters are the same as stated in chapter~\sect{sense}.
}
\end{sidewaystable}%
}

\newcommand{\FigBgVsResolution}{
\begin{figure}[t]
\centering
\includegraphics[width=0.9\textwidth,clip=true,trim=0 0 0 1cm]{figs/backgrounds/BG_vs_resolution.pdf}
\caption{\figlabel{bg:resolution}
Dependence of the intrinsic background rates on $\sigma_2$, the width of the high/low momentum tail in the reconstruction distribution.
From this it can be seen that the background rates from DIO will always dominate, compared to RMC.
}\end{figure}\xspace}

%% Important variables:
% Number of POT
% Number of muons stopped
% Total run time
\chapter{COMET \phaseII: Backgrounds}
\sectlabel{backgrounds}
Having optimised and evaluated the signal sensitivity, it is now important to check the expected background rates.
Interpreting the final result is only possible if the number of background events has been predicted.
The certainty of any observation or the stringency of a final confidence limit is determined by the relative background rate and signal sensitivity; ideally the background rate should be small, well below a single expected event.

The types of background that must be considered for COMET were outlined in \sect{detector:background}, but in this section they will be evaluated for the optimised experiment design using the improved simulation.

\FigDIOBackground
\section{Muon Decay in Orbit (\acs{DIO})}
\sectlabel{bg:dio}
%\begin{easylist}
%# Discussed in previous section \sect{sense:momentum}
%# DIO spectrum near end-point: which expansion is correct?
%# Final DIO rate above momentum threshold with resolution
%\end{easylist}
In order to decide the optimal threshold for the momentum cut, it was necessary, in section \sect{sense:momentum}, to study the rate of \ac{DIO} events.
With a low-momentum threshold of \VarMomThreshold, the number of expected DIO events per muon stop is \VarDIOPerMuStop{}.
Given the \VarTotalMuStops~muon stops that should take place during \phaseII, the total expected number of background events due to DIO is \VarDIOTotal.

However, it is important to note just how steeply falling the DIO rate is in this region.
\Fig{bg:dio:rates} shows how, given a fixed run time, the number of DIO background events is affected by changing the low-momentum threshold.
Changing the momentum threshold also affects the signal acceptance, so that for a fixed run time the \ac{ses} is also affected.
Simultaneously, with more muons stopped the \ac{DIO} rate will increase for a fixed momentum threshold.
\Fig{bg:dio:rates} was produced using the same DIO and signal spectra shown in \fig{sense:spectra:resolution}, which included energy losses in the material of the target, beamline, and detector, as well as a 200~keV/c Gaussian resolution function. 
In addition, all the acceptance parameters of the sensitivity chapter were held fixed, except for the momentum cut efficiency.

\FigDIOEndPointComparison
The fact that the run-time, signal acceptance, and background rate depend so strongly on the momentum threshold makes the theoretical prediction for the \ac{DIO} spectrum particularly important.
The two most recent calculations of the high-energy tail of the DIO spectrum are shown in \fig{bg:dio:spectra}, as well as the cruder function that fits the whole spectrum range, used in SimG4.
The more conservative spectrum from the 2011 paper~\cite{Czarnecki2011} has been used in this study.
However, if one were to use the more recent 2015 spectrum~\cite{Czarnecki2014}, background rates due to DIO would fall to 0.057 events at the same momentum threshold.
This would agree with that paper's statement that the inclusion of radiative corrections suppresses the DIO background by 15\% at the end-point.

In total, based on the 2011 DIO spectrum, \sci{3}{-18} electrons would be produced per muon stop with momentum greater than \VarMomThreshold.
Therefore, the DIO rejection efficiency, including both geometric effects and the threshold on the detected momentum, suppresses the detection of DIO end-point electrons by about 98\%.

\section{Radiative Muon Capture (\acs{RMC})}
\TabRMCEndPoints%
\sectlabel{bg:rmc}
During the process of nuclear muon capture, there is a finite probability of a hard photon being radiated from the muon, nucleus, or the exchanged $W$-boson.
This is known as \acf{RMC} and is distinguished from radioactive gamma-ray production during nuclear de-excitation or decay of the daughter nucleus.
The maximum energy this photon can take, $\max(E_e^\textrm{RMC})$, differs from the \mueconv signal and \ac{DIO} end-point by the minimum energy needed to change the nucleus from $N(A,Z)$ to $N(A,Z-1)$:
\begin{align}
\max(E_e^\textrm{RMC})=&\big(M_\mu - B_{\mu,\textrm{binding}} - E_\textrm{rec}\big)-\big(M_{N_2}+\sum_iM_{h_i}-M_{N_1}-\Delta{}Z\cdot{}M_e\big),\\
                      =&E^\textrm{Conversion}_e  - \Delta{}M,
\end{align}
with $M_{N_1}$ and $M_{N_2}$ the mass of the parent and daughter nuclei respectively, and $M_{h_i}$ the mass of the $i$-th hadron (proton, neutron, alpha, \etc).
The mass of the electron, $M_e$, must also be included since a free electron must be ejected from the atomic orbitals for every proton removed from the nucleus, equivalent to $\Delta{}Z$.
$\Delta{}M$ is then the total energy lost to changes in the atomic mass%
\footnote{
One has to think of the atomic mass difference, rather than the nuclear mass difference.
The distinction is important since consideration of the nuclear masses alone would ignore the effect on the atomic electrons.
Under nuclear muon capture (radiative or otherwise), the number of protons in the nucleus is reduced by at least one, and accordingly atomic electrons become unbound.
The notation here follows that of the COMET TDR~\cite{TDR2016} and CDR~\cite{CDRphase2}, where the electron mass is absorbed into the value of $\Delta{}M$ (although there it is called $\Delta_{Z-1}$).
}
of the nuclei and other emitted hadrons.

When an aluminium-27 nucleus captures a muon various daughter nuclei are possible.
If no other particles are emitted, as part of direct nucleon capture, the daughter nucleus will be magnesium-27.
In general, this could be left in an excited state, but to reach the end-point of the \ac{RMC} spectrum it will be left in the ground-state configuration.
The atomic mass difference between these two nuclei is shown in \tab{bg:rmc:massDifferences}, where it can be seen that the RMC end-point is separated from the \mueconv signal energy by around 3~MeV.

\FigRMCExperiments
%\newcolumntype{d}[1]{D{.}{\cdot}{#1} }

\Tab{bg:rmc:experiments} shows a summary of the experimental data on aluminium, taken from the summary by Gorringe~\cite{RevModPhys.76.31}.
It is interesting to note that in all experiments to date none of the empirical fits to RMC have suggested an end-point above 90.1~MeV~\cite{PhysRevC.37.1633,PhysRevC.46.1094,PhysRevC.59.2853}.
This is close to the end-point predicted by the various transitions besides ${}^{27}$Al$(\mu,\gamma\nu){}^{27}$Mg.
The two most recent of these experiments both measure the branching ratio for \ac{RMC} producing photons with $E>57~$MeV to ordinary muon capture to be \sci{1.43}{-5}. 

To produce a background event in the detector, the high-energy photons produced from RMC must be converted to a high-energy electron.
Asymmetric pair production from this photon is one mechanism by which this can occur, although producing the positron at rest is highly suppressed.
In addition, around 1~MeV is consumed by the mass of the electron-positron pair.
Compton scattering provides another mechanism by which the high-energy photon can convert to a high-energy electron.
In the limit where the photon is reflected directly back on itself, the resultant electron has momentum greater than the incoming photon by around $M_e/2$.
As such it is Compton scattering which is more of a concern for COMET.

Finally, given the maximum photon energy is below the momentum threshold of \VarMomThreshold, the electron's energy needs to be mis-reconstructed by around 2.4~MeV.
If the final resolution function were a pure Gaussian with a 200~keV/c width (at signal energies), then mis-reconstruction by such an amount would be an 11~$\sigma$ event, \ie $P(p_\textrm{recon}-p_\textrm{true}>\SI{2.4}{MeV/c})=\num{1.9e-28}$.
However, the resolution function is not likely to be a pure Gaussian.
On-going studies for reconstruction with the  \phaseI StrECAL currently suggest that the tails of the resolution function are much larger than for a pure Gaussian.
Although the \phaseI Straw Tracker reconstruction is still being finalised, early studies suggest 
$P(p_\textrm{recon}-p_\textrm{true}>\SI{2.4}{MeV/c})=\VarResolutionProbFromRMCToSignal$~\cite{YFujiiStrECALRecon}.
Although this is considerably larger than the pure Gaussian, this is based on the most preliminary results of reconstruction algorithms.
If need be, cuts on fit quality can be improved over the current \phaseI values; the sensitivity estimate presented here should be robust against this since it includes the CDR estimate for the signal efficiency of such a cut.

\subsection{Calculation and Simulation of RMC}
Conversion of the RMC photons can take place in any of the material around the target, such as the beam blocker, solenoids, or cryostat.
To estimate the acceptance of the electrons produced from this conversion a simulation was performed where RMC photons were input at the stopping target, using the realistic muon stopping distribution obtained previously.
Geant4 implements the process of photo-conversion via both of the afore-mentioned methods.

To build a realistic end-point spectrum, the same recipe as used in the \phaseI TDR was applied~\cite{TDR2016}.
The spectrum shape near the end-point is modelled by the equation~\cite{CHRISTILLIN1980331}:
\begin{equation}
	\eqlabel{bg:rmc:spect}
	\Gamma(\textrm{RMC})\propto(1-2x+2x^2)(x)(1-x)^2,
\end{equation}
where $x=E_\gamma/\max(E_\gamma^\textrm{RMC})$.
The largest observed branching ratio for \ac{RMC} compared to ordinary muon capture---\num{1.83e-5}---is used to set the normalisation.
The relative rate of \ac{RMC} resulting in photons with energies above 90~MeV is, based on the spectrum in equation~\eq{bg:rmc:spect}, \num{7.7e-3} of the experimentally observed rate.
\FigRMCSimResults

Given the total number of muon stops during \phaseII, the probability of ordinary muon capture, and the relative rates for \ac{RMC}, one expects some \VarRMCTotalPhotonPhaseII  photons to be produced in the target with energies above 90~MeV.
Since this is an intrinsic background, the lifetime of this process will be the same as for signal events. 
Assuming, therefore, the same timing window acceptance, and applying the same reconstruction and TDAQ efficiencies as for signal, this number is reduced to \VarRMCTotalPhotonPhaseIIAccepted.

To check the probability that these photons produce a high-momentum electron in the detector, 
some \num{6e7} RMC events were generated in the stopping target, with initial photon energies greater than 90~MeV and distributed according to the spectrum in equation~\eq{bg:rmc:spect}.
\Fig{bg:rmc:simulation} shows the momentum and rate at which these were detected.
Based on this, the fraction of events reaching the detector with momentum larger than 98~MeV/c was \num{6.2e-7}, whilst above 100~MeV/c only \num{3.3e-8} electrons per \ac{RMC}$|_{p>90}$ event were detected.

The total number of electrons arriving with momentum larger than 98~MeV/c, therefore, is \VarRMCTotalElectronPhaseIIAccepted.
Since the resolution function is somewhat uncertain, this number can be interpreted as setting the limit for the high-momentum mis-reconstruction tail.
To have fewer than 0.01 background events produced by \ac{RMC}, the tails of the resolution distribution should contain fewer than $0.01/\VarRMCTotalElectronPhaseIIAccepted = \scinot{0.01/\NumRMCTotalElectronPhaseIIAccepted}$ events.

Using the estimate of the resolution function's tail from \phaseI, mentioned above, we predict therefore that \phaseII will produce \VarRMCPerMuStop per muon stop or \VarRMCTotal during the entire \phaseII run.

%As a result, if the current \phaseI resolution were applied, we would expect around 8~background events during the entire run.
%However, since the resolution function at this stage is so poorly known, it seems too premature to define a value.
%Instead, the predicted RMC rate provides a constraint on the high-energy tail of the resolution function: to do better than 0.1 background events during \phaseII from RMC, fewer than 1 in \num{2e4} high-energy electrons can be reconstructed with a momentum more than 2.4~MeV/c larger than the true value.
%On the assumption that such a resolution is achievable, we predict here that the number of backgrounds events due to RMC is less than \VarRMCPerMuStop per muon stop, or \VarRMCTotal during the entire \phaseII run.

\subsection{Aluminium-26 and \acs{RMC}}
Based on the above energy calculation, the end-point for \ac{RMC} against ${}^{26}$Al (to ${}^{26}$Mg and with no other particles emitted) would be 108.5~MeV.
Clearly such photons would be extremely dangerous to COMET if they are produced.

Aluminium-26 comes in two isomers, one with a half-life of around 6 seconds~\cite{PhysRevLett.106.032501}, the other lasting around 700 thousand years~\cite{AUDI20033}.
Since Al-26 is unstable its abundance in natural aluminium is low.
However, it can be produced by various methods, such as proton and deuteron bombardment of magnesium and sodium, or photo-neutron emission of the aluminium-27 isotope~\cite{THOMPSON1965486}.
With such production mechanisms, it is likely that aluminium-26 will be produced in the COMET stopping target, via the interaction of daughter nuclei of muon capture (which are typically magnesium and sodium)
with protons, deuteron, gammas, and neutrons coming from either the beam or as products of muon capture.
The exact rate of Al-26 production, however, is a complicated value to estimate and one that unfortunately cannot be estimated here.

One can, however, set an acceptable rate of production if the induced background event rate is to be kept at the level of 0.1 events.
Based on the previous simulation of Al-27 \ac{RMC}, the probability of an electron being detected within 5~MeV of the \ac{RMC} end-point is \num{5.6e-18} per nuclear muon capture.
With \VarTotalMuStops muon stops during \phaseII, and assuming that the branching ratios for both ordinary and radiative muon capture are the same for Al-26 and Al-27 (61\%), then the
concentration of Al-26 in the stopping target must be less than 1\% (by number density) on average during the entire \phaseII run.
The branching ratio for radiative muon capture is, in reality, likely to be slightly more than in Al-27, based on the fact that the neutron excess of a given isotope ($N=A/2-Z$) seems a better indicator of the branching ratio than atomic number, and that these two numbers appear anti-correlated~\cite{RevModPhys.76.31} ( the greater the neutron excess, the lower the RMC branching ratio).

Clearly though, there is more work to be done on this, including a better understanding of \ac{RMC} events coming from Al-26 and of the rate of Al-26 production from muon beams.
Data from the \alcap experiment might be able to help with this, however, since the production rate could depend a lot on the exact beam conditions, measuring this on site is important.
This should therefore be one of the measurements prioritised at \phaseI.

Finally, if this does produce a sizeable background contribution, one can imagine several techniques to mitigate or reduce the challenge this poses.
Provided one can measure the concentration of Al-26 in the stopping target at the end of the run, and ideally at various stages whilst running, then the number of backgrounds can be predicted and possibly subtracted.
Additionally, it could be possible to remove the stopping target and replace it with a fresh one such that the Al-26 concentration never rises beyond an unsafe level.

%\CHECK{Mention that this could be a useful source of calibration}
\section{Radiative Pion Capture (\acs{RPC})}
\sectlabel{bg:rpc}
When low-energy negative pions are stopped in material they behave similarly to negative muons and form pionic atoms.
The probability that the pion is then captured by the nucleus rather than decays in orbit is, however, considerably larger than for a muon.
Furthermore, given the extra 30~MeV/c$^2$ of the pion mass and the lack of an outgoing neutrino, the end-point for \acf{RPC} by the nucleus is well above the \mueconv signal energy.
As for photons of \ac{RMC}, the \ac{RPC} photons can then be converted via Compton scattering or pair production to signal-like electrons.
Pion capture could, therefore, be a dangerous source of backgrounds and was one of the dominant sources of background events at \sindrumII, along with cosmic sources~\cite{sindrum2006}.

Since the pion interacts via the strong force, negative pions capture almost immediately in the nucleus, on the order of picoseconds~\cite{Engelhardt:1975ct}.
The timing of backgrounds caused by pion capture are therefore determined predominantly by the time of pion production, and the delay due to transportation of the pion and signal-like electron.
If a background arises from pion capture and the pion was produced in the main muon pulse, then the pion or resultant background electron must have been significantly delayed.
Delayed \ac{RPC} backgrounds are therefore suppressed by the time-gated detector window.
If however a background arises because the pion was produced outside of the main proton pulse, due to late-arriving protons (or from antiprotons, but we will treat these separately below), then this background is considered prompt.
These prompt \ac{RPC} background events are, therefore, suppressed by the extinction factor.

The background rate per \ac{POT} for prompt and delayed pions is therefore:
\begin{align}
	R_\textrm{delayed}=&R_{\pi/p}\mathcal{B}_\textrm{RPC}f_{e,\textrm{RPC}}A_\textrm{time}, \\
	R_\textrm{prompt}=&R_{\pi/p}\mathcal{B}_\textrm{RPC}f_{e,\textrm{RPC}}\epsilon_\textrm{extinction},
\end{align}
where $R_{\pi/p}$ is the pion stopping rate per \ac{POT} and $\mathcal{B}_\textrm{RPC}$ is the branching ratio of \ac{RPC} for stopped pions.
$\epsilon_\textrm{extinction}$ is the extinction factor, whereas $A_\textrm{time}$ is the acceptance of the time-gated detector window to electrons coming from delayed RPC.
$f_{e,\textrm{RPC}}$ is the probability that an RPC photon converts to an electron which reaches the detector with signal-like momentum.
To a reasonable approximation, this can be factorised as:
\begin{equation}
f_{e,\textrm{RPC}}=f_{\gamma\rightarrow{}e^-}A_\textrm{geom}A_\textrm{mom}
\end{equation}
where $f_{\gamma\rightarrow{}e^-}$ is the conversion rate of RPC photons to an electron, $A_\textrm{geom}$ and $A_\textrm{mom}$
are, respectively, the geometric acceptance and momentum cut efficiency for such electrons.
However, such a factorisation misses out various correlations, such as where in the experiment the conversion takes place and at what momentum the secondary electron is produced, so that only a single value for $f_{e,\textrm{RPC}}$ will be reported here.

\FigRPCData
\subsection{Photons from Radiative Pion Capture (\acs{RPC})}
%In much the same way as \ac{RPC}, when a pion is captured in a nucleus there is a chance that it will produce a prompt, hard photon.
There is a range of experimental and theoretical data on \ac{RPC}. 
\Fig{bg:rpc:data} shows what is perhaps the most useful data currently available: the observed spectrum of photons coming from RPC for magnesium and calcium.
Magnesium being adjacent to aluminium on the periodic table, this spectrum is a reasonable proxy for the spectrum of \ac{RPC} on aluminium.
The relative rate for \ac{RPC} compared to ordinary pion capture is discussed in Amaro \etal~\cite{Amaro:1997ed}.
For experimental and theoretical studies of the three isotopes summarised in that paper---carbon, oxygen, and calcium---the measured and predicted branching ratios are all within 1.19\% and 2.27\%.
To be conservative, we take here the branching ratio for RPC on aluminium to be the largest of these at 2.27\%.

\subsection{Pion Stopping Rate}
\FigPionStopDist
\FigPiVsMuMomenta
To simulate the pion stopping rate, the pions from the main production simulation were resampled multiple times to build up a large number of pion stops.
\Fig{bg:piStop:dist} shows the distribution of pions stopping in the target, in one-dimensional projections to the ICEDUST global coordinate system.
In that coordinate system, pions arrive from large values of Z at the target, so the pion beam in \fig{bg:piStop:dist:z} is going from right to left.
By comparison with the plots in \fig{sense:stops}, pions tend to stop further downstream in the target.
This is readily understood by the fact that pions reaching the stopping target tend to have much higher momentum than muons reaching the target, as shown in \fig{bg:piVsMu:momenta}.

In total, the number of pions stopping in the target per \ac{POT} is $R_{\pi/p}=$\VarPiStopsPerPOT.

\subsection{Simulating \acs{RPC}}
\FigRPCSimulatedSpectrum
\FigRPCSimResults
Using the realistic pion stopping distribution shown in \fig{bg:piStop:dist}, \ac{RPC} photons were generated in the target and Geant4 used to convert and track electrons resulting from this process.
To model the distribution of photon energies from \ac{RPC}, the experimentally obtained spectrum from magnesium of ref.~\cite{Bistirlich:1972jy} was used.
To build the model, the raw spectrum was first digitised and then smoothed, using `TGraphSmooth::SuperSmooth()' from the  ROOT library~\cite{ROOT}.
These steps are shown in \fig{bg:rpc:spectrum}.

Based on a simulation of \num{4e6} RPC photons, the distribution of electrons and positrons reaching the detector was obtained.
The timing and momenta of such electrons is shown in \fig{bg:rpc:sim}, where it is clear that although many signal-like electrons are detected, they all arrive well before the gated-time threshold of 600~ns.

The probability of an \ac{RPC} photon producing an electron that reaches the detector with momentum between \VarMomThreshold and 105~MeV/c is: $f_{e,\textrm{RPC}}=$\VarDetectedEsPerRPC.
To estimate the rate of delayed \ac{RPC} backgrounds, we also need to know the value of $A_\textrm{time}$.
As can be seen in \fig{bg:rpc:sim:momVtime}, the timing for electrons originating from RPC photons and detected with momentum greater than 30~MeV/c is independent of the momentum.
By fitting the tail of this distribution with a single exponential, as shown in \fig{bg:rpc:sim:time}, the lifetime of the high-energy electrons is found to be \VarRPCLifetime, such that with the timing window between 600 and 1200~ns we find the acceptance of the timing window to be: $A_\textrm{time}=\VarRPCTimingEfficiency$.

\Tab{bg:rpc:estimates} summarizes these numbers, from which we find that the rate of backgrounds for delayed \ac{RPC} is \VarRPCDelayedPerPOT per POT, whilst prompt RPC occurs at \VarRPCPromptPerPOT per POT.
\begin{table}[tb]
\centering
        \begin{tabular}{llm{0.5\textwidth}}
	\hline
        Parameter & \multicolumn{1}{l}{Value} & Description \\
	\hline
        $R_{\pi/p}$                    & \VarPiStopsPerPOT & Pion stopping rate per \ac{POT}  \\ 
        $\mathcal{B}_\textrm{RPC}$     & \num{2.27e-2} & Branching ratio of \ac{RPC} \\ 
	$f_{e,\textrm{RPC}}$           & \VarDetectedEsPerRPC & Probability of an RPC photon producing signal-like electrons in the detector \\ 
	$A_\textrm{time}$              & \VarRPCTimingEfficiency & Acceptance of timing window to secondary electrons from RPC \\ 
        $\epsilon_\textrm{extinction}$ & \VarExtinctionFactor[2] &  Extinction factor\\ 
	\hline
\end{tabular}
\caption{\tablabel{bg:rpc:estimates}
Parameters and their values in the determination of the \ac{RPC} background rate.
}
\end{table}

\section{Antiprotons in the Beam}
\sectlabel{bg:antiprotons}
When the primary 8~GeV proton beam interacts with the production target, antiprotons might be produced as a proton--antiproton pair:
\begin{equation}
p + N(A,Z) \rightarrow p + N^*(A,Z) + p+\bar{p},
\end{equation}
which, given their negative charge, can remain on-axis along the muon beamline if they have momentum around 40~MeV/c.
Because their mass is relatively large, at a fixed value of momentum antiprotons travel much more slowly than other particles in the muon beam, causing the beam's time structure to be smeared for antiprotons and their secondaries.
The pulsed beam and time-gated detector window are, therefore, not effective at suppressing backgrounds created by antiprotons in the beam.

%Secondaries of antiproton interactions with matter include X-rays and pions, though pion production dominates~\cite{BobBernsteinExperimentersGuide,Mishustin:2004xa}.
%
The interaction of antiprotons with matter has a strong analogy with muons and negative pions, particularly at low energies, when the antiprotons stop.
%There is a strong analogy between antiproton stopping in matter and the phenomona around negative muon and pion stopping.
In matter, antiprotons with energies of a few tens of keV---similar to that of atomic electrons---can become bound in the Coulomb potential of the atom's nucleus.
X-rays emitted in the ensuing electromagnetic cascade are typically not more than a 100~keV~\cite{Aramaki201352}.
Unlike for muons, the antiproton will only rarely reach the atomic ground state, with interactions with the nucleus becoming dominant before this is possible.
Such interactions can take the form of an immediate annihilation or the formation of a composite nucleus where the antiproton becomes bound within the nuclear potential~\cite{Wong:1984fy,Mishustin:2004xa}.
The binding energy of some of these nuclear levels can reach up to hundreds of MeV, and so in the transition, pions and other hadrons can be readily produced.
Eventually the antiproton annihilates in the nucleus, producing large multiplicities of pions and other mesons.
Based on the data in ~\cite{Mishustin:2004xa}, about 1.56 negative pions should be produced for every stopped antiproton.

As such, although antiprotons themselves are not an immediate source of high-energy photons or electrons, they are an additional source of pions, which can produce backgrounds via \acf{RPC}.

\subsection{Antiproton Production Rate and Spectrum}
The literature on antiproton production with 8~GeV protons on a tungsten target and at large angles is somewhat lacking.
%There is really very little literature on the production of antiprotons from a tungsten target for a range of angles.
Accordingly most hadron models are particularly under-constrained when it comes to antiproton production.
The QGSP_BERT_HP model used as the basis for SimG4 is, in fact, completely unable to produce antiprotons.

In the COMET TDR~\cite{TDR2016}, the yield of antiprotons per \ac{POT} is given as \num{4e-5}, based on the findings of a literature review and simulations undertaken for the MECO experiment~\cite{Meco024}.
\Fig{bg:antiprotons:meco24} shows the production cross section for antiprotons as a function of incident proton momentum, based on that study.
%The dependence of the antiproton production cross section on incident proton momentum based on that study is shown in \fig{bg:antiprotons:meco24}.

Simulations of antiprotons for \phaseI have used this information and shown that absorber foils would be needed along the beamline.  
Since the \phaseII geometry in ICEDUST re-uses most of the implementation for the Production Target Capture and Torus1 sections, the upstream absorbers are also contained in the geometry used to study antiprotons here,
although the absorber near 90\degree is not.

\FigAntiprotonMeco
\FigAntiprotonData
Whilst tungsten targets have not been studied at the relevant angles and proton energies, a set of papers \cite{Boyarinov:1994tp,Kiselev:2012sj} do exist covering antiproton production up to 2 radians for tantalum (which is adjacent to tungsten on the periodic table), copper, aluminium, and beryllium targets and using protons with 10.14~GeV kinetic energy.
Between them, these two papers provide the invariant triple-differential cross section as a function of antiproton momentum for production angles of 10, 59, 97 and 119\degree, defined as:
\begin{equation}
\eqlabel{bg:antiprotons:lit}
	F(p,\theta,\phi)=E \frac{d^3\sigma}{dp^3}=E\frac{d^3\sigma(p,\theta,\phi)}{p^2dpd\Omega}
\end{equation}
where $E$ and $p$ are the antiproton energy and momentum.  The earlier Boyarinov paper actually reports $f=F/A$, where $A$ is the relative atomic mass of the nucleus.
To convert the differential invariant cross sections given in the literature (equation~\eq{bg:antiprotons:lit}) into a differential production rate per \ac{POT}, $d^3R(\theta)/dpd\Omega$, the following formula is used:
\begin{align}
	\eqlabel{bg:antiprotons:rate}
	\frac{d^3R(p,\theta,\phi)}{dp^3}&=\frac{F(p,\theta,\phi)}{E}\frac{\rho N_A l}{m_N}
\end{align}
where $\rho$ is the mass density of the target, 19.25~g/cm$^3$ for Tungsten, $m_N$ the atomic mass, 183.86~g/mol for tungsten, $l$ the length of the target, and $N_A$ is Avogadro's number.
\Fig{bg:antiprotons:data} shows the data from these papers, after scaling the cross sections for each material to the number of nucleons in tungsten.
The data has also been converted to a differential rate of production, given the \phaseII production target dimensions, using equation~\eq{bg:antiprotons:rate}.
From this data, it is clear that as one moves to larger angles, the spectrum becomes considerably softer whilst the overall rate falls quickly, in much the same way as for pion production.

\subsection{Modelling Antiproton Production}
To develop the data in the Boyarinov~\etal papers into a model for antiproton production, three dependencies must be considered:
the incident proton's momentum, the outgoing antiproton's momentum, and the outgoing antiproton's direction with respect to the incident proton.

To account for these dependencies, the cross section was factorised according to the equation:
\begin{align}
\eqlabel{bg:antiprotons:factorisation}
	\frac{d^3R(p,\theta,\phi)}{dp^3}=&\xi{}P(p,\theta_\mathrm{min})\Phi(\theta,\phi),
\end{align}
where $\xi$ is a scale factor to correct for the different incoming proton energy, $P(p,\theta_\textrm{min})$ models the momentum distribution at a given angle and is normalised to unity (\ie $\int{}P(p,\theta_\textrm{min})dp=1$), and $\Phi(\theta,\phi)$ encapsulates the dependence on the antiproton's direction.  Each of these factors are discussed in-depth below.

\subsubsection*{Dependence on the Incident Proton's Momentum: $\xi$}
The factor $\xi$ is included in the antiproton production model to account for the difference in proton energy between the available data and the COMET proton beam.
It is defined as the ratio of the total production cross sections at the different energies:
\begin{equation}
\xi=\frac{\sigma_\mathrm{tot}(E_p=8 GeV)}{\sigma_\mathrm{tot}(E_p=10.14 GeV)}
\end{equation}

The data in the Boyarinov papers was taken using protons with kinetic energies of 10.14~GeV.
The threshold for antiproton production against a single, free nucleon is 5.6~GeV; the cross-section increases rapidly above this energy, as indicated in \fig{bg:antiprotons:meco24}.
Because of this, COMET has chosen to use an 8~GeV (kinetic energy) beam, and accordingly it is important to scale the antiproton cross sections from the Boyarinov~\etal papers to account for this.

\Fig{bg:antiprotons:meco24} suggests that the cross section at 8~GeV is about one order of magnitude smaller than at 10.14~GeV, although the data and models included in that plot are for a lead target as opposed to tungsten.

In a paper by Danielewicz~\cite{PhysRevC.42.1564}, a theoretical calculation is performed to estimate the rate of antiproton production.
Equation (4.3) of that paper gives the antiproton production cross section for nucleon-proton interactions as:
\begin{equation}
\sigma_{NN}^{\bar{p}}=C(\sqrt{s}-4m)^{7/2}
\end{equation}
where $\sqrt{s}$ is the Mandelstam variable for the total invariant incoming energy, and $m$ the mass of a nucleon.
$C$ depends on properties of the nucleus and is fixed in that paper using experimental data.
Based on this equation, the cross section for antiproton production at 8~GeV should be a factor of 0.12 times that at 10.14~GeV.
Since this agrees well with \Fig{bg:antiprotons:meco24}, we take $\xi=0.12$.

\subsubsection*{Dependence on the Outgoing Antiproton's Momentum: $P(p,\theta_\textrm{min})$}
The momentum spectrum of antiprotons at a given angle is captured in the factor $P(p,\theta_\textrm{min})$.
We base the shape of this spectrum on the data from the Boyarinov papers.

\FigAntiprotonEndpoint
In order to complete the data and cover the full range of energies at which antiprotons might be produced, we add two additional points to the experimental data for the maximum and minimum momentum.
Including the minimum momentum only adds the constraint that at zero momentum the cross section also be zero.
On the other hand, to calculate the maximum momentum, the kinematic end-point was found by considering the entire nucleus and the two outgoing protons to recoil coherently against the antiproton.
The value of the end-point kinetic energy and longitudinal and transverse momenta for both tungsten and carbon are shown in \fig{bg:antiprotons:end-point} using the formulae derived in appendix~\sect{appendix:antiprotonEndpoint}.
This end-point will be a highly conservative estimate, since in reality not all the nucleus will recoil coherently;  the de Broglie wavelength for a proton with 8~GeV kinetic energy is about 0.15~fm, compared to the 7~fm or so of a tungsten nucleus%
\footnote{Antiproton production from protons in this energy regime was historically referred to as `sub-threshold production', since it is close to the threshold for single nucleon interactions, around 5.6~GeV.
The fact antiprotons are observed below this energy is therefore proof that it is not a single, stationary nucleon that interacts with the proton; the nuclear environment as a whole must be considered.
Either the nucleons are, themselves, in motion or clusters of bound nucleons are involved.
  The older literature on this topic in fact refers to a parameter called the `cumulative number' which was related to the number of nucleons that would need to be involved to produce secondary hadrons (\eg antiprotons or pions) with such outgoing energies, given the incoming proton energy.  
These sub-threshold secondary particles were themselves sometimes called cumulative particles.}.
Additionally, achieving this end-point configuration would be highly phase-space suppressed.

\FigAntiprotonFits
With the addition of these two end-points, to interpolate and extrapolate the data, a polynomial of order 4 to 6 (depending on the number of available data points) was used to fit from zero up to the last experimental data-point.
For the high-momentum tails of each spectrum, three fits were tried: a straight line fit between the last experimental data-point and the kinematic end-point described above; an exponential fitted to the kinematic end-point and the last two experimental data points; and an average of both of these fits.
\Fig{bg:antiprotons:fits} shows the results of this fitting procedure, where it can be seen that a low-momentum peak is visible and a high-momentum tail well described.
Whilst these spectra are likely a poor representation of the true production spectrum, they serve as useful upper bounds which can be used as inputs for the antiproton background rate estimation.

The momentum distribution will, of course, depend on the angle of production relative to the proton beam.
With only four production angles available in the experimental data, however, for a given angle, $\theta$, we assign the momentum distribution based on the data at the next smallest production angle, $\theta_\textrm{min}$.
Since the spectrum becomes softer at larger values of $\theta$, this will tend to further overestimate the high-energy tail of the momentum distribution.
In the case of angles smaller than 10\degree, given the lack of other data, we fix the  shape of the momentum distribution to be the same as that at 10\degree.

The function $P(p,\theta_\textrm{min})$ is the piecewise function that is fit to the data and kinematic end-points and then normalised so that the integral is one.
The overall normalisation is instead absorbed into the angular dependence factor, $\Phi(\theta,\phi)$.

\subsubsection*{Dependence on the Outgoing Antiproton's Direction: $\Phi(\theta,\phi)$}
\FigAntiprotonAngularDependence
The function $\Phi(\theta,\phi)$ accounts for the variation of the cross section as the angle between the incoming proton and outgoing antiproton changes.
By symmetry, it can be seen that there should be no dependence on $\phi$ (both the proton beam and target are assumed to be unpolarised).

To estimate the cross section at a given angle, the fitted distributions described in the last section were integrated.
The results of these integrations is shown in \fig{bg:antiprotons:angular}.

%When these distributions are used to generate particles they are essentially normalised so their integral is unity.
%For a given value of $\theta$ the momentum distribution $dP/dp$ is chosen to match the first spectrum measured at a larger value of $\theta$.
%This keeps the model conservative since the true momentum spectrum should soften as one moves to higher angles which this will not do.
%$d\Phi/d\theta$ is then used to vary the overall normalisation as a function of theta.
Boyarinov \etal state~\cite{Boyarinov:1994tp} that the angular dependence of antiproton production should take the form:
\begin{equation}
%\frac{d\Phi(\theta)}{d\theta}
\Phi(\theta)=\alpha e^{\beta\cos\theta}
\end{equation}
where $\alpha$ and $\beta$ are constants. 

It is for this reason that the axes of \fig{bg:antiprotons:angular} are $\cos\theta$ and $\ln(R)$: if such a function were valid, the points would form a straight line.
In fact it seems the variation in these parameters are better described by a quadratic function, although the fitted quadratic functions all reach a minimum for $\cos\theta\ne-1$, which seems unphysical.

However, to keep things conservative, rather than use  a linear or quadratic fit to the integrated data, we use the linear interpolation between each adjacent datum to provide the function, $\Phi(\theta,\phi)$.
For the same reason, we use the integrated cross sections obtained for the fits that use a linear function for the high-momentum tail.

\TabAntiprotonRegions
\subsection{Simulating Antiproton Backgrounds}
\FigAntiprotonSimFluxes
To study backgrounds from antiprotons, four separate simulations were performed, one for each of the angular regions described in \tab{bg:antiprotons:regions}, with \num{8e7} antiprotons fed in to each simulation.
Within each angular region, antiprotons were generated uniformly across the production target, with momenta distributed according to $P(p,\theta_\textrm{min})$.
Given how steeply the angular distribution tails off, in order to check the acceptance of antiprotons produced at large angles, events were generated isotropically within the angular region, and then re-weighted during analysis based on $\xi\Phi(\theta,\phi)$, ie. the probability that an antiproton is produced in this direction.

The beamline's acceptance to antiprotons and the secondary pions that they produce are shown in \fig{bg:antiprotons:sim:fluxes}.
The events in this plot are unweighted by the antiproton angle.
From these plots, it is clear that the material at the entrance to the Torus1 solenoid removes a large fraction of the antiprotons---about 99\% of the antiprotons produced between 97 and 119\degree.
It can also be seen that the acceptance in the bent solenoids for antiprotons is about two orders of magnitude greater for antiprotons produced at angles greater than 119\degree, compared to those produced between 97 and 119\degree.

For pion production and acceptance, however, the unweighted rate for the four angular regions is very similar, particularly towards the end of Torus2.
This, and the fact that the transmission probability has a negative gradient for each angular region along the whole beamline, implies that the main source for pions is the interaction of antiprotons in the primary target.
Pions produced by antiprotons hitting collimators or the beam-pipe will quickly be removed since they will tend to stay close to this material.
If pions were produced at the foil at the entrance to the Torus1 then they would be in the centre of the beam-pipe and have a greater chance of being transported, although this does not seem to be a major source.

\FigAntiprotonSimHeightsTwoDPbar
\FigAntiprotonSimHeightsTwoDPiMin
\TabAntiprotonResultsAntip
\TabAntiprotonResultsPiSecond
\afterpage{\clearpage}

\Fig{bg:antiprotons:sim:2D-antip} shows the trajectories of antiprotons along the beamline, where each trajectory has been weighted according to the antiproton's production direction with respect to the proton beam axis.
Similarly, \fig{bg:antiprotons:sim:2D-pi} shows the same information but for pions instead.
The colour scale and range is the same for each of the four plots in these figures.
One can see, then, that although the acceptance per antiproton is considerably larger for larger angle production (119--180\degree), the production rate is so small that the final probability of observing an antiproton is quite low.
As such, it is the antiprotons produced in the forwards direction and scattered backwards by the production target that will dominate the antiproton background rate.

\FigAntiprotonSimTime
\Tab{bg:antiprotons:sim:antip} and \tab{bg:antiprotons:sim:pi} show the number of antiprotons and pions that enter the Torus1 (`Torus1'), reached the final collimator (`Torus2'), and arrived at the target (`Target').
Also shown is the sum over all events with each event weighted by the initial antiproton direction and proton energy scale correction ($\xi\cdot\Phi(\theta,\phi)$), and normalised to the number of antiprotons introduced into the simulation.
No antiprotons are seen stopping in the target for the angular region from 0 to 59\degree.
To estimate the stopping rate of antiprotons produced in this direction, 
the weighted sum for antiprotons just before the collimator was multiplied by the median acceptance of the collimator, $P(\textrm{Target}|\textrm{Torus2})$, which occurs for the region from 59 to 97\degree.

The timing distributions for antiprotons and pions are shown in \fig{bg:antiprotons:sim:time}, from where it is clear that timing information from the pulsed proton beam is not meaningful for antiprotons.
However, since the majority of secondary pions that are successfully transported to the target originate in the production target, the pulsed beam timing is a strong way to suppress the backgrounds these can produce via \ac{RPC}.
In fact, given that these pions originate in the production target, the background rate due to RPC of pions directly produced by the proton beam can be reused, including the timing cut.
For antiprotons, the rate of RPC backgrounds per pion stop is also used, however the mean number of negative pions produced per antiproton stop is included, as well as the overall efficiency of the timing window, assuming a flat timing distribution.
%All of these factors are summarised in \tab{bg:antiprotons:factors}.
For both pions and antiprotons, the efficiency of the TDAQ and reconstruction are also included, with the same values as for signal electrons.
Based on these factors, the numbers in \tab{bg:antiprotons:sim:antip} and \tab{bg:antiprotons:sim:pi} can be converted to background estimates, which are shown in \tab{bg:antiprotons:estimates}.

\TabAntiprotonEstimates
The total predicted antiproton rate per \ac{POT} is therefore \VarBgAntiprotonsPerPOT, which is completely dominated by antiprotons produced in the forward direction but scatter within the production target and are successfully transported to the stopping target.
This number is considerably larger than would be desirable for the COMET experiment.
It is important, therefore, to remember that this is an especially conservative estimate:
\begin{itemize}
\item The kinematic end-point is a large overestimate for the true end-point of the spectrum since the incoming proton will never interact with the entire nucleus.  
\item The straight-line-fit to the tail, that was used to generate antiprotons in the simulation will produce higher momentum antiprotons than are likely to occur, which changes the penetration depth in material.
      In addition, the straight-line-fit increases the integrated yield of antiprotons significantly, which impacts on the event-by-event weight.
      For example, for the region from  59--97\degree, there are up to two orders of magnitude difference on the predicted antiproton yield depending on the high-energy description.
%      \CHECKmargin{Add to the first table the integrals with the other tail descriptions.}
\item Antiproton events were generated with a uniform distribution in the production target. Given that the proton beam profile will be close to a double Gaussian, with most of the probability concentrated in the centre of the target, there will typically be more material for the antiprotons to pass through as they leave the target.  This will increase the rate of absorption and thus reduce the antiproton rate further.
%\item The data this model is based on uses 10~GeV initial protons. The antiproton yield is expected to grow rapidly at these sorts of energies.
%      According to the model in~\cite{Meco024}, the cross section at 8~GeV is at least an order of magnitude lower than for incident protons with 10~GeV.
\item The final value assumes that all the negative pions produced from stopping antiprotons are captured by nuclei in the target.  In reality, many of these will escape from the target and decay in flight or be stopped and captured elsewhere.  The pion multiplicity factor used here is likely an overestimate, therefore.
\end{itemize}
At the same time, the fact that these results are somewhat statistically limited, given the small number of antiproton stops seen from the forward going antiprotons, one can expect a sizeable uncertainty on this value.
In addition, some of the other factors that have been included, such as from the RPC background estimation, also increase the uncertainty since there might be other correlations that such a factorisation cannot include.

On the other hand, the fact that it is scattering from forward produced antiprotons has some advantage, since there is experimental data in the forward region---admittedly with materials other than tungsten or with proton energies other than 8~GeV---that could be used to develop further understanding of this background.

\subsection{Reducing Antiproton Backgrounds Further}
%\begin{easylist}
%# Ways to improve this / further optimisations:
%## Antiprotons close to the beam axis after the production target: on-axis blocker (activation?)
%## Absorber material in lower half of TS3
%# Thicker absorber at the muon beamline entrance
%# Need to check the muon beam intensity
%\end{easylist}
If the background rate due to antiprotons presented here proves correct, then changes to the current beamline design should be implemented to reduce it.
%Given that the antiproton with the current beamline design appears larger than desirable, it would be necessary to take steps to reduce it.
There are several options that can be pursued, but each must be checked for its impact on the muon beam intensity.

Firstly, one can simply increase the thickness of the foil at the entrance to the Torus1.
In a similar way, one can introduce additional foils along the beamline, such as at the TS3 solenoid in between the Torus1 and Torus2.
The downside of both of these methods is that they will be in the path of the muon beam, and since the $dE/dx$ curve for negative muons below 100~MeV has a negative gradient, this material will likely increase the mean muon momentum (but, of course, decrease the overall intensity).

If material is added, then it would be beneficial if this material were low-Z.  This would maximize the ratio of the rate of antiproton loss to the muon loss rate.
In fact, a bug in the geometry during initial simulations of antiprotons left air at the exit to the production section.  
The impact on the muon and pion momentum and acceptance was found to be negligible, but the antiproton acceptance into the Torus1 was zero at the limit of the sample size of the simulations.

It may be possible to identify a region in the beamline close to the production target where a separation occurs between antiprotons and low-energy muons and pions.
This would be highly favourable as it would reduce the impact on signal sensitivity, although in the studies performed here such a region has not been apparent.
Additionally, material close to the production target would need to be considered from an activation and safety perspective.

Finally, as well as adding material along the beamline, one could consider a slight reduction in the proton beam energy.
Even a reduction by 200 MeV would reduce the total antiproton production cross-section by between 40 and 50\% based on the methods described above to determine the $\xi$ weighting factor.
Reducing the beam power further, by around 500 MeV would reduce the cross section by 75 to 90\% of the current estimate, \ie reduce the production rate by about one order of magnitude lower.

\section{Direct Beam-Related Backgrounds}
\sectlabel{bg:beam}
%\begin{easylist}
%	# Number of POT simulated
%	# Number of observed high-momentum electrons or muons
%	# Lifetime of muons at the target (need to smear this a bit ?): Fit the projection of high-momentum particles to the time with an exponential and extrapolate
%	# Survival probability per high-momentum particle before the target to reach the straw tracker
%\end{easylist}
Although neither a muon or pion in their rest frame can decay to electrons greater than 55~MeV, if these particles are boosted sufficiently the resulting electron can be produced with signal-like momentum.
For muons at 78~MeV/c, Michel electrons can occur at 105~MeV/c if produced in the direction of the muon's momentum (and neglecting muon polarisation).
For pions, the threshold for signal-like electrons to be produced during decay is 58~MeV/c, although the branching fraction for this process is helicity suppressed to around \num{1e-4}.

The bent solenoids and the pulsed beam help to suppress these sorts of backgrounds, whilst the beam blocker, downstream of the target disks, reduces the geometric acceptance.
Following the results of chapter~\sect{phaseII-optimisation}, the inner radius of the collimator at the exit of Torus2 is set to 12~cm, whilst the beam blocker has a radius of 25~cm. 
As the beam passes along the stopping target region, the beam envelope grows due to the reduction in the solenoidal field strength.
The radius of the beam blocker, however, is large enough that even with this growth there should be no direct line of sight between the muon beam and the spectrometer.
As such, the only way for beam particles to reach the detector are by a hard scatter off the target or beam blocker, or to be produced via decay of another particle close to the target itself.

To check the probability that signal-like electrons are produced in the beam and arrive at the detector in the time window,
the output of the large production target simulation was resampled five times, so that around \num{1.1e9}~\ac{POT} events were studied, equivalent to about 8~\phaseII bunches.
\Fig{bg:beam:MomVsTime:strawTrk} shows the momentum and timing of electrons at the straw tracker, from which it is clear that no background events were observed within the gated-time detector window.
One electron is observed at 98~MeV/c, but this occurs around 100~ns after the proton bunch.
This immediately sets a limit on the background rate---less than 1 in \num{1.4e9} per POT.
\FigBgBeamMomVsTime

To be able to understand this background better, it is necessary to apply some extrapolation.
Indeed, it will never be feasible to directly simulate the $10^{21}$ or so POT that would be necessary to deduce the background rate without some extrapolation.
To make this extrapolation, we look at the same distribution immediately after the beam blocker.  
At this point there are far more electrons so it is easier to fit and model the distributions.
\Fig{bg:beam:MomVsTime:stopTgt} shows the momentum vs.\ time plot for electrons immediately after the beam blocker.

The rate of observing delayed, signal-like electrons due to high-energy beam particles is modelled by factorisation of the acceptances:
\begin{align}
R_\mathrm{beam}=&A_\textrm{geom}A_\textrm{mom}A_\textrm{time},\\
=&\left(\frac{N_\textrm{obs}}{N_\textrm{POT}}\right)\left(\frac{\int\limits_{104.2}^{105.5}f(p)dp}{\int\limits_{p_\textrm{low}}^{105.5}f(p)dp}\right)\left(\frac{\int\limits_{600}^{1200}g(t)dt}{\int\limits_{30}^{1200}g(t)dt}\right),
\eqlabel{bg:beam:factors}
\end{align}
where $A_\textrm{geom}$, $A_\textrm{mom}$, and $A_\textrm{time}$ are geometric acceptance and efficiencies of the momentum and timing thresholds, respectively.
$N_\textrm{obs}$ is the number of events actually seen at the straw tracker, whilst $N_\textrm{POT}$ is the number of simulated \acp{POT}.
In the case of prompt beam electrons, the above equations are the same, except that $A_\textrm{time}$ becomes the extinction factor multiplied by the fraction of time spent in the gated-time detection window.
\FigBgBeamExtrapolate

The functions $f(p)$ and $g(t)$ represent the momentum and time projections of electrons after the beam blocker. 
Their factorisation in this manner is valid due to the lack of strong correlation visible in \fig{bg:beam:MomVsTime:stopTgt}.
It is reasonable to assume that, close to the signal region, the momentum distribution observed immediately after the beam blocker will have the same shape as the momentum distribution at the straw target.
Since the acceptance function is essentially flat at the high-energy limit (see \sect{optim:stopTgtPos} and \sect{sense:momentum}), only the normalisation of the spectrum should change, and this we have encapsulated already in the geometric acceptance.

Finally, the timing distribution, $g(t)$, is also taken by fitting the observed electron distribution after the beam blocker.
Here, however, one needs to be careful about extrapolating the distribution from after the beam blocker to the straw tracker.
The extrapolation must account for two effects: an overall translation of the distribution to later times, and a broadening of the distribution (an increase in the fitted lifetime).
Both effects are determined by the time needed for particles at the beam blocker to propagate through the spectrometer.
With a radius of 2~m, the typical distance travelled along the Spectrometer will be at least about 6~m, and since the electrons are relativistic and travelling at close to the speed of light, transport through the spectrometer can be expected to add about 20~ns.
This assumes that the electron's pitch angle is equal to zero so that they follow the beam axis perfectly.
With a large pitch angle the distance travelled, and therefore transportation time, will be longer, although since larger pitch angles drift more in the bent solenoids, these particles will tend to be removed more effectively.
The effect of broadening on the distribution arises from lower energy electrons moving more slowly, however for the relativistic signal-energy electrons this effect will be negligible.

\Fig{bg:beam:acceptance} show the momentum and time functions based on the distribution of electrons after the stopping target.
The equations for the lines of best fit are also shown, and the region used to fit indicated by the solid black fitted line.
The low-momentum threshold for this study, $p_\textrm{low}$, is set to 65~MeV/c, since this removes electrons from \ac{DIO} of muons along the beamline, which have relatively long life-times but cannot be produced at signal energies.
Above this threshold, there are 65 events at the straw tracker, giving a geometric acceptance of \VarBeamBgGeometric.
For the efficiency of the momentum cut, using the fit to the projection shown in \fig{bg:beam:acceptance:momentum}, the ratio of events expected between 104.2 and 105.5~MeV/c compared to the integrated rate from 65 to 105.5~MeV/c is \VarBeamBgMomentum.
For the timing efficiency, the fit given in \fig{bg:beam:acceptance:time} is first shifted so that the mean of timing distribution at the beam blocker sits at the time of the latest arriving electron above 65~MeV/c, as indicated
by the vertical magenta lines. 
This amounts to a shift of 110.5~ns, which is about twice as large as if one were to use the mean of the straw tracker electrons.
Having applied this shift (which we implement as a shift in the start and stop time for the integration), we find the suppression of electrons due to the time-gated window gives a factor \VarBeamBgTiming.

\TabBgBeamFactors
All of these factors are summarised in \tab{bg:beam:factors}.
Finally, then, we can pull all these factors together using \eq{bg:beam:factors}, from which we find that the predicted prompt beam-related background rate per POT is~\VarBeamBgPromptPerPOT, whilst the delayed background rate due to beam particles is \VarBeamBgDelayedPerPOT.

Given the extrapolation that has been necessary, this result comes with a large degree of uncertainty.
However, in principle this includes all sources of electrons in the beam: high-energy muons, pions, and electrons coming directly from the production target with high energy.
This also includes electrons produced from neutron capture, provided the neutron itself originated downstream of the production target section, which were not included here in order to accelerate the simulation speed.
In addition these neutrons should be studied carefully using multiple hadron production models but, as discussed in \sect{bg:notCovered}, these neutrons are left to future studies.

\section{Cosmic Ray Background}
\sectlabel{bg:cosmics}
Cosmic rays at sea level consist primarily of positive and negative muons.
These are produced when high-energy particles originating in the sun or beyond the solar system strike the upper atmosphere and produce a particle shower.
About one muon per square centimetre per minute is expected at sea level, with the average muon energy around 4~GeV~\cite{PDG2014}.
At these energies, not only can such muons readily produce high-momentum electrons, but they are weakly ionising and therefore penetrate material relatively easily.

This makes cosmic muons a potential source of backgrounds, as the high-energy muons can pass through the experiment hall to the beamline whereupon they can produce signal-like electrons.
In addition, this background would lack any timing structure.
As a result, a  \ac{CRV} is included in both \phaseI and \phaseII, made with four layers of scintillator bars as the active material.
When a cosmic ray passes this, events in the detector can either be tagged, or any trigger on this event vetoed and not read out, such that signal-like electrons produced from cosmic rays should not be included in the analysis.

At the time of writing, the \ac{CRV} has been optimised and studied for \phaseI~\cite{TDR2016,CM18:CRV,CM20:CRV}.
These studies suggest that a miss-rate of fewer than 1 in $10^4$ events is achievable.
For \phaseII however, given the changes to the experiment geometry, the situation is a little different.
%If the muon penetrates into the detector and decays from there, or if a high-energy electron produced outside of the detector but close to it, is 
Principally, the stopping target in \phaseI is surrounded by the \ac{CDC}, such that cosmic muons must escape detection in both the \ac{CRV} and \ac{CDC}, and then decay close to the target.
In \phaseII, the stopping target is located well away from the detector, which means cosmic muons can only be vetoed by the \ac{CRV}.
Furthermore, the \ac{CRV} geometry in \phaseII should cover both the target and the intermediate beamline, in addition to the detector solenoid itself.

\phaseII is expected to run about \VarRunTime~s as suggested in section~\sect{sense}.
During COMET operation, a single accelerator cycle is expected to last 2.48~s, with about 0.7~s for extraction, giving a duty factor of 0.28.
In addition, with a time-gated window lasting 600~ns and bunches separated by 1.170~$\mu$s, the overall duty factor drops to about 0.14.
As such, during \phaseII running there are really only about \num{2.20e6}~s when a cosmic muon can produce backgrounds.
During this time then, \num{4.8e4} cosmic muons should pass through each horizontal square centimetre of the COMET hall.
If we assume the miss-rate for the \phaseII will be no worse than for \phaseI, \ie \num{1e-4}, then for each for each square centimetre of the experiment, more than 48 muons should be studied.

\FigBgCosmicPrimary
\subsection{Simulating Cosmic Muons}
To simulate the cosmic ray background, input muon events were provided by the ND280 experiment~\cite{George:2010aca}.
These contained positive and negative muons that had been produced in an air shower simulation using the CORSIKA simulation package~\cite{Corsika}, for the atmosphere above J-PARC.
\Fig{bg:cosmics:primary} shows the distributions of momentum and directions that these files contained.

Muons were generated in a horizontal plane directly above the experiment hall (and centred on the origin of the ICEDUST coordinate system, the production target).
A 50$\times$50~m${}^2$ square within in this plane was used to generate the events.
This immediately adds a limitation to this simulation that very horizontally-moving muons cannot be studied, but it is needed to increase the simulation efficiency.
%Given the angular distribution shown in \fig{bg:cosmics:primary} and the offset of the detector solenoid with respect to the production target, between XX and XX of all cosmic events can be simulated this way.
About \num{1.76e8} cosmic events were generated in this way; for the 2500~m${}^2$ region above the experiment where the simulation was started, more than \num{1.2e8} events should be studied, according to the above calculations.

Material above the beamline can degrade and stop cosmic muons so it is important this is modelled well.
The concrete and iron shielding for the entire experimental hall is implemented, as well as the shielding necessary for the \ac{CRV}.
However, at this stage the geometry in the simulation only has the CRV design intended for \phaseI around the detector solenoid (in its \phaseII position).
Since the CRV will need to be extended to include all the way up to the stopping target in \phaseII, there will likely be more shielding material inserted above the beamline.

\subsection{Results of the Simulated Cosmic Background Rate}
\FigBgCosmicBeam
\FigBgCosmicMomenta
The tracks of high-energy electrons that enter the detector region are shown in \fig{bg:cosmics:beam}.
\Fig{bg:cosmics:momenta} shows the momentum distribution of observed electrons.  
There are large fluctuations across the momentum distribution, but this is most likely due to statistical fluctuations rather than truly physical.
From 80 to 120 MeV/c, 140 electrons hit at least one layer in the straw tracker, although only 19 of these hit a second plane.
The number of background events due to cosmic rays can therefore be estimated as:
\begin{align}
	R_\textrm{cosmics}=R_\mu\frac{n_e^\textrm{obs}}{N_\textrm{sim}} \Delta{}p (1-\epsilon_\textrm{CRV}) A_\textrm{duty}A_\textrm{timing}A_\textrm{TDAQ,Recon},
\end{align}
where $R_\mu$ is the rate of cosmic muons expected through the experiment hall per second, $n_e^\textrm{obs}$ is the observed number of events per MeV/c, $N_\textrm{sim}$ is the total number of simulated cosmic ray muons, and $\Delta{}p$ is the width of momentum thresholds, from \VarMomThreshold to 105.5~MeV/c.
$\epsilon_\textrm{CRV}$ is the \ac{CRV} efficiency, such that $(1-\epsilon_\textrm{CRV})$ is the expected miss-rate. 
The factors $A_\textrm{duty}$, $A_\textrm{timing}$, and $A_\textrm{TDAQ,Recon}$ represent the accelerator duty factor, fraction of time spent during the time-gated detection window, and the combined TDAQ and reconstruction efficiency, respectively.  
The values for each of these parameters are given in \tab{bg:cosmics:params}.
From this we can estimate that there will be \VarCosmicRatePerSecond backgrounds per second due to cosmic rays.
\TabBgCosmicParameters

To reduce the cosmic ray background further, the most important parameter is the \ac{CRV} efficiency, the complement of which multiplies the entire background rate.
Since this background is completely uncorrelated with the beam, the shorter the running time, the lower this background rate will be.
If this cannot be reduced further, then it might be necessary to add more material above the beamline or above the experiment hall.
Fortunately, rather than depend on simulation for this, \phaseII has the benefit of \phaseI, which will have measured and characterised the cosmic ray flux on the experiment hall floor.

%%Since all previous studies have only required a hit in one or more straw tracker planes for a track to be `accepted', in the interest of consistency we shall do the same here.
%The 19 events between 80 and 120 MeV/c coming from 176 million cosmic ray simulated events imply \VarCosmicRatePerMuPerMeV signal-like electrons per primary muon per MeV/c, or \VarCosmicRatePerMu within the momentum thresholds.
%As such, for the rate of cosmic rays passing through the 50$\times$50 m${}^2$ patch of sky above the hall, we can expect to be 
%With the signal momentum range of \VarMomThreshold to 105.5~MeV/c, and assuming 1 muon per minute per cm${}^2$, 
%
%Of these events, about XXX electrons with signal-like momenta were observed to hit a straw tracker plane, which are shown in the event display in \fig{bg:cosmic:eventDisplay}.
%No events were observed hitting more than one of the straw planes with the necessary momentum to be considered a background, and as such no events would have been reconstructed.
%Accordingly, we can only set a limit on the rate of backgrounds due to cosmic events coming from overhead muons.
%At 90\% C.L.\ the rate must be smaller than XXX to expect to see no background events in \num{1.7e8}

\section{Summary of Background Rates}
\sectlabel{bg:summary}
\TabBackgroundFinalVals
\Tab{bg:summary} gives an overview of the predicted rates.
The overall expected number of background events during \phaseII is \VarTotalBgPhasII events, assuming \VarRunTime~s of run time.
This is dominated by the cosmic ray background and events from antiprotons that stop in the target.
It is clear, though, that further work must be undertaken on these predictions in the future.
In light of the various uncertainties, discussed in more depth below, we have tried to be as conservative as reasonably possible in this estimate; improving the uncertainties would hopefully reduce the estimated number of events.

\clearpage
\subsection{Backgrounds Not Considered}
\sectlabel{bg:notCovered}
\subsubsection*{Neutrons from the Production Target}
Although some of the methods above are different, compared to the \phaseI TDR and \phaseII CDR there is only one source of backgrounds that has not been covered here:
the production of high-energy electrons from the capture of neutrons that originate in the production target.
These neutrons were killed in order to reach a large number of muons and pions in the simulation, although neutrons produced along the muon beam (from the collimators, stopping target, \etc) have been included with the beam background.
Whilst there is a large amount of concrete and iron material in between the production target and the detector, the production target is likely a major source of neutrons in the experiment, and this needs to be studied rigorously using multiple hadron production and neutron transport models.

Such studies have advanced for \phaseI, with the background rate from neutrons estimated to be about \num{3.1e-29}~per \ac{POT}~\cite{TDR2016}.
With a different experiment layout, production target material, and higher proton beam power, separate studies should be performed  for \phaseII.
A na\"{i}ve scaling of this to the \phaseII number of protons on target would suggest that \num{2.7e-8} events can be expected in \phaseII.
Since the detector sits further away from the production target and since there is additional beamline then it is likely that the neutron rate is lower.
%The increased proton power would already be included in the naive scaling applied above to first order.
The change of target material between \phaseI and \phaseII from low to high-Z material will increase the neutron production rate per unit mass of target.
However, to compensate for the equivalent reduction in muon yield that this produces, the target in \phaseI will be longer and as such the neutron rate per POT might well be similar.
%The different target material will change the neutron production cross section, but since the target is also made longer in \phaseI to maximize the muon stopping rate, the increase in neutron yield due to the different target geometry and material might not be significant.

\subsubsection*{Detector Performance}
It has been implicitly assumed in the preceding work that the StrECAL is able to perfectly distinguish electrons from positrons, muons, and pions.
This seems a reasonable assumption given the charge and relativicities of the particles, but it would be important to consider the rate of mis-identification in the event that these particles are able to be transported to the detector.
These particles are already suppressed by the long beamline, however, and only muons have been seen to be transported to the detector in the simulations for this thesis.

In a similar vein, the impact of `pile-up events' should be considered.
Pile-up occurs when two separate particles enter or strike the same region of a detector in a shorter space of time than the detector can resolve.
In the ECAL, pile-up could be an issue for triggering but since the ECAL's primary purpose in the analysis is to provide a time stamp, this should not be an issue for backgrounds.

Pile-up in the Straw Tracker should be easier to resolve because of the finer granularity of the detector, and the use of multiple hits on multiple straws to reconstruct the trajectory.
Instead, it is the occupancy of the Straw Tracker that should be considered.  
How often is a 105~MeV/c electron found due to the mistaken combination of hits in a high-occupancy event?
Machine learning techniques for \phaseI's \ac{CyDet} have been developed to significantly suppress hits from tracks other than signal electrons~\cite{TDR2016} and it is likely that a similar approach will be possible in \phaseII.
In addition, although the beam power and production target material in \phaseII increase the multiplicity of particles in the beam, the longer beamline may well bring the occupancy down to lower levels than observed in \phaseI.

%Other sources of background might exist---putting a limit on all processes that produce fewer than a single electron from $10^{21}$~POT is no easy feat---and it 

\subsection{Impact of Momentum Resolution}
\FigBgVsResolution
The momentum resolution is the key factor in determining the background rates for the intrinsic backgrounds (\acf{RMC} and \acf{DIO}).
However, since the distribution of residuals (the resolution function) is not well known for \phaseII, there is a degree of uncertainty over the size of these backgrounds.

To understand how robust these estimates are to changes to the resolution function, a simple parametrisation for the resolution function was used.
In this simplified scheme, the probability density for mis-reconstructing an electron's momentum by a certain amount, is given by the function:
\begin{equation}
	P(p_\textrm{reco}-p_\textrm{true}) = P(\Delta{}p) \propto e^{(\Delta{}p/\sigma_1)^2}+\kappa e^{(\Delta{}p/\sigma_2)^2}
\end{equation}
where $\kappa$ is dimensionless and used to set the relative weight of the second Gaussian, and $\sigma_1$ and $\sigma_2$ define the widths of the respective Gaussian.

\Fig{bg:resolution} shows the background rates due to \ac{DIO} and \ac{RMC} as a function of $\sigma_2$ for different values of $\sigma_1$ and $\kappa$.
The current \phaseI results suggest $\kappa$ to be about 0.08, and $\sigma_1$ to be less than 200~keV/c, and so these values are used as baseline estimates.
To produce the plot in \fig{bg:resolution}, we fix the momentum thresholds to be \VarMomThreshold and \VarMomThresholdHigh, as found from the sensitivity estimates.
Over the range of $\sigma_2$ values shown in \fig{bg:resolution}, the signal acceptance changes by less than 4.2\%, given these momentum thresholds.
  
From this it is clear that, provided the value of $\sigma_2$ is kept below 1~MeV/c, fewer than 0.01 background events would be expected from \ac{RMC}, whilst the background rate from RMC grows exponentially above this point.
However, it is also clear that the rate from DIO electrons will dominate regardless of the resolution.
This should not be surprising given that the \ac{DIO} spectrum reaches up to the \mueconv end-point whilst the RMC end-point is separated by some 2.4~MeV/c.

\subsection{Principal Uncertainties}
Whilst it is not possible to make a quantitative estimate of the uncertainty at this time, it is clear that for some backgrounds the uncertainty is large.

\begin{description}
\item[Statistical Uncertainty and Use of Extrapolation]
In many instances the generated sample sizes for these background predictions were too small to directly evaluate the background rate.
In these cases it was necessary to extrapolate from parameters and regions with higher statistical power, such as lower energy parts of a spectrum, earlier in time, or use a more upstream part of the experiment.
In addition, it was often necessary to resample particles produced from an earlier stage of the simulation, such as for the muon and pion stopping distributions, or the beam fluxes leaving the production target region.
This introduces a different kind of error, which is very difficult to quantify, namely that there will now be some correlation between the events seen downstream.
In the future, generating larger Monte Carlo samples using the continually increasing computing power available to the COMET experiment will be important in reducing this uncertainty.

\item[StrECAL Reconstruction and Resolution]
In the case of \ac{DIO} and \ac{RMC}, the two intrinsic sources of background, the main source of uncertainty is in the resolution function.
The exact distribution for the residuals after reconstruction is not well known for the \ac{StrECAL}, especially for \phaseII.
In particular, for \ac{RMC}, the high-energy tail needs to be studied better. 
It has been assumed that fewer than \VarResolutionProbFromRMCToSignal events will be reconstructed with momentum larger by at least 2.4~MeV/c than the true value.
This assumption must be confirmed, however.

\item[RPC Spectrum]
In the case of \acf{RPC}, the dominant uncertainty comes from the spectrum of photons produced during RPC.
The study here and in the \phaseI TDR depend entirely on a single set of experiments, which do not cover RPC from aluminium, only on magnesium and calcium.
This will be difficult to improve further without additional observations, and it would be useful to study this at \phaseI if possible, perhaps by changing the Torus1's dipole field to improve pion acceptance.

\item[Antiproton Production]
Given the lack of experimental evidence for antiproton production in the `sub-threshold' energy regime of the backwards direction, there is large uncertainty over the expected rate.
Whilst the model developed here was kept conservative, as one of the dominant backgrounds it is important to study this further.
A dedicated study comparing multiple different hadron production codes to all the possible experimental datasets (including forward and backward produced antiprotons from the different targets with different proton momentum) could reduce our uncertainty if no new experimental data is forthcoming.
Such a study would then be able to extrapolate better for the target material, proton energy, and production direction used in COMET.

In addition to the production rates, there is reasonable uncertainty on the mechanisms by which antiprotons can produce background electrons.
Here it has been assumed that the primary mechanism is via pion production, but the rates for such production in aluminium is also not well known.
Such an assumption must also be tested.
%Direct emission of high-energy electrons is supposed to be relatively small~\cite{}, but it has not been rig

\phaseI will be able to play a sizeable role in reducing this uncertainty as one of the first experiments to look at antiproton production under the \COMET conditions.
The fact that the production target will likely be made from graphite in \phaseI as opposed to the tungsten of \phaseII reduces the direct usefulness of such measurements however.
On the other hand, given that there is more (although still rather limited) data for graphite will allow \phaseI measurements to better understand the transportation of antiprotons along the COMET beamline.
Furthermore, in addition to antiproton measurements during the main \phaseI runs, additional runs are under consideration to specifically study the antiproton background.
These could involve changing the production target material is to something other than graphite or changing the beam energy and studying the change in antiproton yields.

\item[Cosmic Ray Veto Geometry and Shielding]
Without a complete \ac{CRV} design, it is tough to be confident that the energy loss of cosmic muons is well captured in the \phaseII simulation.
In addition, the miss-rate of \num{1e-4} should be experimentally confirmed and ideally improved.

Again, after \phaseI the uncertainty on this background should be significantly reduced.
The muon flux in the experiment hall, at the beamline position will have been measured to a high precision in all directions.
Furthermore, since the CRV design will be largely unchanged the miss-rate will be well understood.

\item[RMC on Al-26]
RMC on Al-26 was also discussed as a potential concern, since RMC on Al-26 can produce signal-energy photons. However Al-26 has a tiny abundance in natural aluminium, and the rate of production in the COMET muon beam is difficult to predict.
This will be measurable at \phaseI, so that by \phaseII, the uncertainty on this process should be much smaller.

\item[Electron Spectrometer Dipole Field]
	The acceptance into the detector is affected by the field of the preceding beamline. 
	For the Electron Spectrometer, the unphysical uniformity and lack of fringe field in the simplistic dipole field map used here may introduce effects that would not be seen in the real experiment.
\end{description}

\section{Further Studies and Improvements}
\sectlabel{bg:improvements}
This chapter has presented a near-comprehensive set of estimates for background sources in \phaseII.
It should not, however, be treated as the final set of predictions, but rather as an initial set of studies to be developed further.
As well as the previous suggestions regarding uncertainties,
the following list suggests additional methods that could be pursued to improve and expand the preceding set of estimates.

\begin{description}
	\item[Explore techniques for improving the statistical power of limited Monte Carlo data]
		In addition to the resampling and extrapolation techniques used here, cross checks with various other methods should be explored.
		Together these would provide an increase in the statistical power of a limited set of simulated data, and provide a validation and uncertainty estimate for one another.
		Methods such as particle weighting and biasing could also be employed.
		Additionally, the transmission of each section of the beamline could be studied individually, building a set of multi-dimensional (particle type, momentum, position, \etc) transfer matrices.
		The total beamline acceptance for a given particle is then the product of these matrices.
	\item[Use different hadron production models] Primarily this is needed to cross check the muon and pion yields, and the production target optimisation.
		This is already well accommodated in ICEDUST, but should be exercised properly.
	\item[High-momentum electron acceptance vs. beamline position] Many background estimates would benefit from knowing the downstream acceptance of high-momentum electrons at a given location in the beamline.
		This could be a multi-dimensional histogram or matrix, and would ideally also contain the acceptance's dependence on the electron direction, and transverse location.
		With this information, simulations for the cosmic ray backgrounds, photon conversion after RPC and RMC, and beam simulations would only need to know the probability of creating a high-momentum electron at a given location to deduce the resulting background rate.
		Optimisation studies would also benefit, as mentioned in chapter~\sect{phaseII-optimisation}.
\end{description}

The timescale for these studies is fairly varied: whilst \phaseII is several years away still, in some cases the true deadline will be \phaseI.
Decisions for the design of \phaseI will impact, in some cases, the design of \phaseII, whilst the run plan for \phaseI will include dedicated studies aimed at reducing key uncertainties of \phaseII.
For both of these reasons it is important that there is a continuous effort to maintain and refine the understanding of COMET \phaseII.
%\begin{easylist}
%# Neutron background with different hadron codes and \alcap neutron spectrum from mu capture
%# Improved cosmic ray veto geometry
%# High-energy electron acceptance
%# Pion stops elsewhere in the beamline
%# StrECAL resolution function
%# Improved proton beam timing structure 
%# RMC from Al-26 that is produced from Al27 via neutrons. Photons can be made up to 108 MeV
%# Assumption that the PID abilities of the StrECAL are 100\%, is this true? At what level would muons be reconstructed at electrons?
%# RPC from pions stopping elsewhere in the beam -> Map of acceptance to RPC electrons vs. position of pion stop in the beamline
%\end{easylist}
