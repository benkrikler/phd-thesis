\chapter{Phase-II Optimisation}
\sectlabel{phaseII-optimisation}
\begin{easylist}
# Before a substantial sensitivity estimate can be made, need a solidly optimised design
# Aiming for \sensePII within a single year of running
# Designs previously optimised \cite{CDR}, and these results are used as nominal design / starting point
# Fresh optimisation using new software / simulation, updated fieldmaps, physics lists and geometry
# Some aspects fixed already since \phaseI under construction: Experiment hall, Torus1, detector solenoid, fieldmap and coil parameters?
# Key areas for optimising:
## Production target dimensions
## Torus1 dipole field strength
## Torus2 dipole field strength
## Electron spectrometer dipole field strength
## collimator shapes and locations
## stopping target and beam blocker position and form
## DIO blockers on spectrometer
\end{easylist}

\section{Optimisation Strategy}
\begin{easylist}
# Take some aspects as fixed
# Limit scope and approach:
## Ideally, each aspect optimised in combination to maximise signal acceptance and reduce background
## How decoupled are each section?
## In practise such an optimisation is not easy to do, instead aim to produce a baseline optimisation so that all backgrounds / issues can be identified
## This can then form basis for further optimisation, with perhaps a smarter more integrated approach
# Method:
## Production target optimisation
### Maximise muon and pion yield between 0 and 80 MeV at entrance to muon beamline
### Parameters to vary: target length, target radius
## Muon beam optimisation
### Maximise muon stopping rate in stopping target
### Minimise pion stopping rate
### vary dipole along TS2 and TS4
### vary Collimators: TS2 and at TS3
## Electron spectrometer optimisation
### Optimise dipole to increase signal acceptance
### Optimse DIO blockers so DIO rate per straw is less than 1 kHz
### Vary solenoidal field to increase separation?
## Stopping target / beam blocker optimisation
### Maximise reflection of signal electrons from upstream by tuning target position
## Detector optimisation
\end{easylist}

\section{Optimisation Goals}
\begin{easylist}
# Set sensitivity goal and optimise to reach this
# Single event sensitivity only considers signal acceptance, but also need to understand backgrounds in terms of final confidence limit that can be set
\end{easylist}

\section{Production Target Optimisation}
In the \phaseII \ac{CDR}, the production target is given as being 16~cm in
length and 4~mm in radius~\cite{CDR}.  Since then, there have been changes to
the magnetic field in this region, as well as the lengths and locations of
solenoids, shielding and beam-pipe, and the proton beam.  Previous studies have
looked at comparing the Tungsten target proposed for Phase-II to other
materials~\cite{thesis-AEdmonds}, and also drawn a comparison between
MARS~\cite{MARS}, Geant4~\cite{Geant4} and the limited data available.

The goal in this study then is to optimise the production target with the
up-to-date configurations.  This study aims to maximise the total muon and pion
yield below 80~MeV at the entrance to the Torus1 bent solenoid, by varying the
radius and length of the production target.

%\begin{easylist}
%# Some studies in the past (Andy's thesis) have compared Tungsten with Graphite and optimised Phase-I target
%# Geometry in CDR was 16~cm length and 4~mm radius
%# Proton beam, capture solenoids and radiation shielding has changed since then, so re-optimisation of the production target was needed
%\end{easylist}

\subsection{Configuration} 
%\icedustDescription{one}{tow}
%begin{easylist}
% Configuration
%## Only using Geant4 in ICEDUST
%# ICEDUST description: 
%## \icedustDescription{heads/1512w51_develop(3a0ee59)__3_UNCOMMITTED__}{heads/Patch_Geant4-G4MultiLevelLocator(11fc8f0)}
%# Production target:
%## Material: Tungsten
%## Back of target fixed 8~cm behind nominal centre (intersection of Proton beam axis and muon beam-line axis)
% Proton beam configuration
%## Uncertainty over beam profile makes this exercise difficult
%# Treat incoming beam as a pencil beam which is not realistic, but reasonable for this study \CHECK{Is it? Need to quantify this}
%# Use proton beam parameters from TDR2014:
%## Double Gaussian with horizontal spread of 5.8 mm and vertical spread of 2.9 mm.
%## Energy distribution also Gaussian with mean kinetic energy 8.01 GeV and 0.135~MeV spread in sigma.
%## The beam was fed in 1~cm upstream of target front face, which is the largest cause for errors in this optimisation study.
%end{easylist}

\Tab{optimisation:ProdTgtSec:configuration} gives the key parameters for the
beam input and other aspects of this simulation.  The location and orientation
of the target were held fixed, since the proton beamline is fixed with respect
to the muon beam axis.  Once a realistic proton beam becomes available, these
values would also benefit from optimisation, however.  During the scan over
length, the back face of the target was kept 8~cm away from the muon beam since
the radiation shielding has previously been optimised, and since beyond this
the magnetic field will no longer be able to capture the pions and muons
produced.

It must be noted that at this point in time there is an appreciable
uncertainty in the proton beam profile and position.  In particular, whilst the
proton beamline upstream has been delivered, the effect of the magnetic
field and necessary dipole and quadrupole magnetics are still being studied by
the proton beam-line group.  The beam profile is given in the \phaseI \ac{TDR}
as having a Gaussian profile and energy distribution, but no divergence or
location is given.  The effect of the proton beam distribution on the overall
sensitivity shall therefore be considered later on.

Protons originated from a plane (distributed as a two dimensional Gaussian
across this surface) but since there is therefore some scope to tune the proton
beam's position, the input particle plane was moved to remain 1~cm away from
the front surface of the target.  Since the aim is to maximise the muon and
pion yield by varying only the length and radius, shifting the proton beam
input plane in this way removes any variation of target acceptance due to
divergences of the proton beam in the magnetic field.

\begin{table}[t]
\centering
\begin{tabular}{lp{0.7\textwidth}}
	\hline
	\multicolumn{2}{l}{\bf Proton Beam}\\
	Horizontal spread, $\sigma_x$ & 5.8 mm \\
	Vertical spread, $\sigma_y$ & 2.9 mm \\
	Mean energy, $\mu_E$ & 8.01 GeV \\
	Energy spread, $\sigma_E$ & 0.135 MeV \\
	\hline
	\multicolumn{2}{l}{\bf Target}\\
	Material & Tungsten  \\
	Orientation & 10\degree between target's principal axis and the muon beam axis.    \\
	Location & Back face fixed 8~cm away from muon beam axis.  \\
	Length & 16~cm in CDR. Varied in steps of 4~cm from 4 to 32~cm.\\
	Radius & 4~mm in CDR. Varied from 2 to 10~cm in steps of 2~cm and from 10 to 30~cm in steps of 4~cm.  \\
	\hline
	\multicolumn{2}{l}{\bf Software configuration}\\
	Packages & \ttfamily{heads/1512w51_develop(3a0ee59)__3_UNCOMMITTED__}  \\
	Externals & \ttfamily{heads/Patch_Geant4-G4MultiLevelLocator(11fc8f0)}  \\
	Fieldmap & 160104 \CHECK{} \\
	\hline
	\multicolumn{2}{l}{\bf Sample Sizes}\\
	Length scan &  3e5 \ac{POT} (30 runs of 1e4) \\
	Radius scan & 4.9e5 \ac{POT} (49 runs of 1e4)  \\
	Final scan &   \\
\hline
\end{tabular}
\caption{Key parameters in the configuration of the Production Target optimisation.}
\tablabel{optimisation:ProdTgtSec:configuration}
\end{table}

\subsection{Length Scan}
Different length targets were simulated with 3e5 POT per length.
Target length was varied in steps of 4~cm from 4 to 32~cm, whilst the target radius was held fixed at the CDR value of 4~mm.

\Fig{optimisation:ProdTgtSec:Length:Momentum} shows the momentum distributions of pions and muons for different target lengths.
Target length is given as half-length which is the Geant4 convention.  
\Fig{optimisation:ProdTgtSec:Length:Integral} then shows these distributions integrated up to different momentum.
From these plots it can be seen that for both muons and pions, the optimum target length occurs around a total length of 32~cm.

Additionally it can be seen from
\fig{optimisation:ProdTgtSec:Length:IntegralRatio} that the shape of the
momentum distributions changes only weakly as a function of the target length.
These plots were produced by normalising the integrated momentum contours of
\fig{optimisation:ProdTgtSec:Length:Integral} to the total integral below
400~MeV.  As a result, it is possible that the actual shape variation is even
weaker than apparent here, since in the present sample size, the high momentum
tail is not well sampled at small target lengths, such that a skew in the
normalisation might occur.

\begin{figure}[t]
\centering
\subfloat[][\figlabel{optimisation:ProdTgtSec:Length:Momentum:Muons}Muons]{\includegraphics[width=0.45\textwidth,trim=0 0 0 1.5cm,clip]{figs/optimisation/ProdTgtGeom/Length_mu-minus_momentum}}
\subfloat[][\figlabel{optimisation:ProdTgtSec:Length:Momentum:Pions}Pions]{\includegraphics[width=0.45\textwidth,trim=0 0 0 1.5cm,clip]{figs/optimisation/ProdTgtGeom/Length_pi-minus_momentum}}
\caption{
Change to momentum distributions at the entrance to the first 90 degrees of the bent muon beam solenoid for different target lengths.
Target length is given as half-length which is the Geant4 convention.  
}
\label{optimisation:ProdTgtSec:Length:Momentum}
\end{figure}

\begin{figure}[t]
\centering
\subfloat[][\figlabel{optimisation:ProdTgtSec:Length:Integral:Muons}Muons]{\includegraphics[width=0.45\textwidth,trim=0 0 0 1.5cm,clip]{figs/optimisation/ProdTgtGeom/Length_mu-minus_integral_toZero}}
\subfloat[][\figlabel{optimisation:ProdTgtSec:Length:Integral:Pions}Pions]{\includegraphics[width=0.45\textwidth,trim=0 0 0 1.5cm,clip]{figs/optimisation/ProdTgtGeom/Length_pi-minus_integral_toZero}}
\caption{\figlabel{optimisation:ProdTgtSec:Length:Integral}
Integrated muon and pion yields up to a certain momentum at the entrance to the first 90 degrees of the bent muon beam solenoid as a function of target length.
}
\end{figure}
\begin{figure}[t]
\centering
\subfloat[][\figlabel{optimisation:ProdTgtSec:Length:IntegralRatio:Muons}Muons]{\includegraphics[width=0.45\textwidth,trim=0 0 0 1.5cm,clip]{figs/optimisation/ProdTgtGeom/Length_mu-minus_integral_ratios}}
\subfloat[][\figlabel{optimisation:ProdTgtSec:Length:IntegralRatio:Pions}Pions]{\includegraphics[width=0.45\textwidth,trim=0 0 0 1.5cm,clip]{figs/optimisation/ProdTgtGeom/Length_pi-minus_integral_ratios}}
\caption{\figlabel{optimisation:ProdTgtSec:Length:IntegralRatio}
Change in the momentum distribution of muons and pions at the entrance to the first 90 degrees of the bent muon beam solenoid as a function of target length.
}
\end{figure}

\subsection{Radius scan}
In parallel to the length optimisation scan, different radii targets were also simulated.
Targets with radii of 2, 4, 6, 8, 10, 14, 18, 22, 26, and 30~mm were tested.
The target length was held at the \ac{CDR} value of 16~cm in total.

The results of these scans can be seen in \fig{optimisation:ProdTgtSec:Radius:Momentum} and \fig{optimisation:ProdTgtSec:Radius:Integral},
where it can be seen that a maximum in both the muon and pion yields at the entrance to the Torus1 section is achieved at a radius of about 10~mm.
As in the length scan, the shape variation of the momentum distributions is rather weak as a function of target radius.

\begin{figure}[t]
\centering
\subfloat[][\figlabel{optimisation:ProdTgtSec:Radius:Momentum:Muons}Muons]{\includegraphics[width=0.45\textwidth,trim=0 0 0 1.5cm,clip]{figs/optimisation/ProdTgtGeom/Radius_mu-minus_momentum}}
\subfloat[][\figlabel{optimisation:ProdTgtSec:Radius:Momentum:Pions}Pions]{\includegraphics[width=0.45\textwidth,trim=0 0 0 1.5cm,clip]{figs/optimisation/ProdTgtGeom/Radius_pi-minus_momentum}}
\caption{
Change to momentum distributions at the entrance to the first 90 degrees of the bent muon beam solenoid for different target radii.
}
\label{optimisation:ProdTgtSec:Radius:Momentum}
\end{figure}

\begin{figure}[t]
\centering
\subfloat[][\figlabel{optimisation:ProdTgtSec:Radius:Integral:Muons}Muons]{\includegraphics[width=0.45\textwidth,trim=0 0 0 1.5cm,clip]{figs/optimisation/ProdTgtGeom/Radius_mu-minus_integral_toZero}}
\subfloat[][\figlabel{optimisation:ProdTgtSec:Radius:Integral:Pions}Pions]{\includegraphics[width=0.45\textwidth,trim=0 0 0 1.5cm,clip]{figs/optimisation/ProdTgtGeom/Radius_pi-minus_integral_toZero}}
\caption{\figlabel{optimisation:ProdTgtSec:Radius:Integral}
Integrated muon and pion yields up to a certain momentum at the entrance to the first 90 degrees of the bent muon beam solenoid as a function of target radius.
}
\end{figure}
\begin{figure}[t]
\centering
\subfloat[][\figlabel{optimisation:ProdTgtSec:Radius:IntegralRatio:Muons}Muons]{\includegraphics[width=0.45\textwidth,trim=0 0 0 1.5cm,clip]{figs/optimisation/ProdTgtGeom/Radius_mu-minus_integral_ratios}}
\subfloat[][\figlabel{optimisation:ProdTgtSec:Radius:IntegralRatio:Pions}Pions]{\includegraphics[width=0.45\textwidth,trim=0 0 0 1.5cm,clip]{figs/optimisation/ProdTgtGeom/Radius_pi-minus_integral_ratios}}
\caption{\figlabel{optimisation:ProdTgtSec:Radius:IntegralRatio}
Change in the momentum distribution of muons and pions at the entrance to the first 90 degrees of the bent muon beam solenoid as a function of target radius.
}
\end{figure}

\subsection{Final Result}
Since the length and radius scan were performed in parallel, a final cross check was performed where the optimal radius was confirmed at the optimised target length.
The integrated spectrum is shown in \fig{optimisation:ProdTgtSec:Combined:Integral} where it can be seen that the optimum radius once the target length is increased to 32~cm is still 10~mm.

\begin{easylist}
	# Figure comparing Phase-II to Phase-I
	# Conclusions
\end{easylist}

\begin{figure}[t]
\centering
\includegraphics[width=0.45\textwidth,trim=0 0 0 1.5cm,clip]{figs/optimisation/ProdTgtGeom/Radius_pi-minus_integral_ratios}
\caption{
Comparison of the combined muon and pion yield at the entrance to Torus1 for \phaseI (blue) and \phaseII (red).
}
\figlabel{optimisation:ProdTgtSec:ComparePhI-II}
\end{figure}

