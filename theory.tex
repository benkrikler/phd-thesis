\chapter{Theory}
\chapterquote{Who ordered that?}{The author upon presentation of `ika-no-ikizukuri' (live squid sashimi) at a collaboration meeting in Fukuoka, Kyushu Island}
%\begin{easylist}
%# "Who ordered that?" -- The author upon presentation of `ika-no-ikizukuri' (live squid sashimi), in Fukuoka, North Kyushu
%	## Not predicted before observation
%	## Neutrino masses and oscillations one of the only observations of our time not to have been predicted
%	## 
%# History of the muon and its importance to the development of the standard model
%	## Early searches for mu to e gamma,
%	##	multiple neutrino flavours,
%	##	observation of muon neutrino
%# Lepton Flavour: Neutrinos and the Charged Leptons
%# Neutrino oscillations, PMNS matrix, neutrino masses,
%# Charged Lepton Flavour Violation searches
%	## Theoretical motivation
%	   ### Lepton non-universality => CLFV
%	   ### Baryon asymmetry
%	   ### Neutrino mass generation and scale
%	## Example models: SUSY
%	## Example models: neutrino seesaw? 
%	## Observables and current limits
%	## Interplay between mu-e conversion and mu -> e gamma experiments
%# Muon-to-electron Conversion and the Physics of Muonic Atoms
%\end{easylist}

\section{The Standard Model }
%At the turn of the 20\textsuperscript{th} century, physics felt itself on confident ground.
%Newton's laws of mechanics had been applied to large ensembles of particles producing the statistical theory of thermodynamics which had driven the industrial revolution.
%Maxwell had unified the electric and magnetic fields which led to the development of electricity and electrical components such as capacitors and inductors.
%For a while the theory of electromagnetism was in tension with Newton's mechanics and the concept of an invisible omnipresent `aether' seemed a reasonable explanation.
%The Michelson-Morley experiments, however, laid this to rest proving light to travel at the same speed regardless of the speed of the observer or source.
%It was not long though before Einstein, working in his patent office, used this to develop the theory of special relativity.
%
%Around the same time, the ultra-violet catastrophe that predicted infinite energy be emitted due to the heat of an object in the ultra-violet part of the spectrum was solved by Planck, bringing the quantisation of light and the concept of a photon.
%It was Einstein again who used this to explain the photo-electric effect that eventually won him the Nobel prize.
%Finally, it was the explanation of the hydrogen atom's emission spectrum by Bohr, using de Broglie's concept of an electron wave, and the theory of quantum mechanics was born.
%
%At this point, atoms were made of neutron, protons and electrons, coming a long way from the plum-pudding 
%
%----
%
%In the early 1930's, physicists felt themselves on confident ground, having completely rewritten the physics textbook during the last quarter century.
%On large scales, 400 year old Galilean invariance had given way to special relativity, whilst Newtonian mechanics and gravity had fallen to the general theory of relativity.
%And on the tiniest of scales, quantum mechanics had revolutionised the behaviour (and our philosophy) of matter and energy.
%Electrons, protons, and neutrons were the children that filled the playing field of this new school, as well as the photon which often made appearances.
%
%Electrons were known to hang around protons and neutrons which themselves were normally found in even more tightly-bound `nuclei'.
%The electromagnetic attraction that the electrons felt for the protons (and therefore the nucleus as a whole, given the neutrons' electromagnetic indifference) was well understood.
%Theories existed to explain why occasionally a neutron would be seen to change into a proton, releasing an electron in the process.
%Other theories were also able to explain the strong force that seemed to keep the protons and neutrons together, proposing a new member that would appear and disappear quickly enough that no-one outside the nucleus would see it.
%
%This theory had been proposed by one Yukawa~\cite{Yukawa}
%
%----
%
%I remember being a school student and asking my chemistry teacher was how the nucleus of an atom, consisting only of positive protons and neutral neutrons, was able to stay together despite the repulse electrostatic forces.
%It was a slightly loaded question, since I already knew of the strong force, but I hoped that at he would enlighten me as to how it worked.
%I did not know this then, but physicists in the 1930s had asked almost the identical question, realising there must be some additional force that holds the protons together.
%
%One Japanese physicist, Hideki Yukawa,suggested this force was mediated by some particle being exchanged between the protons and neutrons, in a similar way to how the electromagnetic force was known to exist via photon exchanges.
%If this new particle were to be massive and unstable then one could explain how such a strong force would die off so quickly away from the nucleus.
%Better yet, Yukawa was able to estimate the mass of this exchange particle via the uncertainty principle and even predicted how long it should live for: it should be about 100~MeV and live for around $10^{-8}$ seconds \CHECK{Are these the correct values?}.
%
%Then in 1937, a particle with very close to this mass was discovered from cosmic ray events.
%Surprisingly though, the discovered particle lived for around $10^{-6}$ seconds -- much longer than predicted by Yukawa's model, prompting Rabi to make the famous quote that 
%------

%\begin{easylist}
%    # Turn of the century, phsyics nearly complete
%    ## Complete picture of protons, neutrons, electrons
%    ## Suggestion of short-lived meson to mediate the strong force that bound the nucleus together
%    ## Weak force that would sometimes change protons and neutrons but conserve isospin
%    # First un-predicted particle of the SM: Muon
%    ## Long-lived meson but with a mass close to requirement for strong force mediator
%    ## Only decayed to an electron
%    ## Electron had a spectrum of energies, similar to beta-decay measurements
%    ## Massless neutrinos used to explain beta-decay to conserve 4-momentum
%    ## Now need two distinct neutrino flavours and the conservation of flavour conservation
%    ## Searches for the second type of neutrino flavour proved fruitful
%    ## Searches for muon decay to an electron with a photon were unsuccesful
%    # Lepton flavour conservation
%    ## Noethers Theorem: Continuous symmetry generates a conseved current: eg. Electric charge, momentum, energy, total angular momentum
%    ## No such symmetry for lepton flavour
%    ## Embedded in the electroweak theory
%    ## Yukawa couplings and the CKM matrix (I feel uncertain about this, but since I should know it anyway, it's perhaps something I include briefly then make sure to revise before the viva).
%\end{easylist}
In 1897, the British physicist Thomson discovered the first sub-atomic particle: the electron.
In the next 30 years or so, the supposedly indivisible object of the atom was divided multiple times, giving way to the neutron, the proton, and the first of their anti-particles soon to be known as the positron.
Far from the simple, solid sphere previously assumed, a dynamic atom was now understood, with a cloud of negative electrons bound to a positive nucleus consisting of neutrons and protons.

That the electrons were bound to the nucleus was readily understood due to their opposing charges.
What it was that occasionally turned a proton into a neutron, or vice versa, and in the process emitting an electron was less clear.
Meanwhile the nucleus itself posed a further problem was posed, since something had to be overcoming the repulsion between the like-charged protons.

In 1935, Yukawa -- a physicist from Japan, where this thesis will often return -- proposed that, as for the electromagnetic force binding electrons to the nucleus via the exchange of some carrier particle (the photon), 
an exchanged particle could explain the strong force that bound the protons and neutrons in a nucleus.
Unlike the photon though, this particle would be have to be massive and readily absorbed to the protons and neutrons.
Yukawa was even able to predict the mass of this particle using the uncertainty principle to be around 100~MeV~\cite{}.

\CHECK{Add figure with original images from the muon discovery}
In 1937 a particle with a mass very close to this prediction was observed in cosmic ray events.
But the initial hopes that this was indeed the Yukawa particle faded quickly as this particle easily penetrated through the matter of the detectors, whilst Yukawa's particle should be rapidly absorbed.
So unexpected was this new particle with its mass in between that of an electron and a proton and its relatively long lifetime, that Rabi is was also forced to ask, ``Who ordered that?'
It took some time, but eventually this particle became known as the muon.

The muon was interesting because it seemed to interact very weakly with matter, and because it seemed only to decay to an electron.
Given that an electron has 0.511~MeV, about 200 times lighter than a muon, the obvious question was in what way the muon could decay to an electron.
For this to happen, something else has to be emitted in order to conserve 4-momentum.
There were two obvious possibilities for the `something else': a photon, or a neutrino-antineutrino pair.

Neutrinos were particles that had up to now been a theoretical tool to help explain beta decay.
Beta decay occurs when a nucleus emits an electron or positron and in so doing swaps a neutron for a proton or a proton for a neutron.
The difference in the mass between the original and final nucleus was a fixed value, and yet electrons were seen with a range of energies all less than the value of this difference.
The neutrinos was proposed as a solution: a massless particle that interacted very weakly therefore being nearly impossible to detect would be able to carry away the missing energy.
By studying the spin of the parent and daughter nucleus it was also clear that the neutrino had spin of $\hbar/2$.

%For the decay of the muon there were both similarities and differences with beta-decay.
%Similarities because it only produced electrons with a range of energy less than the mass of the muon.
%Differences because the range of energies could not really be explained by the production of a single neutrino.
%By conservation of total angular momentum, the number of neutrinos had to be two.

Muon decay was similar in that the electron had less energy than was available to it, but different because it appeared from the spectrum of electrons that not one but two neutrinos were being emitted.
Being spin half particles, either the neutrino was its own anti-particle, or one would be a neutrino and the other an anti-neutrino.
Either way, having two neutrinos emitted posed its own challenge since the two neutrinos would be able to annihilate with one another, and muon decay to a photon and electron would become comparably large.
Searches for a muon decaying to a photon and electron were performed~\cite{}, but came back empty handed;
clearly something else, something new, had to be introduced to distinguish one neutrino from the other, such that they were unable to annihilate one another.
\CHECK{Add figure with configuration of original \muegamma experiment}.

This something became known as 	`lepton flavour'.  
One neutrino carried away the `flavour' of the muon, the other carried the `flavour' of the electron.
If you start out with a muon, you must keep either a muon or a muon-neutrino; if you start with an electron you must finish with an electron, or an electron-neutrino.
This is nowadays known as lepton flavour conservation and was cemented into theory when muon-neutrinos were identified in an experiment at Brookhaven that saw muons being produced from the neutrinos emitted when
a pion (the modern name for Yukawa's particle) decays to a muon~\cite{MuNeutrinoDiscovery}.
The discovery of the muon therefore had not only suggested a new charged particle but also a new type of neutrino and a new law of conservation!

Noethers theorem tells us that for every system with a continuous symmetry, some quantity will remain conserved.
This is the rule that gives us conservation of momentum, energy and angular momentum (the conserved properties) in systems that are the same regardless of the place, time, or direction (the continuous symmetries).
An extension of this theorem allows applies for local transformations of the systems `gauge', which is any property that when changed has no impact on the physics outcome, such as the absolute value of the ground in an electric circuit.
In particle physics, this extension gives rise to the various particle charges, such as the electromagnetic charge caused by the $U(1)$, or simple phase, of a particle's wavefunction and the even more abstract $SU(2)$ and $SU(3)$ hypercharges of the weak and strong (colour) forces.
In the case of the conservation of lepton flavour however, no such symmetry exists.
Instead, this conservation is embedded into the \ac{SM} of particle physics through the electroweak theory and lepton universality.

In addition to the leptons and their interactions via the electroweak forces, the \ac{SM} describes the quark sector and their interactions via \ac{QCD}.
Quarks are the particles that make up the neutrons and protons (not only have we divide the atom, but we also its constituents) and even the Yukawa's predicted particle, the pion.
Built around the frameworks of local gauge invariance, quantum field theory, and spontaneous symmetry breaking, the \ac{SM} has been one of the most rigorously tested theories, yet has held up incredibly well to measurements so far.

\section{Neutrino oscillations}
%\begin{easylist}
%    # Break lepton flavour conservation
%    # Neutrinos have mass
%    # PMNS matrix, similar to CKM matrix but with very different form
%    # Additional source of CP violation, good for baryon asymmetry?
%    # Majorana neutrinos?
%    # Charged lepton flavour violation via neutrinos
%    ## Mu to e gamma via a neutrino-W loop
%    ## \mueconv via neutrino loops
%\end{easylist}
In the 1970s, evidence began to emerge that the concept of Lepton Flavour Conservation might have holes.
Raymond Davis Jr. and his group at Brookhaven measured the number of electron neutrinos coming from the sun and found there to be about one third too few~\cite{}.
This puzzle remained open until it was solved by experiments in Canada and Japan in the early 2000s - the neutrinos were being produced in the expected quantity but as they travelled from where they were produced to the detectors on earth, they were changing their flavour and evading detection!
Nowadays, experiments in Japan and the US, such as the T2K experiment~\cite{T2K:nim} and Nova~\cite{}, produce beams of muon neutrinos only to detect them hundreds of kilometres away as electron neutrinos.

To explain this, the neutrinos are now understood to have mass as opposed to their original proposal as massless, albeit at a scale well below any other massive particle.
Not only do they have mass, but the neutrino states with definite mass are not the same as the flavour states by which neutrinos interact with everything else.
Since it is the mass eigenstate that determines a particle's propagation, but the flavour eigenstate that determines the neutrinos interaction, an oscillation occurs where the probability of detecting a neutrino in a given flavour state depends on how far it has propagated since production.

As for the discovery of the muon, the unexpected \CHECK{wasn't it?} discovery of non-zero mass of the neutrino and the mixing between flavour states has opened a whole host of new questions.
How does the neutrino acquire its mass?
  Why is the mass scale so much smaller than any other observed particle?
As the only chargeless, massive fermion in the \ac{SM}, what is the nature of this mass?
Is it produced via an interaction with the Higgs or via a Majorana mechanism, in turn making the neutrino into its own antiparticle?

\CHECK{Add figure with \muegamma via neutrino}.
Either way it is clear that the neutrinos themselves do not conserve lepton flavour, and this immediately makes it possible that the charged leptons -- the electron, the muon, and the tau -- also break Lepton Flavour Conservation.
How this arises is clear from \fig{theory:mu-e-gamma-viaNu}, where the muon neutrino annihilates with the electron neutrino whilst energy and momentum are conserved by the photon.
This is in fact the same diagram that motivated lepton flavour in the first place, and yet now we know it to be possible!
However, the rate for such a diagram is heavily suppressed, given by:
\begin{equation}
\mathcal{BR}(\mu\rightarrow{}e\gamma)\propto\left|\sum_iU^*_{\mu i}U_{ei} \frac{m^2_{\nu_i}}{M^2_W}\right|^2 < 10^{-54},
\end{equation}
Where $U_{\alpha i}$ is an element of the mass-mixing matrix transforming the $\alpha$ flavour state to the $i$-th mass state with mass $M_{\nu_i}$, and $M_W$ is the mass of the $W$-boson.
This rate can be considered double suppressed: the summation over the different mass eigenstates produces a GIM suppression, whilst the mass imbalance between the neutrinos and the $W$-boson suppresses this further.

Emission of a photon is not the only process made possible by neutrino oscillations and \ac{LFV}.
A negative muon that has become bound to the nucleus of an atom (discussed further in \sect{theory:atomicMuon}) will also be able to convert to an electron without emitting neutrinos via the diagrams
shown in \fig{theory:muec-viaNu}.  
In this case however, the rate calculation is complicated by the quark contents of the nucleus, form factors of the quarks in the nucleons and nucleons in the nucleus, and the cancellations that occur between each diagram.
\CHECK{Add figure with four \muec via neutrino diagrams}.

\section{Searches for charged lepton flavour violation}
\begin{easylist}
   # List of observables and current limits
   # Attractiveness of searches with intense muon beams
   # Development of limits for muon LFV searches with time
   # Present experimental searches and hints
\end{easylist}

\section{Models for CLFV}
\begin{easylist}
    # Importance of CLFV from a theoretical perspective
    ## Baryon asymmetry
    ## Minimal flavour violation
    ## Neutrino mass generation
    ## Lepton universality
    # Example model: SUSY
    # Example model: Neutrino seesaw
    # Complementarity between searches: \mueconv vs. \muegamma
\end{easylist}
\section{The \mueconv process}
\sect{theory:atomicMuon}
\begin{easylist}
  # Coherent conversion, Signal, normalisation
  # Muonic atoms
  ## Formation and the electromagnetic cascade to the ground state
  ## Decay in orbit of bound muons
  ## Muon nuclear capture
\end{easylist}
