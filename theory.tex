\chapter{Theory}
\chapterquote{Who ordered that?}{The author upon presentation of `ika-no-ikizukuri' (live squid sashimi) at a collaboration meeting in Fukuoka, Kyushu Island}
%\begin{easylist}
%# "Who ordered that?" -- The author upon presentation of `ika-no-ikizukuri' (live squid sashimi), in Fukuoka, North Kyushu
%	## Not predicted before observation
%	## Neutrino masses and oscillations one of the only observations of our time not to have been predicted
%	## 
%# History of the muon and its importance to the development of the standard model
%	## Early searches for mu to e gamma,
%	##	multiple neutrino flavours,
%	##	observation of muon neutrino
%# Lepton Flavour: Neutrinos and the Charged Leptons
%# Neutrino oscillations, PMNS matrix, neutrino masses,
%# Charged Lepton Flavour Violation searches
%	## Theoretical motivation
%	   ### Lepton non-universality => CLFV
%	   ### Baryon asymmetry
%	   ### Neutrino mass generation and scale
%	## Example models: SUSY
%	## Example models: neutrino seesaw? 
%	## Observables and current limits
%	## Interplay between mu-e conversion and mu -> e gamma experiments
%# Muon-to-electron Conversion and the Physics of Muonic Atoms
%\end{easylist}

\section{The Standard Model }
\begin{easylist}
    # Turn of the century, phsyics nearly complete
    ## Complete picture of protons, neutrons, electrons
    ## Suggestion of short-lived meson to mediate the strong force that bound the nucleus together
    ## Weak force that would sometimes change protons and neutrons but conserve isospin
    # First un-predicted particle of the SM: Muon
    ## Long-lived meson but with a mass close to requirement for strong force mediator
    ## Only decayed to an electron
    ## Electron had a spectrum of energies, similar to beta-decay measurements
    ## Massless neutrinos used to explain beta-decay to conserve 4-momentum
    ## Now need two distinct neutrino flavours and the conservation of flavour conservation
    ## Searches for the second type of neutrino flavour proved fruitful
    ## Searches for muon decay to an electron with a photon were unsuccesful
    # Lepton flavour conservation
    ## Noethers Theorem: Continuous symmetry generates a conseved current: eg. Electric charge, momentum, energy, total angular momentum
    ## No such symmetry for lepton flavour
    ## Embedded in the electroweak theory
    ## Yukawa couplings and the CKM matrix (I feel uncertain about this, but since I should know it anyway, it's perhaps something I include briefly then make sure to revise before the viva).
\end{easylist}
\section{Neutrino oscillations}
\begin{easylist}
    # Break lepton flavour conservation
    # Neutrinos have mass
    # PMNS matrix, similar to CKM matrix but with very different form
    # Additional source of CP violation, good for baryon asymmetry?
    # Majorana neutrinos?
    # Charged lepton flavour violation via neutrinos
    ## Mu to e gamma via a neutrino-W loop
    ## \mueconv via neutrino loops
\end{easylist}
\section{Searches for charged lepton flavour violation}
\begin{easylist}
   # List of observables and current limits
   # Attractiveness of searches with intense muon beams
   # Development of limits for muon LFV searches with time
   # Present experimental searches and hints
\end{easylist}
\section{Models for CLFV}
\begin{easylist}
    # Importance of CLFV from a theoretical perspective
    ## Baryon asymmetry
    ## Minimal flavour violation
    ## Neutrino mass generation
    ## Lepton universality
    # Example model: SUSY
    # Example model: Neutrino seesaw
    # Complementarity between searches: \mueconv vs. \muegamma
\end{easylist}
\section{The \mueconv process}
\begin{easylist}
  # Coherent conversion, Signal, normalisation
  # Muonic atoms
  ## Formation and the electromagnetic cascade to the ground state
  ## Decay in orbit of bound muons
  ## Muon nuclear capture
\end{easylist}
