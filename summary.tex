\chapter{Summary}
\begin{easylist}
# Chapter 1
## The muon has played a historic role in developing the Standard Model of particle physics
## Outstanding questions around the Standard Model, both experimental and theoretical, motivate searches for Charged Lepton Flavour
## CLFV is a very sensitive probe for New Physics that can solve these issues
# Chapter 2
## Searches for CLFV using intense muon beams are particularly sensitive due to the high intensity beams available nowadays 
## There is a strong complementarity and interplay between the various searches from both theoretical and experimental perspectives
## Mu-e conversion is sensitive to all models, but particularly to those that require massive exchange particles, or direct involvement of quarks or gluons
## The Coherent nature of current \mueconv searches makes it very experimentally attractive since the signal requires only a single monoenergetic electron 
# Chapter 3
## The current limit on \mueconv was set in 2006 by \sindrumII to be \senseSindrum
## The COMET experiment is designed to produce around four orders of magnitude improvement on this limit, at \sensePII
## It will reach this with a staged approach, achieving two orders of magnitude better during \phaseI, at \sensePI
## The principal means by which COMET makes these improvements are the use of:
### a high power 8~GeV primary proton beam to maximise the pion yield and suppress antiproton production;
### a 5~T superconducting solenoidal field designed to maximise the capture of backwards produced pion from the production target;
### a long solenoidal muon beam transport section that is bent and has a dipole field and collimators included, which produce a high purity muon beam;
### a pulsed muon beam with a medium-Z target material, giving a longer muon lifetime than previous experiments, allowing for timing information to suppress most beam-related backgrounds
### a large geometric acceptance detector, improved by magnetic mirroring at the stopping target in \phaseII, which has a low material budget and high granularity to ensure better than 200~keV/c resolution
## Performing \phaseI first allows these element to be tested with lower intensity beams and before committing to the full \phaseII design
# Chapter 4
## If any observation is to be confirmed as signal, or to achieve the most stringent limits in the event of a null-observation, the background rate must be kept as low as possible.
## Being able to predict and evaluate this requires a very accurate, very efficient simulation.
## In addition, the reconstruction software must be able to achieve the necessary resolution and keep the high-energy tails to an absolute minimum.
## The integration of all the offline software, including simulation, reconstruction, calibration and analysis, is well underway.
## The ICEDUST framework is the COMET experiment's offline toolset, which was based on the software of the near-detector for the T2K experiment, ND280.
## Three major Monte-Carlo productions have been run using this software, which itself is now used collaboration-wide for various studies.
## All studies presented in this thesis were performed using ICEDUST, including the optimisation studies for \phaseII.
# Chapter 5
## Given the effort and focus on \phaseI,  \phaseII has received less attention, with the last update on performance given in the 2009 CDR.
## Using the now mature ICEDUST software, revisiting the \phaseII design was prudent in order to validate and improve the past estimates and help inform decisions for \phaseI.
## A comprehensive set of optimisations was performed, covering the production target, muon beam transport, stopping target region, and electron spectrometer and detector.
## These optimisations were all summarised in \tab{optim:AllParameters}.
# Chapter 6
## Based on these optimisations, the predicted single signal-event sensitivity per \num{2e7} seconds of beam time is estimated to be \VarPredictedSES.
## This is a factor XXX improvement over the CDR prediction as summarised in \tab{sense:comparisons}.
# Chapter 7
## It is important also as well to understand the backgrounds that would occur with the updated design and improved simulation.
## Although at this stage many of these results are statistically limited and
    thus the errors should be taken as large~\CHECK{Can I be more quantitative?
    Assume error of 100\% or more?}, the total background rate is predicted to be XXX.
## This includes intrinsic backgrounds from DIO, and RMC, prompt and delayed beam related processes like high-energy particles in the beam, RPC and antiproton production, and cosmic events
## Not included is the rate of background events caused by neutrons produced at the production target, and issues related to pile-up and particle miss-identification.
# Future work 
## This thesis, therefore, presents a baseline design and a set of Sensitivity and Background Estimates for \phaseII of the COMET Experiment.
## There are many aspects that could be studied further, or improvements that could be made to the work presented in this thesis, which have been listed at the end of the relevant chapters.
## Work to revisit the stopping target region, for example, has already shown that the sensitivity could be further improved by a factor of about 2.5 and is summarised in appendix~\sect{appendix:stopTgtImprove}.
\end{easylist}
