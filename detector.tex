\newcommand{\FigBentSolenoidRelativeDrift}{
\begin{figure}[t]
\centering
\includegraphics[width=0.6\textwidth]{figs/detector/BentSolenoids_RelativeDrift}
\caption{
Angular dependence of the magnitude of vertical drift in a bent solenoid field.
The total variation (black) remains below 10\% for pitch angles below 50\degree.
}
\figlabel{detector:bent-solenoids:angularDependence}
\end{figure}
}

\newcommand{\FigPhaseII}{
\begin{figure}[t]
\centering
\includegraphics[width=0.9\textwidth]{figs/detector/PhaseII_schematic}
\caption{
Schematic layout of COMET \phaseII. 
The 8 GeV proton beam enters from the top-left, producing (amongst other things) pions.
Pions and muons travelling backwards with respect to the proton beam are then transported around 180 degrees of bent solenoid, during which time most of the pions decay producing an intense muon beam.
About 40\% of these muons then stop in the stopping target (centre of image).
Any electrons coming from  \mueconv are then transported through another 180 degrees of bent solenoid into the detector system.
}
\figlabel{detector:PhaseII:setup}
\end{figure}
}

\newcommand{\FigPhaseI}{
\begin{figure}[t]
\centering
\includegraphics[height=0.4\textheight]{figs/detector/PhaseI_schematic}
\caption{
Schematic layout of COMET \phaseI. 
}
\figlabel{detector:PhaseI:setup}
\end{figure}
}

\newcommand{\TabBackgroundSummary}{
\begin{tabular}{lldd}
     \hline
     \hline\\[-1.8ex]
     Type           & Background & \multicolumn{2}{c}{Predicted number of events per run} \\
                    &  & \multicolumn{1}{c}{\phaseI \cite{TDR2014}} & \multicolumn{1}{c}{\phaseII \cite{CDRphase2} } \\
     \hline\\[-1.8ex]
     Intrinsic & Muon Decay-in-Orbit                       & 0.01              & 0.15    \\
               & Radiative Muon Capture                    & 0.00056           & <0.001  \\
               & $\mu^-$ Capture w/ n Emission             & <0.001            & <0.001  \\
               & $\mu^-$ Capture w/ Charged Part. Emission & <0.001            & <0.001  \\
     Prompt    & Radiative Pion Capture                    & 0.00023           & 0.05    \\
               & Beam Electrons                            & 0.00083           & <0.1^*  \\
               & Muon Decay in Flight                      & \le0.0002         & <0.0002 \\
               & Pion Decay in Flight                      & \le0.00023        & <0.0001 \\
               & Neutron Induced                           & -                 & 0.024   \\
               & Other beam induced B.G.                   & <2.8\times10^{-6} & -       \\
     Delayed   & Delayed Radiative Pion Capture            & \sim0             & 0.002   \\
               & Anti-proton Induced                        & 0.007             & 0.007   \\
               & Other delayed B.G.                        & \sim0             & -       \\
     Cosmic    & Cosmic Ray Muons                          & -                 & 0.002   \\
               & Electrons from Cosmic Ray Muons           & <0.0001           & 0.002   \\
     \hline\\[-1.8ex]
     \multicolumn{2}{c}{Total background}                      & 0.019         & 0.34    \\
     \multicolumn{2}{c}{Signal (Assuming $B=1\times10^{-16}$)} & 0.31          & 3.8     \\
     \hline
     \hline
\end{tabular}
}


\chapter{The COMET Experiment}
%\section{Muon to Electron Conversion: Signal and Backgrounds}
% - COMET stands for COherent Muon to Electron Transitions
% - Cite the experimenter's guide by Bob Bernstein

%Introduction:
The COMET experiment will search for COherent Muon to Electron Transitions with a single-event sensitivity of \sensePII or better.
This amounts to an improvement of four orders of magnitude compared to the current limit by \sindrumII~\cite{sindrum2006}, made possible by significant changes to the way the experiment operates compared to its predecessor.

Reaching such a sensitivity requires that COMET stops many muons in the target, whilst maintaining a high-signal acceptance but
suppressing potential background sources to well below a single event during the lifetime of the experiment.
%\begin{itemize}
%\item stop many muons in aluminium,
%\item have a high-signal acceptance,
%\item suppress potential background sources to well below a single event.
%\end{itemize}
This tension between simultaneously high-signal sensitivity and background suppression can be translated into the more specific requirements of:
\begin{itemize}
\setlength{\itemsep}{-1ex}
\item a very high-intensity, low-energy, and high-purity muon beam,
%\item a low-energy muon beam,
\item a thin stopping target and low-material budget detector,
\item the use of timing information of signal process with respect to backgrounds.
\end{itemize}

The design of COMET realises these goals by employing several novel experimental techniques and, as such, it has been decided to operate in two stages, \phaseI and \phaseII.
\phaseI aims both to help understand these techniques, the muon beam, and key backgrounds rates, as well as making an intermediate measurement of \mueconv at a sensitivity of \sensePI---two orders of magnitude better than the \sindrumII experiment.
\phaseII will follow and should achieve the final objective of \sensePII. 
%Firstly however I will discuss some of the key aspects common to both \phaseI and \phaseII.
%Although \phaseI will run sooner, since it is heavily motivated by \phaseII, I shall describe \phaseII in more depth first and return \phaseI subsequently.

\section{The COMET Signal}
Muon-to-electron conversion, as described in the previous chapter, is the process where a muon, bound within the Coulomb potential of an atomic nucleus, decays to an electron without emitting neutrinos.
If the nucleus is not excited in the process, coherent transitions dominate, and the electron is produced monoenergetically.
In addition to the energy, the lifetime of the muon once bound to the nucleus is also characteristic  of the signal process.

Thus the signal for COMET is a single electron, with a well-defined energy and an exponential lifetime matching that of the bound muon.
In the aluminium target used in COMET, the lifetime of the muon is 864~ns, whilst the energy of the electron will be 104.97~MeV.

\section{Overview of Background Processes}
\sectlabel{detector:background}
The \acf{ses}, defined in the previous chapter, is a statement of the experiment's ability to observe the \mueconv signal.
It does not, however, account for the background rates.
Clearly, for any observation to be confidently labelled as signal or, in the event of a null-observation, in order to set the tightest possible confidence limit, the background rate must be kept comparably low.
In order to understand the design of the \COMET experiment, a simple appreciation of the key backgrounds is therefore necessary.
%To understand the design of both phases of COMET a simple appreciation of the signal and types of backgrounds that must be considered is necessary.

%Given the desired sensitivity of COMET is \sensePI  at \phaseI and \sensePII by \phaseII, background rates must be kept equally rare.

\Tab{detector:backgrounds} summarizes the results of previous studies for background rates at \phaseI and \phaseII.
There are four groups of background source: intrinsic, prompt, delayed, and cosmic.
\TabBackgroundSummary%
\begin{description}
\item [Intrinsic processes]
                    are those that arise from muons stopping in the target and will always be present regardless of the muon beam properties.
Muon \ac{DIO}, which was described in the previous chapter, is one such background.
%One Of these, the dominant background is muon \acf{DIO}, which was described in the previous chapter.
In addition, radiative nuclear capture of a muon is kinematically capable of producing photons very close to the signal energy.
If the photon is converted to a high-energy electron this can produce background events.

%Although the decay of a free muon cannot produce electrons with energy greater than half the muon mass, once bound to a nucleus the neutrinos can be configured to carry away almost no kinetic energy, leaving only the nucleus and electron to determine the kinematics of the end-point configuration.
%From this it can immediately be seen that the maximum energy of the electron produced from muon decay-in-orbit is the same as for \mueconv (up to the neutrino mass), and indeed
%a tail in the spectrum of electrons coming from muon \ac{DIO} extends all the way up to this point.
%However, clearly such a configuration occupies a tiny part of the phase space as can be seen from \fig{detector:DIOSpectrum} which shows the spectrum of electrons from muon \ac{DIO} in aluminium.
%It can be seen that whilst the high-energy tail does reach up to about 105~MeV/c, the rate drops away very steeply above the free-muon decay end-point, falling some 18 orders of magnitude from its peak around 50~MeV/c.
%%Since the nucleus is so massive compared to the electron, it requires very little energy to have it conserve momentum with the outgoing electron.

\item [Prompt and delayed beam-related backgrounds]
                             come from impurities in the muon beam, be they pions, antiprotons, or high-energy muons and electrons.
The key distinction between these two is the timing at which the background is detected with respect to the arrival of the proton beam.
For example, pions that reach the stopping target region are dangerous since they can produce high-energy gamma rays which can pair produce to create 105~MeV electrons.
Since pion capture against a nucleus is extremely fast, the timing of pion-induced backgrounds is determined predominantly by the arrival time of pions into the target region.
In order to survive from the production target to the target region without decaying the pions must be relatively high-momentum: about 60~MeV/c or greater.  

As a result, backgrounds from pion capture are typically expected close to the arrival time of protons at the production target.
Prompt processes such as this are suppressed by using a pulsed proton beam (discussed in more depth later) and ensuring very few protons in between pulses.
Other beam related issues include the decay-in-flight of muons and pions to electrons.
An electron of 100~MeV/c can be produced by muons or pions with momentum greater than 70 or 50~MeV/c respectively, and so the flux of higher energy particles in the beam must also be reduced.

Delayed processes are those where the timing of the proton beam cannot be used to improve suppression since the characteristic time between the primary proton striking the target and the detection of the background for these processes is large compared to the time between beam pulses.
Possible sources of this delay include the mirroring of particles by the magnetic field, or by some heavy particle from the production target producing pions and high-energy electrons.
At a given momentum, antiprotons travel more slowly than muons or pions given their considerably larger mass, washing out any timing information from their production.

\item [Cosmic Backgrounds]
 arise from high-energy muons that pass through the building and enter the detector or beamline.  
Events where a muon decays to an electron, which is then detected as 105~MeV, are counted as backgrounds.
In particular, muons that produce high-energy electrons close to the target are dangerous since cuts on the reconstructed direction and position will be less effective.
\end{description}

All these processes will be discussed and evaluated in more depth in chapter~\sect{backgrounds}.

\section{General Experimental Techniques}
Suppressing background rates while maintaining a high-signal efficiency leads to several novel techniques being used in the COMET experimental set-up.
The following techniques are common to both \phaseI and \phaseII.

\subsection{Proton Beam Energy and Production Target}
The muon beam used in COMET is produced from the decay of a secondary pion beam created by protons striking a target.
If maximising the muon intensity were the only concern, then both the proton beam power and atomic mass of the target material would similarly be maximised since the pion production cross section
grows with these two parameters.
However, the need to suppress background rates and maintain the mechanical and operational stability of the target constrains both of these parameters.

In particular, protons striking an individual, stationary (and, in theory, unbound) nucleon with more than about 5.6~GeV have sufficient energy to produce antiprotons, which travel relatively slowly and can produce backgrounds.
Since the antiproton yield grows very quickly above this threshold, it has been chosen to use protons with 8~GeV kinetic energy.

As well as the beam energy, the intensity is ideally maximised to increase the number of protons on target per second.
For \phaseII running, the main ring will operate at 7~$\mu$A so that a beam power of 56~kW is achieved.  
\phaseI on the other hand will use a lower beam intensity of 0.4~$\mu$A or 3.2~kW.

Whilst a heavy metal target is preferable since it increases the number of nucleons that interact with the proton beam and therefore the pion yield,
the target must maintain its mechanical strength.
This requires the selection of a high-melting point material and possibly the use of active cooling.
To simplify the situation in \phaseI, a graphite target will be used which can be passively cooled by thermal radiation.
In \phaseII however, tungsten has been selected due to its high-melting point of 3422 C, although water cooling may be additionally employed.

\FigPionSpectraVsAngle

Finally, only those pions and muons emitted in the backwards direction with respect to the proton beam are captured and transported to the muon beamline.
This is a strong way to reduce the high-energy components of the muon and pion distributions since the high-energy tail is greatly suppressed in the backwards direction, whilst the yields for low-momentum pions in both directions are similar.
Presently, however, there is a dearth of experimental measurements for pion production in the backwards direction with 8~GeV protons on a graphite or tungsten target.
\Fig{detector:piYield} shows a measurement of the cross section for pion production with 10~GeV protons on tantalum (which is adjacent to tungsten on the periodic table).

\subsection{Particle Transport through Bent Solenoids}
Both \phaseI and II make use of bent solenoids to help select particles of a particular momentum.
When transported through a bent solenoid, charged particles are dispersed proportionally to their momentum and charge.
This creates a separation between the high and low-momentum components of the beam, such that collimators can selectively remove the high-momentum particles.

Charged particles moving through a straight solenoid follow a helical
trajectory, orbiting a point that moves parallel to the solenoidal axis with a constant velocity fixed by the longitudinal momentum of the particle.
%The frequency with which the particle rotates about this point (the cyclotron frequency or frequency of gyration) is determined by the transverse momentum.
The transverse momentum, on the other hand, determines the frequency with which the particle rotates about this point (the cyclotron frequency or frequency of gyration).

By comparison, if a charged particle moves through a bent solenoid channel, the particle can still be considered to orbit a point, only now 
the motion of that point can be shown to drift vertically, out of the plane of bending.
This drift arises partially due to the gradient introduced to the field by bending the solenoid but also from the 
non-rectilinear coordinate system of the field lines.  
The total drift, $D$, of a particle with mass and charge, $m$ and $q$ respectively,  through a solenoid bent with a fixed radius of curvature, $R$, is given by:
\begin{align}
	D=&\frac{1}{qB}\left(\frac{s}{R}\right)\frac{p^2_\mathrm{L}+0.5p^2_\mathrm{T}}{p_\mathrm{L}}\eqlabel{detector:bent-solenoids:longVstrans}\\
	 =&\frac{1}{qB}\left(\frac{s}{R}\right)\frac{p}{2}\left(\cos\theta + \frac{1}{\cos\theta}\right)\eqlabel{detector:bent-solenoids:pitchAngle}
\end{align}
where $B$ is the magnetic field strength\footnote{Strictly speaking, $B$ is the field strength along the path of the centre of gyration, which is constant for a fixed transverse distance from the focus of the bent solenoid.},
$s$ is the distance travelled through the solenoid, $p$ the momentum of the particle, with $p_\mathrm{L}$ and $p_\mathrm{T}$ its longitudinal and transverse components with respect to the solenoid axis.
%(see appendix \ref{sec:appendix:bent-solenoid} for a full derivation of these equations). %\eq{detector:bent-solenoids:longVstrans} and \eq{detector:bent-solenoids:pitchAngle}).
The pitch angle, $\theta$, is a property of the helical trajectory taken by the particle and defined as:
\begin{equation}
\theta=\tan^{-1}\left(\frac{p_\mathrm{T}}{p_\mathrm{L}}\right)
\end{equation}
The angular dependence of equation \eq{detector:bent-solenoids:pitchAngle} is shown in
\fig{detector:bent-solenoids:angularDependence} where it can be seen that for
angles below 50 degrees the variation in the drift is less than 10\%, such that
the drift is determined almost completely by the momentum for particles below
these angles.
\FigBentSolenoidRelativeDrift

Since the drift is proportional to the momentum of the particle, particles with zero momentum would remain on axis, whilst higher momentum particles, including those of interest (105~MeV electrons in the \phaseII electron spectrometer and around 40~MeV/c muons in the muon beam line), drift to the sides.  
However, an additional vertical component is introduced to the magnetic field.
If the solenoid were straight the axis of a particle's helical trajectory would follow the field line. 
A vertical component would, therefore, cause the trajectories to move upwards with the field line itself, irrespective of the particle's momentum.
The same result is true in a bent solenoid such that a vertical component can be used to 
counter the drift from the bent solenoid field for particles of a specific momentum, which will then remains on-axis.
%select the momentum of particles which remain on axis.

Two techniques have been considered to introduce this vertical component: tilting the solenoid coils themselves, or adding additional dipole coils around the solenoids.
\COMET has chosen to pursue the latter, using a special, proprietary winding technique developed by Toshiba to introduce a vertical component by placing additional conductor around the solenoid coils.
Since the current through these dipole coils can be altered separately to the solenoid coils, this approach has the advantage that the two components can be tuned individually.
This allows for the optimal dipole field to be found during operation running, or for the on-axis momentum and charge to be shifted for background and acceptance studies or searches for other physical observables.

Particles that drift by large amounts will come into contact with the beampipe and are thereby removed from the beam.
Additional collimating material (typically tungsten or copper) can be introduced to more precisely remove particles with undesirable momenta.

%The dynamics of a charged particle in a magnetic field is determined by the Lorentz equation:
%\begin{equation}
%\vec{F}=\frac{q}{m}\vec{p}\times\vec{B}
%\end{equation}
%where $q$, $\vec{p}$ and $m$ are the particle's charge, momentum and mass respectively, and $\vec{B}$ is the magnetic field.
%In a uniform magnetic field where all field lines are parallel, clearly the motion of the particle follows a helix whose axis is parallel to the field and
%with a helical pitch-angle given by:
%\begin{equation}
%\theta=\tan^{-1}\Big(\frac{P_\mathrm{T}}{P_\mathrm{L}}\Big)
%\end{equation}
%where $P_\mathrm{T}$ and $P_\mathrm{L}$ are respectively the transverse and longitudinal components of the momentum with respect to the magnetic field.
%Such a field can be realised to a high-precision by a cylindrical solenoid coil.
%
%If instead one were to bend a solenoid coil, so that it's axis describes a circular arc, two effects are introduced:  firstly, the uniformity of the field is changed
%such that a higher magnetic field is found on the inside of the bend, and secondly the field lines also bend.
%Each of these changes causes the motion of the particle to deviate from that of a straight solenoid.
%Whilst one can think of the particle as following a helix around the field lines still, the centre of this helix can be shown to drift out of the plane of the bending.
%Firstly, the radial gradient introduced to the field causes a drift which is proportional to the transverse momentum of the particle.
%Secondly, the centrifugal pseudo-force as the particle tracks the now cylindrical field lines, creates a force that acts perpendicularly to the magnetic field.
%Since the field lines follow the solenoidal axis, this also produces a vertical drift, proportional to the longitudinal momentum, however.
%
%Taken together, the result is a vertical drift with a velocity given by:
%\begin{equation}
%\end{equation}

Bent solenoids are used in \COMET for both \phaseI and II to disperse high-energy muons and pions in the muon
beam, and as a spectrometer system for electrons coming
from the stopping target in \phaseII, which will both be described in more
detail below.

\subsection{Stopping Target Material and Beam Pulsing}
\label{sec:stop-tgt}
\FigMuonNuclearParams

The combination of using backwards-going pions and the long, bent-solenoid transport channel is already effective at reducing potential background issues.
In addition to these however, there is one further method which helps both to reduce beam-related backgrounds and improve the detector occupancy and reconstruction requirements:
the use of a pulsed proton beam with a relatively light stopping target.

Since the signal process is coherent, its cross section grows roughly as the square of the number of nucleons (or protons, depending on the model)%
\footnote{Although this growth is offset by the normalisation to the capture rate which is typically treated as incoherent so that it grows linearly with the number of nucleons.
The overall conversion rate for lighter elements is roughly proportional to the atomic number.}
until the muon is contained almost completely within the nucleus, at which point the rate levels off.
It is therefore desirable to use a high-Z target in order to increase the probability of conversion and indeed \sindrumII used both lead and gold targets, with its most stringent limit set on a gold target~\cite{sindrum2006}.

However, as the nucleus gets larger, the lifetime of the muonic atom falls steeply due to the increase in the nuclear capture rate.
This is illustrated in \fig{detector:mu-nucl-params} where it can be seen that for elements heavier than iron ($Z>26$) the muon lifetime is less than 200~ns.
The COMET production target and beamline produces a beam flash that lasts for about 200~ns after the arrival of a proton.
This means that, for targets heavier than iron, timing information is not a useful parameter to distinguish particles in the beam from electrons coming from stopped muons.

Whilst these are the two dominant factors in deciding the target material, other factors like the mechanical stability, cost, isotopic purity and the stability of the daughter nuclei following muon capture on the target must also be considered.
Accordingly, titanium and aluminium are considered the two most viable target materials.  
Titanium, in which the muon lifetime is about 330~ns, would be considerably harder to measure \mueconv so at this stage the COMET experiment is focussed on using aluminium where the muon lifetime is about 864~ns~\cite{Suzuki1987}.

\FigTimingSchematic

The J-PARC accelerator has buckets separated by 550~ns, although separations of multiples of this number can, in principle, be achieved.
For COMET running the intention is to fill every second bucket so that pulses are separated by 1.17~$\mu$s.
\Fig{detector:timing-schema} shows the beam timing schematically.  
A window from about 700 to 1100~ns after the proton beam arrival is then used to look for signal events, by which time most of the beam flash should have passed whilst signal events remain probable.

Having a well-defined bunch structure is crucial for this scheme to work. 
Protons arriving in between bunches would produce (a fragment of) beam flash that could include high-energy muons or pions.   
If these produce signal-like electrons that would a background source that the timing window would be unable to remove.
The extinction factor is used to quantify the probability of protons arriving out of time, and is given by:
\begin{equation}
	R_\mathrm{Extinction}=\frac{N(p~\mathrm{between~bunches})}{N(p~\mathrm{per~bunch})}.
\end{equation}
Original background predictions were made assuming $R_\mathrm{Extinction}$ was around $10^{-9}$~\cite{CDRphase2} (about 1 out-of-time proton for every 7 \phaseI bunches) although recent measurements have been able to demonstrate extinction at a level of $10^{-12}$~\cite{TDR2016} (about 1 out-of-time proton for every 7100 \phaseI bunches).

The bunch structure is initially defined by the linear accelerator (linac) at J-PARC which accelerates protons up to 600~MeV.
The \ac{RCS} then takes these protons up to 3~GeV where up to two buckets can be stored, although for COMET only one bunch at a time will be filled.
The protons are then injected into the \ac{MR} which accelerates them up the final energy of 8~GeV and is capable of storing up to 9 buckets at once.
Using the linac chopper alone would not be sufficient to produce the desired extinction factor since stray protons tend to drift into the unfilled buckets.
Achieving the high-extinction factor then is possible only by using the injection kicker from the \ac{RCS} to the \ac{MR} in a `double-kick' mode.
The kicker excitation length is set to two buckets (so that the \ac{RCS} is completely emptied into the \ac{MR}).  
The kicker is then activated again immediately after the first filled bunch has performed a complete rotation of the \ac{MR} such that protons that had diffused into the second bunch of the \ac{RCS} are now kicked away.
Thus only every second bucket in the \ac{MR} is filled and all other buckets are kept empty.

\section{\COMET \phaseI}
% - Measurement goals
% - StrECAL detector
% - CyDet detector
\phaseI will see the construction of the \COMET hall, the production target capture solenoids, the first 90 degrees of the bent muon transport solenoid, and the detector solenoid.  
The beamline is shown schematically in \fig{detector:PhaseI:setup} where the two interchangeable detector systems can also be seen.

There are two key goals to \phaseI:
\begin{enumerate}
	\item measure \mueconv at a \acf{ses} of \sensePI, and
\item prepare for \phaseII by measuring the beam profile, particle yields and background rates, and prototype the detector technology.
\end{enumerate}
Since the dynamics of bent solenoids are complicated, it is important to study the beam as close to the production target as possible.
However, due to the high-radiation environment around the production target, the detector and electronics cannot be placed too close and must be well shielded.
\phaseI will therefore measure the beam after the first 90 degrees of bent solenoid with the same detector system to be used in \phaseII, namely the \ac{StrECAL} detector---a series of Straw Tracker stations  followed by an ECAL
all sitting in the beam.  

However, since the StrECAL detector will be hit by the full force of the muon beam, it would not be feasible to conduct a \mueconv search using this detector.
As such, for \phaseI, a second detector called the \ac{CyDet} will be used for this purpose.
The \ac{CyDet} uses a \ac{CDC} to reconstruct the trajectories of charged particles and a pair of Cherenkov and Scintillation counters (one upstream and one down) to trigger the read-out of the system.
The \ac{CyDet} escapes the issue of the beam flash that the \ac{StrECAL} would face in \phaseI since only the outer region is instrumented.
Because the detector sits in a 1~T solenoid field (and both the detector and solenoid are co-axial), particles follow helical trajectories with the radius of gyration determined by the transverse momentum of the particle.
The beam is introduced in the centre and typically remains in an envelope of 15~cm whilst the stopping target sits in the centre of the detector with a radius of 10~cm.
As such the detector itself is geometrically blind to charged particles in the beam and electrons coming from muon \ac{DIO} in the target with momentum less than 60~MeV/c which make up the majority of the \ac{DIO} spectrum.
To reconstruct the longitudinal position of the particle's trajectory an all-stereo configuration is used in the Cylindrical Drift Chamber, where each layer of wires is rotated in the opposite direction to the previous layer by an angle of 4\degree with respect to the solenoid axis.

Because the \phaseI detector sits much closer to the stopping target than at \phaseII, there is greater exposure to hadrons emitted following nuclear capture of the stopped negative muons, such as protons, deuterons, alpha particles and so on.
Despite being emitted with kinetic energies of a few tens of MeV, momenta above 60~MeV/c are readily achieved given the large mass of these particles.
For similar reasons these particles are typically very heavily ionising so, if left unchecked, could easily dominate the occupancy of the \ac{CDC}.
The \alcap experiment has shown that for muon capture on Al-27 nuclei, the emission of a proton occurs for about 3\% of every muon capture~\cite{NamThesis}.
At this level, it is believed that no specific shielding is required beyond the carbon inner wall of the \ac{CDC} needed to contain the gas mixture.
\FigPhaseI

Four layers of scintillation bars will surround the outside of the detector to provide a veto for cosmic ray events.
The most dangerous event would be a high-energy muon reaching the target and decaying to a 105~MeV electron which is then detected. 
Dedicated cosmic runs will be performed prior to operation with a beam in order to understand the flux of cosmic muons.

\section{\COMET \phaseII}
\FigPhaseII

COMET \phaseII will be the final stage of the experiment and should achieve the overall goal of measuring \mueconv with an \ac{ses} of \sensePII.
It will extend the muon beamline built for \phaseI by an extra 90\degree, and add two extra solenoid sections: one to hold the stopping target, and a second 180\degree bent solenoid with a large aperture of 60~cm radius.
This layout is shown in \fig{detector:PhaseII:setup}.
The bent solenoid after the stopping target is primarily there to remove the low-energy \ac{DIO} electrons which otherwise could significantly increase the hit rate in the detector.
The final detector system for \phaseII will use the \ac{StrECAL} from \phaseI but probably with thinner diameter straw tubes, thinner straw material, and more tracking stations in order to improve the energy resolution.

%The stopping target itself has typically been designed as thin disks of aluminium, followed by a beam blocker.  
%Even including the growth of the beam envelope as it passes from the 3~T field in the bent muon transport solenoids to the 1~T field of the bent electron spectrometer solenoid, the beam blocker removes all line-of-sight between the entrance of the electron spectrometer and the muon transport solenoids so that most of the beam flash is prevented from reaching the detector.
%The tapering of the field occurs almost completely across the target itself with the intention that signal electrons heading initially upstream will be magnetically mirrored back towards the detector.

As for \phaseI, an active cosmic ray veto will prevent triggering on events caused by cosmic muons.
At least the detector solenoid will be covered, but it is likely that both the spectrometer and the stopping target area are also contained in the veto.
In \phaseI, with the target surrounded by the detector, there is a degree of self-shielding against cosmic events. 
In \phaseII, however, this will not be the case since the target and detector are widely separated.

\section{Key Sub-component Descriptions}
The following sections describe the sub-components and beamline sections for both \phaseI and \phaseII in more detail.
For \phaseI, parameters come from the latest geometry which is close to that of the TDR~\cite{TDR2016} but with some improvements.
For \phaseII, the description is largely based on the design as laid out in the 2009 CDR~\cite{CDRphase2}.

% - Proton beam energy
% - Proton beam timing
% - Production target and capture system
% - Bent Transport system
% - Stopping target
% - Detector system

\newcommand{\Component}[6]{
\def\tempshort{#2}
\ifx\tempshort\empty\def\printshort{}\else\def\printshort{(#2)}\fi
\def\tempmag{#3}
\def\tempconstr{#4}
\def\printmag{Magnet names: \emph{#3}}
\def\printconstr{Constructed by: \emph{#4}}
\def\printpars{{\small #6}}
\ifx\tempmag\empty%
	\def\lineone{\printconstr. }%
	\def\linetwo{}%
\else%
	\def\lineone{\printmag}%
	\def\linetwo{. \printconstr}%
\fi%
\subsubsection*{#1 \printshort}\vspace{-3ex}%
%\begin{tabular}{L{0.2\textwidth}L{0.7\textwidth}}
%\lineone&\multirow{2}{0.8\textwidth}{#6}  \\
%\linetwo& \\
%\end{tabular}
\lineone \linetwo\\
#5
\def\tempparams{#2}
\ifx\tempparams\empty\else
\vspace{-2ex}
\begin{multicols}{2}
\begin{itemize}
	\setlength\itemsep{-1em}
#6
\end{itemize}
\end{multicols}
\vspace{-5ex}
\fi
}
%\newcommand{\Component}[6]{
%\def\tempshort{#2}
%\ifx\tempshort\empty\def\printshort{}\else\def\printshort{(#2)}\fi
%\def\tempmag{#3}
%\def\tempconstr{#4}
%\def\printmag{Magnet names: \emph{#3}}
%\def\printconstr{Constructed: \emph{#4}}
%\def\printpars{{\small #6}}
%\ifx\tempmag\empty%
%	\ifx\tempconstr\empty%
%		\def\lineone{\printpars}%
%		\def\linetwo{}
%		\def\linethree{}
%	\else%
%		\def\lineone{\printconstr}%
%		\def\linetwo{\printpars}
%		\def\linethree{}
%	\fi%
%\else%
%	\def\lineone{\printmag}%
%	\ifx\tempconstr\empty%
%		\def\linetwo{\printconstr}
%		\def\linethree{}
%	\else%
%		\def\linetwo{\printconstr}%
%		\def\linethree{\printpars}
%	\fi%
%\fi%
%\begin{tabular}{L{0.35\textwidth}|L{0.6\textwidth}}
%\multicolumn{2}{L{\textwidth}}{{\bf#1 \printshort}}\\
%\lineone& \multirow{3}{0.6\textwidth}{#5}  \\
%\linetwo& \\
%\linethree& \\ 
%\end{tabular}
%}
\Component{Production Target Section}
{ProdTgtSec}
{CS, MS, TS1}
{\phaseI, but target changed for \phaseII}
{Super-conducting solenoid containing the pion production target.
  The proton beam strikes the production target, which produces a range of particles, principally pions.
  The solenoidal fields captures the backwards-going pions and delivers them into the transport solenoids.
  This contains a significant amount of shielding to protect the magnet coils from overheating in the high-radiation environment.
  The target itself consists of a single, solid, cylindrical rod.
  In \phaseI the target will be made of graphite, whilst in \phaseII it will be tungsten.
 Since graphite has a larger interaction length, the target in \phaseI will likely be longer than in \phaseII.
Between the target and the solenoid coils is a large volume of tungsten and copper shielding to protect the coils from radiation damage and overheating.
In addition, in the forward proton direction sits a large iron beam dump, which will absorb the protons that miss the target and forward-produced pions and other particles.
}
{\item Min. shielding radius = 12~cm%
 \item Target solenoid aperture = 44~cm%
 \item Matching sole.\ aperture = 18~cm%
 \item Target length (\phaseI) = 60~cm%
 \item Target length (\phaseII) = 16~cm (2009 optimisation)%
 \item Field strengths = 5~T at the target, 3~T along matching solenoids}

\Component{Torus1}
{Tor1}
{TS2}
{\phaseI }
{Bent solenoid section containing antiproton absorber foils and muon beam collimators.  Special dipole coils are placed around the normal solenoid coils to add a tuneable vertical component to the field.  During \phaseI, the detector solenoid will sit immediately after the Torus1, although with a small additional matching coil.  In \phaseII this will be replaced with the small TS3 coil to connect it to Torus2.}
{\item Apperture size = 18.5~cm
 \item Bending radius = 3~m 
\item Field strengths = 3 T}

\Component{Torus2}
{Tor2}
{TS4}
{\phaseII}
{Very similar in design to Torus1, although possibly with different collimator placement and designs. The dipole field along this half of the bent muon transport solenoids might well be different to the dipole along the first half (Torus1).}
{\item Apperture size = 18.5~cm
 \item Bending radius = 3~m
 \item Field strengths = 3 T}

\Component{Stopping Target Section}
{StopTgtSec}
{TS5, ST}
{\phaseII}
{The straight section of solenoid housing the muon stopping target.  The
electromagnetic field reduces dramatically around the mid-point of the
StopTgtSec to improve signal acceptance.  The stopping target itself will
consist of 200~um disks made of pure aluminium and then followed by a
tungsten or aluminium beam blocker.  There should be no line of site between
the exit of the Torus2 and the entrance to the downstream solenoid, the
Electron Spectrometer.  This removes high-energy particles in the muon beam.
The reduction in the field strength in this region is used to improve the
signal acceptance by magnetically mirroring back signal electrons that initially head
upstream from the target.
}
{\item Apperture size = 18.5~cm at the entrance, stepped up to  61~cm by the exit
 \item Field strengths = 3 T at the entrance, tapering to about 1~T at the exit}

\Component{Electron Spectrometer}
{ElSpec}
{ES}
{\phaseII}
{The 180\degree bent solenoid used to prevent the very low energy particles reaching the detector.  Also useful for removing backgrounds due to gammas and neutrons coming from the stopping target.  The magnetic field in the Electron Spectrometer is much weaker at close to 1~T.}
{\item Apperture size = 60~cm
 \item Field strengths = 1 T along beam-axis
 \item Bending radius = 2~m}

\Component{Detector Solenoid}
{DetSol}
{DS}
{\phaseI, but possibly extended for \phaseII}
{The final straight solenoid section that houses the actual detector.  This will be re-used in \phaseII after \phaseI finishes, although additional coils may be added to extend the solenoid to house additional straw tracker stations.  }
{\item Apperture size = 96~cm
 \item Field strength = 1 T}

\Component{StrawTracker + ECAL Detector}
{StrECAL}
{}
{\phaseI, upgraded for \phaseII}
{Detector system used in \phaseII to measure conversion electrons.   In \phaseI will be used to measure beam properties.  Consists of 5 Straw Tracker stations (in \phaseI, with possibly more in \phaseII), each transverse to the beam.  Each station consists of 4 perpendicular layers of straw tubes.  The LYSO-based ECAL will measure particle energies with 5\% resolution and serve primarily for a trigger and to support PID.}
{\item Straws per layer =120
 \item layers per station  = 4 (2 X, 2 Y)
 \item Number of stations = 5
 \item Straw length = 69.2 to 130~cm
 \item Wire radius = 10~micron
 \item Straw outer radius = 4.9~mm 
 \item Straw material = Aluminised mylar
 \item No. of ECAL crystals = 1920 Crystal
 \item Crystal dimensions = $2\times2\times12$~cm
 \item Crystal material = LYSO }

\Component{Cylindrical Detector}
{CyDet}
{}
{Only used in \phaseI}
{The primary detector for \phaseI to measure conversion electrons at 200 keV resolution.  Consists of a cylindrical drift chamber arranged coaxially with the beam and stopping target.  Wires in the drift chamber are twisted in opposite directions on alternating layers to allow for stereoscopic reconstruction of the longitudinal component of a particle's trajectory.  In addition to the drift chamber, triggering hodoscopes and scintillation bars at the upstream and downstream ends of the detector provide a timestamp and trigger decision, although this will likely be supplemented by a track trigger using Drift Chamber information.}
{\item No. of Layers = 20 (including 2 guard layers)
 \item No. of Field wires = 14562
 \item No. of Sense wires = 4986
 \item Field wire = 126~micron Al
 \item Sense wire = 25~micron Au plated W }

\Component{Cosmic Ray Veto}
{CRV}
{}
{\phaseI and upgraded and extended for \phaseII }
{An active veto against cosmic muons that enter the detector.  Formed from four layers of scintillating strips, read-out via wavelength shifting fibres.  The CRV surrounds the detector solenoid on all sides and above.  Layers of concrete and iron are contained between the detector solenoid and the CRV in order to protect the CRV from the high neutron fluxes from the beam and stopping target. }
{\item Number of layers = 4 per side
 \item Total number of strips = 3816  
 \item Strip material = Polystyrene scintillator 
 \item Shielding between detector = layers of concrete, iron, polyethylene, and lead }

\Component{Concrete and Iron shielding  }
{}
{}
{\phaseI}
{Experiment hall shielding to capture and constrain the neutron and gamma radiation from the beam, principally around the production target section and proton beam dump.
  Air-tight interlocking concerete and iron blocks will surround the production target region and experiment hall.
  The detectors themselves will be isolated from the production target by concrete and iron blocks, with the only connection being the hole through which the muon transport solenoids pass.
 }
{}


\section{Schedule and Status}
\FigSchedule
The overall schedule for the COMET experiment is shown in \fig{detector:schedule}.
\phaseI is due to start data taking in \ac{JFY} 2018 and construction and development is well underway.

\FigStatusFacility
With regards to the facility, the building that will house the experiment is now finished, sitting to the side of the existing Hadron Hall at \ac{JPARC}.
Cooling and power supplies are being installed and the shielding for the concrete hatch area is being produced.
In the mean time the development of the new beamline to extract protons from the \ac{MR} and deliver them to the COMET area is being installed.
In particular, the Lambertson magnet which directs the protons towards COMET rather than the existing Hadron Hall has been built.
For the muon beamline, the \phaseI section of the bent muon transport solenoids has been fabricated and installed, and is now under commissioning studies.
Construction of the detector solenoid has also begun with the capture solenoids around the production target soon to begin.
A selection of photographs that show the construction of the facility and installation of the bent transport solenoid are show in \fig{detector:setup:facility}.

\FigStatusStrECAL
Much of the recent activity for the collaboration has been on the design and construction of the detector systems.
Beam tests to understand the performance and resolution of prototype ECAL crystals, straw tubes, and the \ac{CDC} have taken place.
For the \ac{StrECAL}, production of all 2500 \phaseI straws has been completed and procurement of the \ac{LYSO} crystals for the ECAL is under way with some 200 or so crystals already purchased.
Ageing tests of the straw tubes are on-going with straws being held for an extended duration under pressure and tension at KEK.
Beam-test data is being analysed to understand the position resolution for a given straw and the energy resolution of the ECAL.
In the case of the ECAL, an energy resolution better than 5\% has already been shown for 105~MeV electrons.
\Fig{detector:status:StrECAL} shows photographs of the prototypes and beam test set-ups of the Straw Tube Tracker and the ECAL.

\FigStatusCyDet
In the meantime the full \ac{CDC} has been strung, with some 20,000 wires---about 15,000 field wires of 125~$\mu$m thickness, and about 5,000 sense wires of 25~$\mu$m diameter---being inserted.
Every wire has had its tension checked using a vibrational resonance method, which showed some 90 wires were outside of design tolerances.  These have since been replaced.
In June 2016, the inner wall of the CDC was successfully inserted, completing the CDC construction, so that leak tests can begin shortly.  
\Fig{detector:status:CyDet} shows photographs from the stringing of the \ac{CDC}.
In parallel, cosmic ray tests have been used to study the performance of CDC prototypes and analysis of the data is under way to deduce the X-T curve for the CDC cells.

In the less-tangible realm that is software and simulation, in April 2015 the offline software reached its first stable release, and has since been used to perform three large-scale Monte Carlo productions of \phaseI.
%The most recent production saw the simulation of around $10^{10}$~\ac{POT}, equivalent to about 20,000 bunches.  
Reconstruction algorithms, including track finding and fitting in the \ac{CDC} are under development and techniques to perform \ac{PID} using the \ac{StrECAL} in the \phaseI beam are also under development.
More discussion on the software and simulation can be found in the next chapter.
