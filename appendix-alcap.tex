\chapter{Summary of the AlCap Experiment}
\sectlabel{alcap}
The following was originally submitted for the proceedings of NuFact `15 and has been published online from the meeting's home page at: \url{https://indico.fnal.gov/internalPage.py?pageId=2&confId=8903}.
Although peer reviewed, it has not appeared in print, and so I reproduce it here in its entirety.


\begin{abstract}
	The \alcap experiment studies the emission products following muon capture on an aluminium nucleus.
	Such a measurement is important in the context of the up-coming muon-to-electron conversion experiments, COMET and Mu2e, which will
	both use an aluminium stopping target.  Despite this, and the potential nuclear and astrophysical implications, 
	the existing range of measurements is incomplete, with the majority of measurements on proton and neutron emissions already some 40 years old.

\alcap first ran in 2013, and will have run twice more by the end of 2015. 
It is a joint effort by the Mu2e and COMET collaborations.
\end{abstract}

\newenvironment{ruledtabular}{}{}

\subimport{nufact15-alcap-proceedings/}{figures}
\subimport{nufact15-alcap-proceedings/}{introduction}
\subimport{nufact15-alcap-proceedings/}{run2013}
\subimport{nufact15-alcap-proceedings/}{run2015a}
\subimport{nufact15-alcap-proceedings/}{run2015b}
\subimport{nufact15-alcap-proceedings/}{summary2}
