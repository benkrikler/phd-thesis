\chapter{Drifts in a Bent Solenoid}
\label{sec:appendix:bent-solenoid}
Solenoidal fields are a simple and reliable way to produce uniform and rectilinear magnetic fields.
The dynamics of a charged particle's motion through a straight solenoid are, therefore, straightforward to understand.

In a bent solenoid, however, motion of charged particles is considerably more complicated.
Given the use and importance of bent solenoids for the COMET experiment, this appendix derives the equations of motion for charged particles in a bent solenoid.

\section{Uniform Solenoidal Field}
In a standard solenoid the magnetic field is uniform and parallel to axis of solenoid.
The Lorentz force for charged particle motion:
\begin{equation}
\frac{d\vec{p}}{dt}=q\vec{v}\times\vec{B},
\end{equation}
where $\vec{p}$ and $\vec{B}$ are the momentum and magnetic field vectors, and $q$ and $m$ are the electric charge and mass of the particle.
Since it is the cross product between the momentum and magnetic field which determines the magnitude and direction of the resulting force, in a uniform magnetic field only the momentum transverse to the magnetic field is affected.
Momentum parallel to the field lines is unchanged and, since in the transverse plane one observes circular motion, the net result is a trajectory that spirals around the field line.
The pitch angle of this helix, $\theta$, is determined by the ratio between the longitudinal and transverse momenta: $\tan(\theta)=P_\textrm{T}/P_\textrm_{L}$, where the `L' and `T' subscripts indicate longitudinal and transverse components of the momentum, respectively.

The frequency, $\omega$, and radius, $a$, with which the particle rotates about the field lines are defined by:
\begin{align}
a=&\frac{\vec{p}\times\vec{B}}{qB^2} = \frac{p_\mathrm{T}}{qB} \\
\vec{\omega}=&\frac{q\vec{B}}{\gamma{}m}
\end{align}
where $\gamma$ is the Lorentz boost factor.

\section{Field in a Bent Solenoid}
\begin{figure}[t]
\includegraphics[width=0.8\textwidth]{figs/appendix/BentSolenoid_GradBField2.pdf}
\caption{\figlabel{app:bentSol:sketch}
The magnetic field in the bent solenoid has a gradient parallel to the large radius of the torus ($r$).
The field reduces inversely proportionally to the transverse distance from the centre of the torus.
These properties can be deduced by Ampere's law, using a circuit as illustrated by the red line.
}
\end{figure}

Producing a cylindrical solenoid  channel can be imagined as directly bending that of a straight solenoid.
By symmetry it can be seen that any gradient introduced to the magnetic field can only be radially, in the plane of the bending.
Further, by considering Ampere's law with a current loop as illustrated by \fig{app:bentSol:sketch}, it can be seen that the variation in the field is given by:
\begin{equation}
\frac{\vec{\nabla B}}{B}=\frac{1}{r}\hat{r}
\end{equation}

\section{Drift Calculation}
There are two sources of drift in a bent solenoid: the gradient in the field, and the centrifugal force arising from the circular coordinate system needed to describe the field lines.
The two can be treated separately in the sense that the motion of a particle moving through a field with straight field lines but with a transverse gradient given by $\nabla B/ B \propto 1/r$ would be described by an equation of motion equivalent to that from the first source of drift in the bent solenoid system. 
Similarly, a system with a uniform field but field lines that follow circular paths would exhibit drift equivalent to the second component mentioned above.

\subsection{Gradient Drift}
``Grad-B'' drift is well described in text books, but in the interest of completeness a short derivation shall be given here.
If one projects the motion of the particle to the transverse plane---considering therefore only the transverse momentum---one can achieve an intuitive understanding of this drift.
On one side of this projection the field strength will be larger, and thus the gyroradius will be smaller.
On the other side, where the field is weaker, the gyroradius will be larger so that with every full turn the particle moves upwards, perpendicularly to the gradient of the field.

To put this more quantitatively, we treat
the drift arising due to the gradient in the field as a perturbation of the motion of the particle in a uniform solenoidal field, on the basis that the radius of gyration is small compared to the variation in the field.
The total velocity $\vec{V}$, is given by:
\begin{equation}
\vec{V}=\vec{v}+\vec{v_g},
\eqlabel{velocity-grad}
\end{equation}
where $\vec{v}$ is the unperturbed velocity of the particle in the transverse plane, and $\vec{v_g}$ is the velocity arising due to the gradient in the field.

Treating the field as a Taylor expansion about the centre-line of the unperturbed helical trajectory:
\begin{equation}
\vec{B}(\vec{r})=\vec{B_0}+\left.(\vec{r}\cdot\nabla)\right|_{\vec{r}=0}\vec{B}+\ldots
\eqlabel{B-field-taylor}
\end{equation}
where $\vec{r}$ is the displacement from the centre of the unperturbed helix.
Substituting equations \eq{velocity-grad} and \eq{B-field-taylor} into the Lorentz force, gives:
\begin{align}
m\frac{d(\vec{v}+\vec{v_g})}{dt}=&q(\vec{v}+\vec{v_g})\times\left(\vec{B_0}+\left.(\vec{r}\cdot\nabla)\right|_{\vec{r}=0}\vec{B}\right) \\
%=&q\vec{v}\times\vec{B_0} \\
%&+q\left(\vec{v}\times\left.(\vec{r}\cdot\nabla)\right|_{\vec{r}=0}\vec{B}+\vec{v_g}\times\vec{B_0}\right)
\end{align}
so that to first order, the perturbing velocity is given by:
\begin{equation}
\frac{d\vec{v_g}}{dt}=\frac{q}{m}\left(\vec{v}\times\left.(\vec{r}\cdot\nabla)\right|_{\vec{r}=0}\vec{B}+\vec{v_g}\times\vec{B_0}\right)
\end{equation}
Since we are only interested in steady-state solutions where $\dot{v_{g}}$ is close to zero, the above equation gives:
\begin{equation}
\vec{v_g}=\frac{\vec{B_0}\times\left(\vec{v}\times\left.(\vec{r}\cdot\nabla)\right|_{\vec{r}=0}\vec{B}\right)}{B^2_0}
\eqlabel{app:bentSol:gradB:fullVector}
\end{equation}

For simplicity, we let the gradient in the field be parallel to $x$, and set the magnetic field parallel to $z$, remembering that for grad-B drift we only need to work in the transverse plane.
Thus we rewrite 
$\left.(\vec{r}\cdot\nabla)\right|_{\vec{r}=0}\vec{B}$ as $x(\partial B/\partial x) \vec{\hat{z}}$.
Since $\vec{v}$ is the unperturbed velocity, we can write it as:
\begin{equation}
\vec{v}=a\left(-\sin(\omega t)\vec{\hat{x}}+\cos(\omega t)\vec{\hat{y}}\right).
\end{equation}
Similarly, since the horizontal projection of the particle's path will be identical to the unperturbed path, we can set $x=\cos(\omega t)$.
By inserting all of these into equation~\eq{app:bentSol:gradB:fullVector} we obtain:
\begin{align}
\vec{v_g}=&v_\textrm{T}
\begin{pmatrix} -\sin(\omega t)\\ \cos(\omega t)\\ 0 \end{pmatrix}
\times a
\begin{pmatrix}0\\0\\\cos(\omega t)\end{pmatrix}\frac{\partial B}{\partial x}\\
=&V_\textrm{T}\begin{pmatrix}0\\0\\\cos(\omega t)\end{pmatrix}\frac{\partial B}{\partial x}\\
\end{align}

