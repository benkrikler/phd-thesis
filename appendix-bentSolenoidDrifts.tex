\chapter{Drifts in a Bent Solenoid}
\label{sec:appendix:bent-solenoid}
The Lorentz force:
\begin{equation}
\frac{d\vec{p}}{dt}=\frac{q}{m}\vec{p}\times\vec{B}
\end{equation}

\section{Uniform Solenoidal Field}
B field is uniform and parallel to axis of solenoid.
Define the Larmor frequency, $\omega$, and radius, $a$, as:
\begin{align}
a=&\frac{\gamma m \vec{v}\times\vec{B}}{qB^2} = \frac{p_\mathrm{T}}{qB} \\
\omega=&\frac{qB}{m}
\end{align}

\section{Field in a Bent Solenoid}
Producing a cylindrical solenoid  channel can be imagined as directly bending that of a normal uniform and linear one.
By symmetry it can be seen that any gradient introduced to the magnetic field can only be radially, in the plane of the bending.
Further, by considering Ampere's law with a current loop in the plane of bending formed by two radial straight lines (with length $|r-R|<L$, where $R$ and $L$ are the bending and aperture radii of the solenoid channel) and an arc,
it can be seen that the variation in the field is given by:
\begin{equation}
\frac{\vec{\nabla B}}{B}=\frac{1}{r}\hat{r}
\end{equation}
%\CHECK{is this accurate?}
\CHECK{Sketch or figure?}

\section{Drift Calculation}
There are two sources of drift in a bent solenoid: the gradient in the field, and the centrifugal force arising from the circular coordinate system needed to describe the field lines.
The two can be treated separately in the sense that the motion of a particle moving through a field with straight field lines but with a transverse gradient given by $\nabla B/ B \propto 1/r$ would be described by an equation of motion equivalent to that from the first source of drift in the bent solenoid system. 
Similarly, a system with a uniform field but field lines that follow circular paths would exhibit drift equivalent to the second component mentioned above.

\subsection{Gradient Drift}
``Grad-B'' drift is well described in text books, but in the interest of completeness a short derivation shall be given here.
The drift arising due to the gradient in the field can be treated as a perturbation of the motion of the particle in a uniform solenoidal field.
The total velocity $\vec{V}$, is given by:
\begin{equation}
\vec{V}=\vec{v}+\vec{v_g},
\eqlabel{velocity-grad}
\end{equation}
where $\vec{v}$ is the unperturbed velocity of the particle in the transverse plane, and $\vec{v_g}$ is the velocity arising due to the gradient in the field.

Treating the field as a Taylor expansion:
\begin{equation}
\vec{B}(\vec{r})=\vec{B_0}+\left.(\vec{r}\cdot\nabla)\right|_{\vec{r}=0}\vec{B}+\ldots
\eqlabel{B-field-taylor}
\end{equation}
and substituting equations \eq{velocity-grad} and \eq{B-field-taylor} into the Lorentz force, gives:
\begin{align}
m\frac{d(\vec{v}+\vec{v_g})}{dt}=&q(\vec{v}+\vec{v_g})\times\left(\vec{B_0}+\left.(\vec{r}\cdot\nabla)\right|_{\vec{r}=0}\vec{B}\right) \\
%=&q\vec{v}\times\vec{B_0} \\
%&+q\left(\vec{v}\times\left.(\vec{r}\cdot\nabla)\right|_{\vec{r}=0}\vec{B}+\vec{v_g}\times\vec{B_0}\right)
\end{align}
so that to first order, the perturbing velocity is given by:
\begin{equation}
\frac{d\vec{v_g}}{dt}=\frac{q}{m}\left(\vec{v}\times\left.(\vec{r}\cdot\nabla)\right|_{\vec{r}=0}\vec{B}+\vec{v_g}\times\vec{B_0}\right)
\end{equation}
Since we are only interested in steady-state solutions where $dot{v_{g}}$ is close to zero, the above equation gives:
\begin{equation}
\vec{v_g}=\frac{q}{m}\frac{\vec{B_0}\times\left(\vec{v}\times\left.(\vec{r}\cdot\nabla)\right|_{\vec{r}=0}\vec{B}\right)}{B^2_0}
\end{equation}
which by considering the form 	after time averaging becomes:
\begin{equation}
\langle\vec{v_g}\rangle_t=\frac{\vec{B_0}\times\left(\vec{v}\times\left.(\vec{r}\cdot\nabla)\right|_{\vec{r}=0}\vec{B}\right)}{B^2_0}
\end{equation}
